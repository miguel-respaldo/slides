% ex: ts=2 sw=2 sts=2 et filetype=tex
% SPDX-License-Identifier: CC-BY-SA-4.0

\section{Instalación de Thonny}

\begin{frame}[c]{¿Qué es Thonny?}
    \begin{columns}
        \column{0.5\textwidth}
        Thonny es un \textbf{entorno de desarrollo integrado} gratuito y
        de código abierto para Python diseñado para principiantes.
        \vspace{\baselineskip}

        \href{https://thonny.org}{https://thonny.org}

        \vspace{\baselineskip}
        Descarguen la \underline{primera opción} para Windows o Mac

        \column{0.5\textwidth}
        \begin{center}
            \includegraphics[scale=0.3]{lfs/thonny-00.png}
        \end{center}
    \end{columns}
\end{frame}

\begin{frame}[c]{Selecciona el modo de instalación}
    \begin{columns}
        \column{0.5\textwidth}
        \begin{center}
            \includegraphics[scale=0.8]{lfs/thonny-01.png}
        \end{center}
        \column{0.5\textwidth}
        \begin{itemize}
          \item Selecciona la opción \textbf{Install for me only}
        \end{itemize}
    \end{columns}
\end{frame}

\begin{frame}[c]{Bienvenida al uso de Thonny}
    \begin{columns}
        \column{0.5\textwidth}
        \begin{center}
            \includegraphics[scale=0.6]{lfs/thonny-02.png}
        \end{center}
        \column{0.5\textwidth}
        \begin{itemize}
          \item Selecciona \textbf{Siguiente/Next}
        \end{itemize}
    \end{columns}
\end{frame}

\begin{frame}[c]{Acuerdos de Licencia}
    \begin{columns}
        \column{0.5\textwidth}
        \begin{center}
            \includegraphics[scale=0.6]{lfs/thonny-03.png}
        \end{center}
        \column{0.5\textwidth}
        \begin{itemize}
          \item Si aceptan los acuerdos de licencia, selecciona
            \textbf{Siguiente/Next}
        \end{itemize}
    \end{columns}
\end{frame}

\begin{frame}[c]{Selección de lugar de instalación}
    \begin{columns}
        \column{0.5\textwidth}
        \begin{center}
            \includegraphics[scale=0.6]{lfs/thonny-04.png}
        \end{center}
        \column{0.5\textwidth}
        \begin{itemize}
          \item Pueden cambiar el destino de la instalación si gustan.
          \item Si están de acuerdo con la ruta de instalación seleccionen
            \textbf{Siguiente/Next}
        \end{itemize}
    \end{columns}
\end{frame}

\begin{frame}[c]{Seleccionar folder del menú de inicio}
    \begin{columns}
        \column{0.5\textwidth}
        \begin{center}
            \includegraphics[scale=0.6]{lfs/thonny-05.png}
        \end{center}
        \column{0.5\textwidth}
        \begin{itemize}
          \item Pueden cambiar el folder del menú si gustan.
          \item Selecciona \textbf{Siguiente/Next}
        \end{itemize}
    \end{columns}
\end{frame}

\begin{frame}[c]{Agregar icono al escritorio}
    \begin{columns}
        \column{0.5\textwidth}
        \begin{center}
            \includegraphics[scale=0.6]{lfs/thonny-06.png}
        \end{center}
        \column{0.5\textwidth}
        \begin{itemize}
          \item Marca la casilla de \emph{Create desktop icon}
          \item Selecciona \textbf{Siguiente/Next}
        \end{itemize}
    \end{columns}
\end{frame}

\begin{frame}[c]{Listos para instalar}
    \begin{columns}
        \column{0.5\textwidth}
        \begin{center}
            \includegraphics[scale=0.6]{lfs/thonny-07.png}
        \end{center}
        \column{0.5\textwidth}
        \begin{itemize}
          \item Verifica que los datos sean correctos
          \item Selecciona \textbf{Siguiente/Next}
        \end{itemize}
    \end{columns}
\end{frame}

\begin{frame}[c]{Instalando}
    \begin{columns}
        \column{0.5\textwidth}
        \begin{center}
            \includegraphics[scale=0.6]{lfs/thonny-08.png}
        \end{center}
        \column{0.5\textwidth}
        \begin{itemize}
          \item Esperamos a que termine.
        \end{itemize}
    \end{columns}
\end{frame}

\begin{frame}[c]{Instalación exitosa}
    \begin{columns}
        \column{0.5\textwidth}
        \begin{center}
            \includegraphics[scale=0.6]{lfs/thonny-09.png}
        \end{center}
        \column{0.5\textwidth}
        \begin{itemize}
          \item Thonny ya esta instalado.
          \item Ejecútalo con el acceso directo o dando "click derecho"
            sobre un archivo con extinción \textbf{py} y seleccionando
            "\emph{Edit with Thonny}"
          \item Selecciona \textbf{Finalizar/Finish}
        \end{itemize}
    \end{columns}
\end{frame}

\begin{frame}[c]{Primera ejecución}
    \begin{columns}
        \column{0.6\textwidth}
        \begin{center}
            \includegraphics[scale=0.6]{lfs/thonny-10.png}
        \end{center}
        \column{0.4\textwidth}
        \begin{itemize}
          \item En la primera ejecución se presenta esta ventana que nos
            da a escoger el lenguaje y configuración inicial.
        \end{itemize}
    \end{columns}
\end{frame}

\begin{frame}[c]{Primera ejecución}
    \begin{columns}
        \column{0.6\textwidth}
        \begin{center}
            \includegraphics[scale=0.6]{lfs/thonny-11.png}
        \end{center}
        \column{0.4\textwidth}
        \begin{itemize}
          \item Aquí pueden seleccionar el lenguaje, \textbf{Español}
          \item Selecciona el boton de \textbf{Let's go!}
        \end{itemize}
    \end{columns}
\end{frame}

\begin{frame}[c]{Ventana de inicio}
 \begin{center}
   \includegraphics[scale=0.6]{lfs/thonny-12.png}
 \end{center}
\end{frame}

\begin{frame}[c]{Código de ejemplo}
  \begin{center}
      \includegraphics[scale=0.6]{lfs/thonny-13.png}
  \end{center}
\end{frame}

