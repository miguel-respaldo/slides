% ex: ts=2 sw=2 sts=2 et filetype=tex
% SPDX-License-Identifier: CC-BY-SA-4.0

\section{Manejo de excepciones}

\begin{frame}[c]{En resumen}
  Veremos los siguientes bloques de manejo de excepciones:

  \vspace{\baselineskip}
  \begin{itemize}
    \item El bloque \textcolor{codeKeyword2}{try} le permite probar un
      bloque de código en busca de errores.
    \pausa
    \item El bloque \textcolor{codeKeyword2}{except} le permite manejar el
      error.
    \pausa
    \item El bloque \textcolor{codeKeyword2}{else} te permite ejecutar
      código cuando no hay ningún error.
    \pausa
    \item El bloque \textcolor{codeKeyword2}{finally} le permite ejecutar
      código, independientemente del resultado de los bloques de prueba y
      excepción.
  \end{itemize}
\end{frame}

\begin{frame}[fragile]
  \frametitle{Manejo de excepciones}
  Cuando ocurre un error, o una excepción como lo llamamos, Python
  normalmente se detendrá y generará un mensaje de error.

  \vspace{\baselineskip}
  Estas excepciones se pueden manejar usando la declaración
  \textcolor{codeKeyword2}{try}:

  \vspace{\baselineskip}
  Ejemplo: El bloque \textcolor{codeKeyword2}{try} generará una excepción,
  porque \textbf{x} no está definida:
  \begin{lstlisting}[language=Python]
  try:
      print(x)
  except:
      print("Ocurrió una excepción ")
  \end{lstlisting}
\end{frame}

\begin{frame}[fragile]
  \frametitle{Manejo de excepciones}
  Dado que el bloque de prueba genera un error, se ejecutará el
  bloque de excepción.

  \vspace{\baselineskip}
  Sin que el bloque de prueba del programa fallará y generará un error:

  \vspace{\baselineskip}
  Ejemplo: Esta declaración generará un error,
  porque \textbf{x} no está definida:
  \begin{lstlisting}[language=Python]
  print(x)
  \end{lstlisting}
\end{frame}

\begin{frame}[fragile]
  \frametitle{Muchas excepciones}

  Puede definir tantos bloques de excepción como desee, por ejemplo,
  si desea ejecutar un bloque de código especial para un tipo especial
  de error:

  \vspace{\baselineskip}
  Ejemplo: Imprimir un mensaje si el bloque try genera un
  \textbf{NameError} y otro para otros errores:
  \begin{lstlisting}[language=Python]
  try:
      print(x)
  except NameError:
      print("La variable x no esta definida")
  except:
      print("Algo más salio mal") 
  \end{lstlisting}
\end{frame}

\begin{frame}[fragile]
  \frametitle{Más/Si no}

  Puede usar la palabra clave \textcolor{codeKeyword2}{else}
  para definir un bloque de código que se ejecutará si no se generan errores:

  \vspace{\baselineskip}
  Ejemplo: En este ejemplo, el bloque \textcolor{codeKeyword2}{try}
  no genera ningún error:
  \begin{lstlisting}[language=Python]
  try:
      print("Hola")
  except:
      print("Algo salio mal") 
  else:
      print("Nada salio mal") 
  \end{lstlisting}
\end{frame}

\begin{frame}[fragile]
  \frametitle{Finalmente}

  Si se especifica el bloque \textcolor{codeKeyword2}{finally}, se ejecutará
  independientemente de si el bloque try genera un error o no.

  \vspace{\baselineskip}
  \begin{lstlisting}[language=Python]
  try:
      print(x)
  except:
      print("Algo salio mal") 
  finally:
      print("El bloque de 'try except' ha terminado") 
  \end{lstlisting}
\end{frame}

\begin{frame}[fragile]
  \frametitle{Finalmente}

  Esto puede ser útil para cerrar objetos y limpiar recursos:

  \vspace{\baselineskip}
  Ejemplo: Intente abrir y escribir en un archivo que no se puede escribir:
  \begin{lstlisting}[language=Python]
  try:
      f = open("demo.txt")
      try:
          f.write("Algún texto ...")
      except:
          print("Algo salio mal al escribir en el archivo")
      finally:
          f.close()
  except:
      print("Algo salio mal al abrir el archivo")
  \end{lstlisting}

  \vspace{\baselineskip}
  El programa puede continuar, sin dejar abierto el objeto de archivo.
\end{frame}

\begin{frame}[fragile]
  \frametitle{Levantar una excepción}

  Como desarrollador de Python, puede optar por generar una excepción si
  se produce una condición.

  \vspace{\baselineskip}
  Para lanzar (o generar) una excepción, use la palabra clave
  \textcolor{codeKeyword2}{raise}.

  \vspace{\baselineskip}
  Ejemplo: Genera un error y detiene el programa si x es menor que 0:
  \begin{lstlisting}[language=Python]
  x = -1

  if x < 0:
      raise Exception("Lo siento, no se permiten números menores a cero")
  \end{lstlisting}
\end{frame}

\begin{frame}[fragile]
  \frametitle{Levantar una excepción}

  La palabra clave \textcolor{codeKeyword2}{raise} se utiliza para
  generar una excepción.

  \vspace{\baselineskip}
  Puede definir qué tipo de error generar y el texto para imprimir al usuario.

  \vspace{\baselineskip}
  Ejemplo: Genera un TypeError si x no es un número entero:
  \begin{lstlisting}[language=Python]
  x = "hello"

  if not type(x) is int:
      raise TypeError("Solo enteros son permitidos")
  \end{lstlisting}
\end{frame}

