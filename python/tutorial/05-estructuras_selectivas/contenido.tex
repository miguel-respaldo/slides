% ex: ts=2 sw=2 sts=2 et filetype=tex
% SPDX-License-Identifier: CC-BY-SA-4.0

\section{Condiciones lógicas y sentencias "si"}

\begin{frame}[c]{Condiciones lógicas}

  Python admite las condiciones lógicas habituales de las matemáticas:

  \vspace{\baselineskip}
  \begin{itemize}
    \item Es igual a \textbf{a == b}
    \pausa
    \item No es igual a \textbf{a != b}
    \pausa
    \item Menor que \textbf{a < b}
    \pausa
    \item Menor o igual a \textbf{a <= b}
    \pausa
    \item Mayor que \textbf{a > b}
    \pausa
    \item Mayor o igual a \textbf{a >= b}
  \end{itemize}

  \vspace{\baselineskip}
  Estas condiciones se pueden utilizar de varias formas, más
  comúnmente en "sentencias \textcolor{codeKeyword}{if}" y bucles.
\end{frame}

\begin{frame}[fragile]
  \frametitle{Sentencias if}

  Una "instrucción if" se escribe utilizando la palabra clave
  \textcolor{codeKeyword}{if}.

  \vspace{\baselineskip}
  \begin{lstlisting}[language=Python]
  a = 33
  b = 200
  if b > a:
    print("b es mas grande que a")
  \end{lstlisting}

  En este ejemplo usamos dos variables, a y b, que se usan como parte de
  la instrucción \textcolor{codeKeyword}{if} para probar si b es mayor
  que a. Como a es 33 y b es 200, sabemos que 200 es mayor que 33,
  por lo que imprimimos en la pantalla que "b es mayor que a".
\end{frame}

\begin{frame}[fragile]
  \frametitle{Sangría}

  Python se basa en la sangría (espacio en blanco al comienzo de una línea)
  para definir el alcance en el código. Otros lenguajes de programación a
  menudo usan corchetes para este propósito.

  \pausa
  \vspace{\baselineskip}
  \begin{alertblock}{}
    Una sentencia if sin sangría provocará un error.
  \end{alertblock}

  \vspace{\baselineskip}
  \begin{lstlisting}[language=Python]
  a = 33
  b = 200
  if b > a:
  print("b es mas grande que a")
  \end{lstlisting}
\end{frame}

\begin{frame}[fragile]
  \frametitle{Elif}

  La palabra clave \textcolor{codeKeyword}{elif} es la forma que usa Python
  para decir "si las condiciones anteriores no eran verdaderas, entonces
  pruebe esta condición".

  \vspace{\baselineskip}
  \begin{lstlisting}[language=Python]
  a = 33
  b = 33
  if b > a:
    print("b es mas grande que a")
  elif a == b:
    print("a y b son iguales")
  \end{lstlisting}

  En este ejemplo, a es igual a b, por lo que la primera condición no es
  verdadera, pero la condición elif es verdadera, por lo que imprimimos en
  la pantalla que "a y b son iguales".
\end{frame}

\begin{frame}[fragile]
  \frametitle{Else}

  La palabra clave \textcolor{codeKeyword}{else} captura cualquier cosa que
  no sea detectada por las condiciones anteriores.

  \vspace{\baselineskip}
  \begin{lstlisting}[language=Python]
  a = 200
  b = 33
  if b > a:
    print("b es mas grande que a")
  elif a == b:
    print("a y b son iguales")
  else:
    print("a es mayor que b")
  \end{lstlisting}

  En este ejemplo, a es mayor que b, por lo que la primera condición no es
  verdadera, también la condición elif no es verdadera, así que vamos a la
  condición else e imprimimos en la pantalla que "a es mayor que b". 
\end{frame}

\begin{frame}[fragile]
  \frametitle{Else}

  Se puede tener \textcolor{codeKeyword}{else} sin el
  \textcolor{codeKeyword}{elif}

  \vspace{\baselineskip}
  \begin{lstlisting}[language=Python]
  a = 200
  b = 33
  if b > a:
    print("b es mas grande que a")
  else:
    print("b no es mayor que a")
  \end{lstlisting}
\end{frame}

\begin{frame}[fragile]
  \frametitle{Manera corta del if}

  Si solo tiene una instrucción para ejecutar, puede ponerla en la misma
  línea que la instrucción if.

  \vspace{\baselineskip}
  \begin{lstlisting}[language=Python]
  if a > b: print("a es mayor que b")
  \end{lstlisting}
\end{frame}

\begin{frame}[fragile]
  \frametitle{Manera corta del if ... else}

  Si solo tiene una declaración para ejecutar, una para if y otra para
  else, puede ponerlas todas en la misma línea:

  \vspace{\baselineskip}
  \begin{lstlisting}[language=Python]
  a = 2
  b = 330
  print("A") if a > b else print("B")
  \end{lstlisting}

  \vspace{\baselineskip}
  \begin{exampleblock}{}
    Esta técnica se conoce como \textbf{operadores ternarios} o
    \textbf{expresiones condicionales}.
  \end{exampleblock}
\end{frame}

\begin{frame}[fragile]
  \frametitle{Manera corta del if ... else}

  También puede tener varias declaraciones else en la misma línea:

  \vspace{\baselineskip}
  \begin{lstlisting}[language=Python]
  a = 2
  b = 330
  print("A") if a > b else print("=") if a == b else print("B")
  \end{lstlisting}
\end{frame}

\begin{frame}[fragile]
  \frametitle{And}

  La palabra clave \textcolor{codeKeyword}{and} es un operador lógico y
  se utiliza para combinar declaraciones condicionales: 

  \vspace{\baselineskip}
  \begin{lstlisting}[language=Python]
  a = 200
  b = 33
  c = 500
  if a > b and c > a:
    print("Ambas condiciones son verdaderas")
  \end{lstlisting}
\end{frame}

\begin{frame}[fragile]
  \frametitle{Or}

  La palabra clave \textcolor{codeKeyword}{or} es un operador lógico y
  se utiliza para combinar declaraciones condicionales: 

  \vspace{\baselineskip}
  \begin{lstlisting}[language=Python]
  a = 200
  b = 33
  c = 500
  if a > b or a > a:
    print("Al menos una condicion es verdaderas")
  \end{lstlisting}
\end{frame}

\begin{frame}[fragile]
  \frametitle{If anidados}

  Puede tener sentencias if dentro de sentencias if, esto se llama
  \textbf{sentencias if anidadas}.

  \vspace{\baselineskip}
  \begin{lstlisting}[language=Python]
  x = 41

  if x > 10:
    print("mayor que diez,")
    if x > 20:
      print("y tambien mayor que 20!")
    else:
      print("pero menor que 20.")
  \end{lstlisting}
\end{frame}

\begin{frame}[fragile]
  \frametitle{La sentencia \textbf{pass}}

  Las declaraciones if no pueden estar vacías, pero si por alguna razón
  tiene una declaración if sin contenido, coloque la declaración pass
  para evitar errores. 

  \vspace{\baselineskip}
  \begin{lstlisting}[language=Python]
  a = 33
  b = 200

  if b > a:
    pass
  \end{lstlisting}
\end{frame}

