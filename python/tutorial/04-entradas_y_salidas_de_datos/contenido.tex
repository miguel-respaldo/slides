% ex: ts=2 sw=2 sts=2 et filetype=tex
% SPDX-License-Identifier: CC-BY-SA-4.0

\section{Entrada de datos}

\begin{frame}[c]{Entrada y salida de datos}

  \vspace{\baselineskip}
  Existen operaciones de entrada y salida, también conocidas como
  operaciones de lectura y escritura, respectivamente.

  \vspace{\baselineskip}
  Una \textbf{entrada} de datos es una operación que permite leer
  un valor y asignarlo a una determinada variable.

  \vspace{\baselineskip}
  Una \textbf{salida} de datos es una operación derivada de un
  proceso que permite mostrar información contenida en variables.

  \pausa
  \vspace{\baselineskip}
  \begin{columns}
    \column{0.1\textwidth}
    \column{0.4\textwidth}
      \textbf{Entradas}
      \begin{itemize}
        \item Teclado
        \item Puerto / Dispositivo
        \item Archivo / Base de Datos
        \item Otro programa
      \end{itemize}
    \column{0.4\textwidth}
      \textbf{Salidas}
      \begin{itemize}
        \item Pantalla
        \item Puerto / Dispositivo
        \item Archivo / Base de Datos
        \item Otro programa
      \end{itemize}
    \column{0.1\textwidth}
  \end{columns}
\end{frame}

\begin{frame}[fragile]
  \frametitle{Entrada de usuario}

  Python permite la entrada del usuario, eso significa que podemos pedirle
  información al usuario, y para ello usamos el método
  \textcolor{codeKeyword2}{input}().

  \vspace{\baselineskip}
  El siguiente ejemplo solicita el nombre de usuario, y cuando ingresó el
  nombre de usuario, se imprime en la pantalla:

  \vspace{\baselineskip}
  \begin{lstlisting}[language=Python]
 usuario = input("Escribe tu usuario: ")
 print("El usuario es: " + usuario)
  \end{lstlisting}

  \pausa
  \begin{exampleblock}{}
    Python detiene la ejecución cuando se trata de la función
    \textcolor{codeKeyword2}{input}().
    y continúa cuando el usuario ha dado alguna entrada.
  \end{exampleblock}
\end{frame}

\begin{frame}[fragile]
  \frametitle{Función input()}

  La función \textcolor{codeKeyword2}{input}() permite entrada de datos del
  usuario y estos los regresa como una \textbf{cadena de texto}.

  \vspace{\baselineskip}
  \textbf{Sintaxis}:

  \vspace{\baselineskip}
  \textcolor{codeKeyword2}{input}(mensaje)

  \vspace{\baselineskip}
  \textbf{Parámetros}:
  \begin{description}
    \item[mensaje] Una cadena que representa un mensaje antes de la entrada
      de datos.
  \end{description}

  \vspace{\baselineskip}
  \begin{lstlisting}[language=Python]
  nombre = input("Escribe tu nombre: ")
  print("Hola, " + nombre)
  \end{lstlisting}
\end{frame}

\begin{frame}[c]{Función eval()}

  La función \textcolor{codeKeyword2}{eval}() evalúa la expresión
  especificada, si la expresión es una declaración de Python legal,
  se ejecutará; si la expresión en un número lo convierte al tipo de
  dato que corresponde.

  \vspace{\baselineskip}
  \textbf{Sintaxis}:

  \vspace{\baselineskip}
  \textcolor{codeKeyword2}{eval}(expresión, globales, locales)

  \vspace{\baselineskip}
  \textbf{Parámetros}:
  \begin{description}
    \item[expresión] Una cadena de texto, que se evaluará como código
      Python.
    \item[globales] Opcional: Un diccionario que contenga los parámetros
      globales.
    \item[locales] Opcional: Un diccionario que contenga los parámetros
      locales.
  \end{description}
\end{frame}

\begin{frame}[fragile]
  \frametitle{Función eval()}

  \begin{lstlisting}[language=Python]
  x = "print(55)"
  eval(x) # aquí me ejecuta print(55), en pantalla solo sale 55

  numero = eval(input("Escribe un número: "))
  print( type(numero) )
  \end{lstlisting}
\end{frame}


\begin{frame}[fragile]
  \frametitle{Leer texto y números}

  Para leer \textbf{números} usamos:
  \begin{lstlisting}[language=Python]
  numero = eval(input("Escribe un número: "))
  print("Escribiste el número:", numero)
  \end{lstlisting}

  \vspace{\baselineskip}
  Para leer \textbf{cadenas de texto} usamos:
  \begin{lstlisting}[language=Python]
  texto = input("Escribe un texto: ")
  print("Escribiste el texto: " + texto)
  \end{lstlisting}

\end{frame}

\section{Salida de datos}

\begin{frame}[c]{Formateando cadenas de texto}

  Para asegurarnos de que una cadena se muestre como se esperaba,
  podemos formatear el resultado con el método
  \textcolor{codeKeyword2}{format}().

  \vspace{\baselineskip}
  El método \textcolor{codeKeyword2}{format}() permite formatear
  partes seleccionadas de una cadena.

  \vspace{\baselineskip}
  A veces, hay partes de un texto que no controlas, porque tal vez provienen
  de una base de datos o de una entrada de usuario.

\end{frame}

\begin{frame}[fragile]
  \frametitle{Formateando cadenas de texto}

  Para controlar dichos valores, se agrega marcadores de posición
  (unas llaves \{\}) en el texto y se ejecutan los valores a través del
  método \textcolor{codeKeyword2}{format}().

  \vspace{\baselineskip}
  \begin{lstlisting}[language=Python]
precio = 49
txt = "El precio es {} pesos"
print(txt.format(precio))
  \end{lstlisting}

  \pausa
  Se puede agregar parámetros dentro de las llaves para especificar cómo convertir el valor:

  \vspace{\baselineskip}
  \begin{lstlisting}[language=Python]
txt = "El precio es {:.2f} pesos"
  \end{lstlisting}
\end{frame}

\begin{frame}[fragile]
  \frametitle{Múltiples Valores}

  Si desea usar más valores, simplemente agregue más valores
  método \textcolor{codeKeyword2}{format}():

  \vspace{\baselineskip}
  \begin{lstlisting}[language=Python]
print(txt.format(precio, articulo, cantidad))
  \end{lstlisting}

  \pausa
  Y agregue más marcadores de posición: 

  \vspace{\baselineskip}
  \begin{lstlisting}[language=Python]
cantidad = 3
articulo = 567
precio = 49
txt = "El costo de {} piezas del artículo con número {} es de {:.2f} pesos"
print(txt.format(cantidad, articulo, precio))
  \end{lstlisting}
\end{frame}

\begin{frame}[fragile]
  \frametitle{Números de indice}

  Se puede utilizar números de índice (un número dentro de las llaves
  \textbf{\{0\}}) para asegurarse de que los valores se coloquen en
  los marcadores de posición correctos: 

  \vspace{\baselineskip}
  \begin{lstlisting}[language=Python]
cantidad = 3
articulo = 567
precio = 49
txt = "El costo de {0} piezas del artículo con número {1} es de {2:.2f} pesos"
print(txt.format(cantidad, articulo, precio))
  \end{lstlisting}
\end{frame}

\begin{frame}[fragile]
  \frametitle{Números de indice}

  Además, si se desea hacer referencia al mismo valor más de una vez,
  use el número de índice:

  \vspace{\baselineskip}
  \begin{lstlisting}[language=Python]
hermanos = 2
nombre = "Juan"
txt = "Su nombre es {1}. {1} tiene {0} hermanos."
print(txt.format(hermanos, nombre))
  \end{lstlisting}
\end{frame}

\begin{frame}[fragile]
  \frametitle{Indices con nombre}

  También puede usar índices con nombre ingresando un nombre entre
  llaves \textbf{\{marca\}}, pero luego se debe usar los nombres cuando
  se pase los valores en los parámetro \\
  \textbf{txt.format (marca = "Ford")}:

  \vspace{\baselineskip}
  \begin{lstlisting}[language=Python]
txt = "Yo tengo un {marca}, es un {modelo}."
print(txt.format(marca = "Ford", modelo = "Mustang"))
  \end{lstlisting}
\end{frame}
