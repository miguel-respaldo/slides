% ex: ts=2 sw=2 sts=2 et filetype=tex
% SPDX-License-Identifier: CC-BY-SA-4.0

\section{Abriendo archivos}

\begin{frame}[c]{Manejo de archivos}

  El manejo de archivos es una parte importante de cualquier
  aplicación web.

  \vspace{\baselineskip}
  Python tiene varias funciones para crear, leer, actualizar y
  eliminar archivos.

  \vspace{\baselineskip}
  La función clave para trabajar con archivos en Python es la
  función \textcolor{codeKeyword2}{open}().
\end{frame}

\begin{frame}[c]{La función \textbf{open}()}

  La función \textcolor{codeKeyword2}{open}() toma dos parámetros;
  \textbf{nombre de archivo} y \textbf{modo}.

  \vspace{\baselineskip}
  Hay cuatro métodos (modos) diferentes para abrir un archivo: 

  \vspace{\baselineskip}
  \begin{description}
    \item[r] Lectura - Valor predeterminado. Abre un archivo para
      lectura, error si el archivo no existe
    \pausa
    \item[a] Adjuntar - Abre un archivo para agregar, crea el
      archivo si no existe
    \pausa
    \item[w] Escritura - Abre un archivo para escritura, crea el
      archivo si no existe
    \pausa
    \item[x] Crear - Crea el archivo especificado, devuelve un error
      si el archivo existe
  \end{description}
\end{frame}

\begin{frame}[c]{Manejo de archivos}

  Además, puede especificar si el archivo debe manejarse en
  modo binario o de texto.

  \vspace{\baselineskip}
  \begin{description}
    \item[t] Texto - Valor predeterminado. Modo texto
    \pausa
    \item[b] Binario - Modo binario (p. ej., imágenes)
  \end{description}
\end{frame}


\begin{frame}[fragile]
  \frametitle{Sintaxis}

  Para abrir un archivo para leerlo basta con especificar
  el nombre del archivo: 

  \vspace{\baselineskip}
  \begin{lstlisting}[language=Python]
  f = open("demo.txt")
  \end{lstlisting}
\end{frame}

\begin{frame}[fragile]
  \frametitle{Sintaxis}

  El código anterior es el mismo que: 

  \vspace{\baselineskip}
  \begin{lstlisting}[language=Python]
  f = open("demo.txt","rt")
  \end{lstlisting}

  \vspace{\baselineskip}
  Dado que "\textbf{r}" para lectura y "\textbf{t}" para texto
  son los valores predeterminados, no es necesario especificarlos.

  \pausa
  \begin{alertblock}{Nota}
  Asegúrese de que el archivo exista o, de lo contrario, obtendrá un error.
  \end{alertblock}
\end{frame}

\section{Lectura archivos}

\begin{frame}[fragile]
  \frametitle{Leyendo archivos}

  Supongamos que tenemos el siguiente archivo, ubicado en la
  misma carpeta donde ejecutamos un código de Python:

  \vspace{\baselineskip}
  \begin{lstlisting}[language=Python]
  Hola! Bienvendios al demo.txt.
  Este es un archivo para pruebas.
  Buena suerte!.
  \end{lstlisting}
\end{frame}

\begin{frame}[fragile]
  \frametitle{Leyendo archivos}

  Para abrir el archivo, use la función open () incorporada.

  \vspace{\baselineskip}
  La función open () devuelve un objeto de archivo, que tiene un
  método \textbf{read}() para leer el contenido del archivo: 

  \vspace{\baselineskip}
  \begin{lstlisting}[language=Python]
  f = open("demo.txt", "r")
  print(f.read())
  \end{lstlisting}
\end{frame}

\begin{frame}[fragile]
  \frametitle{Leyendo archivos}

  Si el archivo se encuentra en una ubicación diferente,
  deberá especificar la ruta del archivo, así: 

  \vspace{\baselineskip}
  \begin{lstlisting}[language=Python]
  f = open("C:\\en\\windows\\es\\demo.txt", "r")
  print(f.read())
  \end{lstlisting}
\end{frame}

\begin{frame}[fragile]
  \frametitle{Leyendo solo unas partes del archivo}

  De forma predeterminada, el método \textbf{read}() devuelve el texto
  completo, pero también puede especificar cuántos caracteres
  desea devolver:

  \vspace{\baselineskip}
  \begin{lstlisting}[language=Python]
  f = open("demo.txt", "r")
  print(f.read(5))
  \end{lstlisting}
\end{frame}

\begin{frame}[fragile]
  \frametitle{Leyendo lineas}

  Puede devolver una línea utilizando el método
  \textbf{readline}():

  \vspace{\baselineskip}
  \begin{lstlisting}[language=Python]
  f = open("demo.txt", "r")
  print(f.readline())
  \end{lstlisting}
\end{frame}

\begin{frame}[fragile]
  \frametitle{Leyendo lineas}

  Al llamar a \textbf{readline}() dos veces, puede leer
  las dos primeras líneas:

  \vspace{\baselineskip}
  \begin{lstlisting}[language=Python]
  f = open("demo.txt", "r")
  print(f.readline())
  print(f.readline())
  \end{lstlisting}
\end{frame}

\begin{frame}[fragile]
  \frametitle{Leyendo lineas}

  Al recorrer las líneas del archivo, puede leer todo el
  archivo, línea por línea:

  \vspace{\baselineskip}
  \begin{lstlisting}[language=Python]
  f = open("demofile.txt", "r")
  for x in f:
      print(x)
  \end{lstlisting}
\end{frame}

\begin{frame}[fragile]
  \frametitle{Cerrando el archivo}

  Es una buena práctica cerrar siempre el archivo
  cuando haya terminado.

  \vspace{\baselineskip}
  \begin{lstlisting}[language=Python]
  f = open("demofile.txt", "r")
  print(f.readline())
  f.close()
  \end{lstlisting}

  \pausa
  \begin{exampleblock}{Nota}
    Siempre debe cerrar sus archivos; en algunos casos,
    debido al almacenamiento en búfer, es posible que los
    cambios realizados en un archivo no se muestren hasta
    que lo cierre. 
  \end{exampleblock}
\end{frame}

\section{Escribiendo en los archivos}

\begin{frame}[c]{Escribir en un archivo existente}

  Para escribir en un archivo existente, debe agregar un
  parámetro a la función \textcolor{codeKeyword2}{open}():

  \vspace{\baselineskip}
  \begin{description}
    \item[a] Adjuntar - se agregará al final del archivo
    \item[w] Escribir - sobrescribirá cualquier contenido existente 
  \end{description}
\end{frame}

\begin{frame}[fragile]
  \frametitle{Escribir en un archivo existente}

  Ejemplo: abrir el archivo "demo2.txt" y agregue un
  contenido al archivo: 

  \vspace{\baselineskip}
  \begin{lstlisting}[language=Python]
  f = open("demo2.txt", "a")
  f.write("Ahora el archivo tiene mas contendio!")
  f.close()

  #abrir y leer el archivo despues de agregar contenido
  f = open("demo2.txt", "r")
  print(f.read()) 
  \end{lstlisting}
\end{frame}

\begin{frame}[fragile]
  \frametitle{Escribir en un archivo existente}

  Ejemplo: abrir el archivo "demo3.txt" y sobrescriba el
  contenido: 

  \vspace{\baselineskip}
  \begin{lstlisting}[language=Python]
  f = open("demo3.txt", "w")
  f.write("Woops! Acabo de borrar el contendio!")
  f.close()

  #abrir y leer el archivo despues de agregar contenido
  f = open("demofile3.txt", "r")
  print(f.read())
  \end{lstlisting}
\end{frame}
