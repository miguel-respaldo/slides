% ex: ts=2 sw=2 sts=2 et filetype=tex
% SPDX-License-Identifier: CC-BY-SA-4.0

\begin{frame}[c]{Colecciones en Python}

  Hay cuatro tipos de datos de recopilación en el lenguaje de programación Python:

  \pausa
  \begin{description}
    \item[List ] es una colección ordenada y modificable.
      Permite miembros duplicados.
    \pausa
    \item[Tuple ] es una colección ordenada e inmutable.
      Permite miembros duplicados.
    \pausa
    \item[Set ] es una colección desordenada y no indexada.
      Sin miembros duplicados.
    \pausa
    \item[Dictionary ] es una colección ordenada* y modificable.
      Sin miembros duplicados.
  \end{description}

  \begin{exampleblock}{}
    * A partir de la versión 3.7 de Python, los diccionarios están ordenados.
    En Python 3.6 y versiones anteriores, los diccionarios están desordenados.
  \end{exampleblock}

  Al elegir un tipo de colección, es útil comprender las propiedades de ese
  tipo. Elegir el tipo correcto para un conjunto de datos en particular
  podría significar la utilización del significado y podría significar un
  aumento en la eficiencia o la seguridad. 
\end{frame}

\section{Listas}

\begin{frame}[fragile]
  \frametitle{Listas}

  \vspace{\baselineskip}
  Las listas se utilizan para almacenar varios elementos en una sola variable.

  \vspace{\baselineskip}
  Las listas son uno de los 4 tipos de datos integrados en Python que se
  utilizan para almacenar colecciones de datos, los otros 3 son Tuple,
  Set y Dictionary, todos con diferentes calidades y usos.

  \vspace{\baselineskip}
  Las listas se crean utilizando corchetes:

  \vspace{\baselineskip}
  \begin{lstlisting}[language=Python]
  una_lista = ["manzana", "platano", "naranja"]
  print(una_lista) 
  \end{lstlisting}
\end{frame}

\begin{frame}[c]{Propiedades de las listas}

  \begin{block}{Elementos de la lista}
    Los elementos de la lista están ordenados, se pueden cambiar
    y permiten valores duplicados.

    Los elementos de la lista están indexados, el primer elemento tiene
    índice \textbf{[0]}, el segundo elemento tiene índice \textbf{[1]}, etc.
  \end{block}

  \pausa
  \begin{block}{Ordenado}
    Cuando decimos que las listas están ordenadas, significa que los
    artículos tienen un orden definido y ese orden no cambiará.

    Si agrega nuevos elementos a una lista, los nuevos elementos se
    colocarán al final de la lista.
  \end{block}

  \begin{alertblock}{Nota}
    Hay algunos métodos de lista que cambiarán el orden,
    pero en general: el orden de los elementos no cambiará. 
  \end{alertblock}
\end{frame}

\begin{frame}[fragile]
  \frametitle{Propiedades de las listas}

  \begin{block}{Modificable}
    La lista se puede cambiar, lo que significa que podemos cambiar,
    agregar y eliminar elementos en una lista después de que se haya creado.
  \end{block}

  \pausa
  \begin{block}{Permite duplicados}
    Dado que las listas están indexadas, las listas pueden tener
    elementos con el mismo valor: 
  \end{block}

  \vspace{\baselineskip}
  \begin{lstlisting}[language=Python]
  una_lista = ["manzana", "platano", "naranja", "manzana", "naranja"]
  print(una_lista) 
  \end{lstlisting}
\end{frame}

\begin{frame}[fragile]
  \frametitle{Longitud de una lista}

  Para determinar cuántos elementos tiene una lista, use la función
  \textcolor{codeKeyword2}{len}():

  \vspace{\baselineskip}
  \begin{lstlisting}[language=Python]
  una_lista = ["manzana", "platano", "naranja"]
  print(len(una_lista)) 
  \end{lstlisting}
\end{frame}

\begin{frame}[fragile]
  \frametitle{Elementos de lista: tipos de datos}

  Los elementos de la lista pueden ser de cualquier tipo de datos:

  \vspace{\baselineskip}
  \begin{lstlisting}[language=Python]
  lista1 = ["manzana", "platano", "naranja"]
  lista2 = [1, 5, 7, 9, 3]
  lista3 = [True, False, False]
  \end{lstlisting}

  \pausa
  Una lista puede contener diferentes tipos de datos:
  \vspace{\baselineskip}
  \begin{lstlisting}[language=Python]
  lista1 = ["abc", 34, True, 40.5, "texto"]
  \end{lstlisting}
\end{frame}

\begin{frame}[fragile]
  \frametitle{Tipo de dato}

  Desde la perspectiva de Python, las listas se definen
  como objetos con el tipo de datos 'lista':

  \vspace{\baselineskip}
  \textbf{<class 'list'>}

  \vspace{\baselineskip}
  \begin{lstlisting}[language=Python]
  una_lista = ["manzana", "platano", "naranja"]
  print(type(una_lista)) 
  \end{lstlisting}
\end{frame}

\begin{frame}[fragile]
  \frametitle{El constructor list()}

  También es posible utilizar el constructor
  \textcolor{codeKeyword2}{list}() al crear una nueva lista. 

  \vspace{\baselineskip}
  \begin{lstlisting}[language=Python]
  una_lista = list(("manzana", "platano", "naranja")) # con doble parentesis
  print(type(una_lista)) 
  \end{lstlisting}
\end{frame}

\section{Accediendo a los elementos de una lista}

\begin{frame}[fragile]
  \frametitle{Elementos de acceso}

  Los elementos de la lista están indexados y puede acceder
  a ellos consultando el número de índice:

  \vspace{\baselineskip}
  \begin{lstlisting}[language=Python]
  lista = ["manzana", "platano", "naranja"]
  print(lista[1]) 
  \end{lstlisting}

  \begin{exampleblock}{Nota}
    El primer elemento tiene indice 0.
  \end{exampleblock}
\end{frame}

\begin{frame}[fragile]
  \frametitle{Indexación negativa}

  La indexación negativa significa comenzar desde el final

  \vspace{\baselineskip}
  -1 se refiere al último elemento, -2 se refiere al penúltimo elemento, etc.

  \vspace{\baselineskip}
  \begin{lstlisting}[language=Python]
  lista = ["manzana", "platano", "naranja"]
  print(lista[-1]) 
  \end{lstlisting}
\end{frame}

\begin{frame}[fragile]
  \frametitle{Rango de índices}

  Puede especificar un rango de índices especificando
  dónde comenzar y dónde terminar el rango.

  \vspace{\baselineskip}
  Al especificar un rango, el valor de retorno será una nueva
  lista con los elementos especificados.

  \vspace{\baselineskip}
  \begin{lstlisting}[language=Python]
lista = ["manzana", "platano", "limon", "cereza", "kiwi", "mango", "melon"]
print(lista[2:5]) 
  \end{lstlisting}

  \pausa
  \begin{alertblock}{Nota}
    La búsqueda comenzará en el índice 2 (incluido)
    y terminará en el índice 5 (no incluido). 

    Recuerde que el primer elemento tiene índice 0.
  \end{alertblock}
\end{frame}

\begin{frame}[fragile]
  \frametitle{Rango de índices}

  Al omitir el valor inicial, el rango comenzará en el primer elemento: 
  \vspace{\baselineskip}
  \begin{lstlisting}[language=Python]
lista = ["manzana", "platano", "limon", "cereza", "kiwi", "mango", "melon"]
print(lista[:4]) 
  \end{lstlisting}
\end{frame}

\begin{frame}[fragile]
  \frametitle{Rango de índices}

  Al omitir el valor final, el rango continuará hasta el final de la lista: 
  \vspace{\baselineskip}
  \begin{lstlisting}[language=Python]
lista = ["manzana", "platano", "limon", "cereza", "kiwi", "mango", "melon"]
print(lista[2:]) 
  \end{lstlisting}
\end{frame}

\begin{frame}[fragile]
  \frametitle{Rango de índices negativos}

  Especifique índices negativos si desea iniciar la búsqueda
  desde el final de la lista:

  \vspace{\baselineskip}
  \begin{lstlisting}[language=Python]
lista = ["manzana", "platano", "limon", "cereza", "kiwi", "mango", "melon"]
print(lista[-4:-1]) 
  \end{lstlisting}
\end{frame}

\begin{frame}[fragile]
  \frametitle{Verificar si un elemento existe}

  Para determinar si un elemento específico está presente en una lista,
  use la palabra clave
  \textcolor{codeKeyword}{in}:

  \vspace{\baselineskip}
  \begin{lstlisting}[language=Python]
  lista = ["manzana", "platano", "naranja"]
  if "manzana" in lista:
    print("Si, la manzana si esta en la lista")
  \end{lstlisting}
\end{frame}

\section{Agregar elementos a la lista}

\begin{frame}[fragile]
  \frametitle{Agregar elementos}

  Para agregar un elemento al final de la lista, use el método
  \textbf{append}():

  \vspace{\baselineskip}
  \begin{lstlisting}[language=Python]
  lista = ["manzana", "platano", "naranja"]
  lista.append("kiwi")
  print(lista)
  \end{lstlisting}
\end{frame}

\begin{frame}[fragile]
  \frametitle{Insertar elementos}

  Para insertar un elemento de la lista en un índice específico,
  use el método \textbf{insert}().

  El método \textbf{insert}() inserta un elemento en el
  índice especificado: 

  \vspace{\baselineskip}
  \begin{lstlisting}[language=Python]
  lista = ["manzana", "platano", "naranja"]
  lista.insert(1, "kiwi")
  print(lista)
  \end{lstlisting}

  \begin{exampleblock}{Nota}
    Como resultado de los ejemplos anteriores,
    la lista ahora contendrán 4 elementos.
  \end{exampleblock}{}
\end{frame}

\begin{frame}[fragile]
  \frametitle{Ampliar lista}

  Para agregar elementos de otra lista a la lista actual,
  use el método \textbf{extend}(). 

  \vspace{\baselineskip}
  \begin{lstlisting}[language=Python]
  lista = ["manzana", "platano", "naranja"]
  otra_lista = ["mango", "papaya", "limon"]
  lista.extend(otra_lista)
  print(lista)
  \end{lstlisting}

  \vspace{\baselineskip}
  Los elementos se agregarán al final de la lista.
\end{frame}

\begin{frame}[fragile]
  \frametitle{Agregar cualquier iterable}

  El método \textbf{extend}() no tiene que agregar listas,
  puede agregar cualquier objeto iterable
  (tuplas, conjuntos, diccionarios, etc.).

  \vspace{\baselineskip}
  \begin{lstlisting}[language=Python]
  lista = ["manzana", "platano", "naranja"]
  tupla = ("mango", "kiwi")
  lista.extend(tupla)
  print(lista)
  \end{lstlisting}
\end{frame}

\section{Comprensión de listas}

\begin{frame}[c]{Comprensión de listas}
  La comprensión de listas ofrece una sintaxis más corta cuando
  desea crear una nueva lista basada en los valores de una lista
  existente.
\end{frame}

\begin{frame}[fragile]
  \frametitle{Comprensión de listas - Ejemplo}

  Basado en una lista de frutas, desea una nueva lista que
  contenga solo las frutas con la letra "a" en el nombre.

  \vspace{\baselineskip}
  Sin la comprensión de la lista, tendrá que escribir una declaración
  \textcolor{codeKeyword}{for} con una prueba condicional dentro: 

  \vspace{\baselineskip}
  \begin{lstlisting}[language=Python]
  lista = ["manzana", "platano", "naranja", "limno", "kiwi"]
  lista_nueva = []

  for x in lista:
    if "a" in x:
      lista_nueva.append(x)

  print(lista_nueva)
  \end{lstlisting}
\end{frame}

\begin{frame}[fragile]
  \frametitle{Comprensión de listas - Ejemplo}

  Con la comprensión de listas, puede hacer todo eso con
  solo una línea de código:

  \vspace{\baselineskip}
  \begin{lstlisting}[language=Python]
  lista = ["manzana", "platano", "naranja", "limon", "kiwi"]
  lista_nueva = [x for x in lista if "a" in x]

  print(lista_nueva)
  \end{lstlisting}
\end{frame}
