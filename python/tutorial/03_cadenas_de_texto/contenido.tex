% ex: ts=2 sw=2 sts=2 et filetype=tex
% SPDX-License-Identifier: CC-BY-SA-4.0

\section{Cadenas ("\emph{Strings}")}

\begin{frame}[fragile]
  \frametitle{Cadenas de texto}

  Las cadenas en Python están rodeadas por comillas simples o comillas dobles.

  \vspace{\baselineskip}
  \textbf{'hola'} es lo mismo que \textbf{"hola"}.

  \vspace{\baselineskip}
  Se puede mostrar una cadena literal con la función
  \textcolor{codeKeyword}{print}():

  \vspace{\baselineskip}
  \begin{lstlisting}[language=Python]
  print("Hola")
  print('Hola')
  \end{lstlisting}
\end{frame}

\begin{frame}[fragile]
  \frametitle{Asignar una cadena a una variable}

  La asignación de una cadena a una variable se realiza con el nombre de
  la variable seguido de un signo igual y la cadena:

  \vspace{\baselineskip}
  \begin{lstlisting}[language=Python]
  a = "Hola"
  print(a)
  \end{lstlisting}
\end{frame}

\begin{frame}[fragile]
  \frametitle{Asignar una cadena a una variable}

  Puede asignar una cadena de varias líneas a una variable
  utilizando tres comillas:

  \begin{lstlisting}[language=Python]
a = """Lorem ipsum dolor sit amet,
consectetur adipiscing elit,
sed do eiusmod tempor incididunt
ut labore et dolore magna aliqua."""
print(a) 
  \end{lstlisting}
\end{frame}

\begin{frame}[fragile]
  \frametitle{Asignar una cadena a una variable}

  O tres comillas simples: 

  \begin{lstlisting}[language=Python]
a = '''Lorem ipsum dolor sit amet,
consectetur adipiscing elit,
sed do eiusmod tempor incididunt
ut labore et dolore magna aliqua.'''
print(a) 
  \end{lstlisting}

  \begin{alertblock}{Nota}
    en el resultado, los saltos de línea se insertan en la misma
    posición que en el código.
  \end{alertblock}
\end{frame}

\begin{frame}[fragile]
  \frametitle{Asignar una cadena a una variable}

  Como muchos otros lenguajes de programación populares, las cadenas
  en Python son arreglos de bytes que representan caracteres Unicode.

  \vspace{\baselineskip}
  Sin embargo, Python no tiene un tipo de datos de carácter, un solo
  carácter es simplemente una cadena con una longitud de 1.

  \vspace{\baselineskip}
  Se pueden usar corchetes para acceder a elementos de la cadena. 

  \begin{lstlisting}[language=Python]
  a = "Hola Mundo"
  #    0123456789
  print(a[1])
  \end{lstlisting}
\end{frame}

\begin{frame}[fragile]
  \frametitle{Recorriendo una cadena}

  Dado que las cadenas son arreglos, podemos recorrer los caracteres
  de una cadena, con un bucle \textcolor{codeKeyword}{for}.

  \vspace{\baselineskip}
  \begin{lstlisting}[language=Python]
  for x in "manzana":
    print(x)
  \end{lstlisting}
\end{frame}

\begin{frame}[fragile]
  \frametitle{Longitud de una cadena}

  Para obtener la longitud de una cadena, se usa la función 
  \textcolor{codeKeyword}{len}().

  \vspace{\baselineskip}
  \begin{lstlisting}[language=Python]
  a = "Hola Mundo!"
  print(len(a))
  \end{lstlisting}
\end{frame}

\begin{frame}[fragile]
  \frametitle{Comprobando cadenas}

  Para comprobar si una determinada frase o carácter está presente
  en una cadena, podemos usar la palabra clave
  \textcolor{codeKeyword}{in}.

  \begin{lstlisting}[language=Python]
  txt= "Las mejores cosas en la vida son gratis!"
  print("gratis" in txt)
  \end{lstlisting}

  \pausa
  Se puede usar con la sentencia \textcolor{codeKeyword}{if}:

  \begin{lstlisting}[language=Python]
  txt= "Las mejores cosas en la vida son gratis!"
  if "gratis" in txt:
    print("Si, 'gratis' esta en la cadena.")
  \end{lstlisting}
\end{frame}

\begin{frame}[fragile]
  \frametitle{Comprobando cadenas}

  Para comprobar si una determinada frase o carácter \textbf{no} 
  está presente en una cadena, podemos usar la palabra clave
  \textcolor{codeKeyword}{not in}.

  \begin{lstlisting}[language=Python]
  txt= "Las mejores cosas en la vida son gratis!"
  print("costoso" not in txt)
  \end{lstlisting}

  \pausa
  Se puede usar con la sentencia \textcolor{codeKeyword}{if}:

  \begin{lstlisting}[language=Python]
  txt= "Las mejores cosas en la vida son gratis!"
  if "costoso" not in txt:
    print("No, 'costoso' NO esta en la cadena.")
  \end{lstlisting}
\end{frame}


\section{Subcadenas}

\begin{frame}[fragile]
  \frametitle{Subcadenas}

  Puede devolver un rango de caracteres utilizando la sintaxis de
  corte/intervalo.

  \vspace{\baselineskip}
  Especifique el índice inicial y el índice final, separados por dos puntos,
  para devolver una parte de la cadena. 

  \vspace{\baselineskip}
  Ejemplo: Obtener los caracteres de la posición 2 a la posición 5 (no incluida):
  \begin{lstlisting}[language=Python]
  b = "Hola Mundo!"
  print(b[2:5])
  \end{lstlisting}

  \begin{exampleblock}{Nota}
    El primer caracter de una cadena tiene indice 0
  \end{exampleblock}
\end{frame}

\begin{frame}[fragile]
  \frametitle{Subcadena desde el principio}

  Al omitir el índice de inicio, el rango comenzará en el primer carácter:

  \vspace{\baselineskip}
  Ejemplo: Obtener los caracteres desde el principio hasta la posición 5 (no incluida):
  \begin{lstlisting}[language=Python]
  b = "Hola Mundo!"
  print(b[:5])
  \end{lstlisting}
\end{frame}

\begin{frame}[fragile]
  \frametitle{Subcadena hasta el final}

  Al omitir el índice final, el rango llegará al final:

  \vspace{\baselineskip}
  Ejemplo: Obtener los caracteres de la posición 2 hasta el final:
  \begin{lstlisting}[language=Python]
  b = "Hola Mundo!"
  print(b[2:])
  \end{lstlisting}
\end{frame}

\begin{frame}[fragile]
  \frametitle{Indices negativos}

  Use índices negativos para comenzar el segmento desde el final de la cadena:
  \vspace{\baselineskip}
  Ejemplo: Obtener los caracteres de la posición -5 hasta la posición -2 (no
  incluida)
  \begin{lstlisting}[language=Python]
  b = "Hola Mundo!"
  print(b[-5:-2])
  \end{lstlisting}
\end{frame}


\section{Modificando cadenas}

\begin{frame}[fragile]
  \frametitle{Mayúsculas}

  Python tiene un conjunto de métodos integrados que puede usar en cadenas.

  \vspace{\baselineskip}
  Ejemplo: El método \textcolor{codeKeyword}{upper}() devuelve la
  cadena en mayúsculas: 
  \begin{lstlisting}[language=Python]
  a = "Hola Mundo!"
  print(a.upper())
  \end{lstlisting}
\end{frame}

\begin{frame}[fragile]
  \frametitle{Minúsculas}

  El método \textcolor{codeKeyword}{lower}() devuelve la
  cadena en minúsculas: 

  \begin{lstlisting}[language=Python]
  a = "Hola Mundo!"
  print(a.lower())
  \end{lstlisting}
\end{frame}

\begin{frame}[fragile]
  \frametitle{Quitando los espacios en blanco}

  El espacio en blanco es el espacio antes y/o después del texto real,
  y muy a menudo se desea eliminar este espacio.

  \vspace{\baselineskip}
  El método \textcolor{codeKeyword}{strip}() quita cualquier espacio en
  blanco al principio y al final de la cadena:

  \begin{lstlisting}[language=Python]
  a = "  Hola Mundo!  "
  print(a.strip()) # regresa "Hola Mundo!"
  \end{lstlisting}
\end{frame}

\begin{frame}[fragile]
  \frametitle{Remplazando cadenas}

  El método \textcolor{codeKeyword}{replace}() remplaza una cadena por otra:

  \vspace{\baselineskip}
  \begin{lstlisting}[language=Python]
  a = "Hola Mundo!"
  print(a.replace("H","L"))
  \end{lstlisting}
\end{frame}

\begin{frame}[fragile]
  \frametitle{Dividiendo cadenas}

  El método \textcolor{codeKeyword}{split}() devuelve una lista donde
  el texto entre el separador especificado se convierte en los elementos
  de la lista.

  \vspace{\baselineskip}
  El método \textcolor{codeKeyword}{split}() divide la cadena en
  subcadenas si encuentra instancias del separador:
  \begin{lstlisting}[language=Python]
  a = "Hola,Mundo"
  print(a.split(",")) # imprime ['Hola', 'Mundo']
  \end{lstlisting}
\end{frame}


\section{Concatenación}

\begin{frame}[fragile]
  \frametitle{Concatenación}

  Para concatenar o combinar dos cadenas, puede usar el operador +.

  \begin{lstlisting}[language=Python]
  a = "Hola"
  b = "Mundo"
  c = a + b
  print(c)
  c = a +  " " + b
  print(c)
  \end{lstlisting}
\end{frame}


\section{Caracteres especiales}

\begin{frame}[fragile]
  \frametitle{Caracteres de escape}

  Para insertar caracteres que son ilegales en una cadena se
  usa un carácter de escape.

  \vspace{\baselineskip}
  Un carácter de escape es una diagonal invertida \textbackslash  seguida del
  carácter que desea insertar.

  \vspace{\baselineskip}
  Un ejemplo de un carácter ilegal es una comilla doble dentro
  de una cadena que está rodeada por comillas dobles: 

  \vspace{\baselineskip}
  \begin{lstlisting}[language=Python]
  txt = "El se hace llamar "Hackerman", porque le gustan las computadoras"
  \end{lstlisting}
\end{frame}

\begin{frame}[fragile]
  \frametitle{Caracteres de escape}

  Para arreglar este problema se puede usar comillas simples

  \vspace{\baselineskip}
  \begin{lstlisting}[language=Python]
  txt = 'El se hace llamar "Hackerman", porque le gustan las computadoras'
  \end{lstlisting}

  \vspace{\baselineskip}
  o el carácter especial \textbackslash"

  \vspace{\baselineskip}
  \begin{lstlisting}[language=Python]
  txt = "El se hace llamar \"Hackerman\", porque le gustan las computadoras"
  \end{lstlisting}
\end{frame}

\begin{frame}[c]{Caracteres de escape}

  Otros caracteres de escape usados en Python son:

  \begin{table}[]
  \begin{tabular}{cl}
    \textbf{Código} &  \textbf{Resultado} \\
    \rowcolor{light-gray}
    \textbackslash'  & Comilla simple \pausa \\
    \textbackslash\textbackslash  & Diagonal invertida \pausa \\
    \rowcolor{light-gray}
    \textbackslash{n}  & Nueva linea \pausa \\
    \textbackslash{r}  & Retorno de carro \pausa \\
    \rowcolor{light-gray}
    \textbackslash{t}  & Tabulador \pausa \\
    \textbackslash{b}  & "Backspace" \pausa \\
    \rowcolor{light-gray}
    \textbackslash{f}  & "From Feed" \pausa \\
    \textbackslash{ooo}  & Valor octal \pausa \\
    \rowcolor{light-gray}
    \textbackslash{xhh}  & Valor hexadecimal \\
  \end{tabular}
  \end{table}
\end{frame}

\section{Métodos de cadena de texto}

\begin{frame}[c]{Métodos de cadena de texto}

  \vspace{\baselineskip}
  Python tiene un conjunto de métodos integrados que puede usar en cadenas.

  \begin{exampleblock}{Nota}
    Todos los métodos de cadena devuelven nuevos valores.
    No cambian la cadena original.
  \end{exampleblock}

  \begin{table}[]
  \begin{tabular}{ll}
    \textbf{Método} &  \textbf{Descripción} \\
    \rowcolor{light-gray}
    capitalize() & Convierte el primer carácter a mayúsculas \pausa \\
    casefold() & Convierte la cadena en minúsculas \pausa \\
    \rowcolor{light-gray}
    center() & Devuelve una cadena centrada \pausa \\
    count() & Devuelve el número de veces que ocurre un \\
            & valor especificado en una cadena \pausa \\
    \rowcolor{light-gray}
    encode() & Devuelve una versión codificada de la cadena \pausa \\
    endswith() & Devuelve verdadero si la cadena termina con \\
               & el valor especificado \pausa \\
    \rowcolor{light-gray}
    expandtabs() & Establece el tamaño de pestaña de la cadena \\
  \end{tabular}
  \end{table}
\end{frame}

\begin{frame}[c]{Métodos de cadena de texto}

  \vspace{\baselineskip}
  \begin{table}[]
  \begin{tabular}{ll}
    \textbf{Método} &  \textbf{Descripción} \\
    \rowcolor{light-gray}
    find() & Busca en la cadena un valor especificado y \\
    \rowcolor{light-gray}
           & devuelve la posición de donde se encontró \pausa \\
    format() & Formatea valores especificados en una cadena \pausa \\
    \rowcolor{light-gray}
    format\_map() & Formatea valores especificados en una cadena \pausa \\
    index()   & Busca en la cadena un valor especificado y devuelve \\
              & la posición de donde se encontró \pausa \\
    \rowcolor{light-gray}
    isalnum() & Devuelve verdadero si todos los caracteres de la \\
    \rowcolor{light-gray}
              & cadena son alfanuméricos \pausa \\
    isalpha() & Devuelve verdadero si todos los caracteres de la \\
              & cadena están en el alfabeto \pausa \\
    \rowcolor{light-gray}
    isascii() & Devuelve verdadero si todos los caracteres de la \\
    \rowcolor{light-gray}
              & cadena son caracteres ASCII \pausa \\
    isdecimal() & Devuelve verdadero si todos los caracteres de la \\
              & cadena son decimales \\
  \end{tabular}
  \end{table}
\end{frame}

\begin{frame}[c]{Métodos de cadena de texto}

  \vspace{\baselineskip}
  \begin{table}[]
  \begin{tabular}{ll}
    \textbf{Método} &  \textbf{Descripción} \\
    \rowcolor{light-gray}
    isdigit() & Devuelve True si todos los caracteres de la cadena \\
    \rowcolor{light-gray}
              & son dígitos \pausa \\
    isidentifier() & Devuelve verdadero si la cadena es un identificador \pausa \\
    \rowcolor{light-gray}
    islower() & Devuelve True si todos los caracteres de la cadena \\
    \rowcolor{light-gray}
              & son minúsculas \pausa \\
    isnumeric() & Devuelve verdadero si todos los caracteres de la cadena \\
                & son numéricos \pausa \\
    \rowcolor{light-gray}
    isprintable() & Devuelve True si todos los caracteres de la cadena \\
    \rowcolor{light-gray}
                  & son imprimibles \pausa \\
    isspace() & Devuelve True si todos los caracteres de la cadena \\
              & son espacios en blanco \pausa \\
    \rowcolor{light-gray}
    istitle() & Devuelve True si la cadena sigue las reglas de un título \pausa \\
    isupper() & Devuelve verdadero si todos los caracteres de la  \\
              & cadena están en mayúsculas \\
  \end{tabular}
  \end{table}
\end{frame}

\begin{frame}[c]{Métodos de cadena de texto}

  \begin{table}[]
  \begin{tabular}{ll}
    \textbf{Método} &  \textbf{Descripción} \\
    \rowcolor{light-gray}
    join()   & Convierte los elementos de un iterable en una cadena \pausa \\
    ljust()  & Devuelve una versión justificada a la izquierda de la cadena \pausa \\
    \rowcolor{light-gray}
    lower()  & Convierte una cadena en minúsculas \pausa \\
    lstrip() & Devuelve una versión de recorte a la izquierda de la cadena \pausa \\
    \rowcolor{light-gray}
    maketrans()  & Devuelve una tabla de traducción para usar en traducciones \pausa \\
    partition()  & Devuelve una tupla donde la cadena se divide en tres partes \pausa \\
    \rowcolor{light-gray}
    replace() & Devuelve una cadena donde un valor especificado se \\
    \rowcolor{light-gray}
                 &reemplaza con un valor especificado \\
  \end{tabular}
  \end{table}
\end{frame}

\begin{frame}[c]{Métodos de cadena de texto}

  \begin{table}[]
  \begin{tabular}{ll}
    \textbf{Método} &  \textbf{Descripción} \\
    \rowcolor{light-gray}
    rfind()  & Busca en la cadena un valor especificado y devuelve la última \\
    \rowcolor{light-gray}
             & posición de donde se encontró  \pausa \\
    rindex() & Busca en la cadena un valor especificado y devuelve la última \\
             & posición de donde se encontró \pausa \\
    \rowcolor{light-gray}
    rjust()  & Devuelve una versión justificada a la derecha de la cadena \pausa \\
    rpartition() & Devuelve una tupla donde la cadena se divide en tres partes \pausa \\
    \rowcolor{light-gray}
    rsplit() & Divide la cadena en el separador especificado y devuelve una lista \pausa \\
    rstrip() & Devuelve una versión de recorte a la derecha de la cadena \pausa \\
    \rowcolor{light-gray}
    split()  & Divide la cadena en el separador especificado y devuelve una lista \pausa \\
    splitlines() & Divide la cadena en los saltos de línea y devuelve una lista \\
  \end{tabular}
  \end{table}
\end{frame}

\begin{frame}[c]{Métodos de cadena de texto}

  \begin{table}[]
  \begin{tabular}{ll}
    \textbf{Método} &  \textbf{Descripción} \\
    \rowcolor{light-gray}
    startswith() & Devuelve verdadero si la cadena comienza con el valor
                   especificado \pausa \\
    strip() & Devuelve una versión recortada de la cadena \pausa \\
    \rowcolor{light-gray}
    swapcase() & Intercambia mayúsculas y minúsculas, y viceversa \pausa \\
    title() & Convierte el primer carácter de cada palabra a mayúsculas \pausa \\
    \rowcolor{light-gray}
    translate() & Devuelve una cadena traducida \pausa \\
    upper() & Convierte una cadena en mayúsculas \pausa \\
    \rowcolor{light-gray}
    zfill() & Rellena la cadena con un número especificado de valores 0 al
    principio  \pausa \\
  \end{tabular}
  \end{table}

  \begin{exampleblock}{Nota}
    Todos los métodos de cadena devuelven nuevos valores.
    No cambian la cadena original.
  \end{exampleblock}
\end{frame}

\section{El método format()}

\begin{frame}[c]{El método format()}

  El método \textcolor{codeKeyword2}{format}() formatea los valores especificados y
  los inserta dentro del marcador de posición de la cadena.
  El marcador de posición se define mediante llaves: \{\}.

  El método \textcolor{codeKeyword2}{format}() devuelve la cadena formateada.

  \vspace{\baselineskip}
  \textbf{Sintaxis}:

  \vspace{\baselineskip}
  cadena.\textcolor{codeKeyword2}{format}(valor1, valor2...)

  \vspace{\baselineskip}
  \textbf{Parámetros}:
  \begin{description}
    \item[valor1, valor2...] Requerido. Uno o más valores que deben
      formatearse e insertarse en la cadena.
      Los valores son una lista de valores separados por comas, una lista
      clave = valor o una combinación de ambos.
      Los valores pueden ser de cualquier tipo de datos.
  \end{description}
\end{frame}

\begin{frame}[fragile]
  \frametitle{Los marcadores de posición}

  Los marcadores de posición se pueden identificar mediante índices
  con nombre \textbf{\{precio\}}, índices numerados \textbf{\{0\}} o
  incluso marcadores de posición vacíos \textbf{\{\}}.

  \vspace{\baselineskip}
  \begin{lstlisting}[language=Python]
txt1 = "Mi nombre es {nom}, tengo {edad}".format(nom= "Juan", edad = 36)
txt2 = "Mi nombre es {0}, tengo {1}".format("Juan",36)
txt3 = "Mi nombre es {}, tengo {}".format("Juan",36)
  \end{lstlisting}
\end{frame}

\begin{frame}[c]{Tipos de formato}

  Dentro de los marcadores de posición, puede agregar un tipo de
  formato para formatear el resultado:

  \begin{table}[]
  \begin{tabular}{ll}
    \textbf{Formato} &  \textbf{Descripción} \\
    \rowcolor{light-gray}
    :<  & Alinea el resultado a la izquierda (dentro del espacio disponible) \pausa \\
    :>  & Alinea el resultado a la derecha (dentro del espacio disponible) \pausa \\
    \rowcolor{light-gray}
    :\^{}  & Alinea el resultado al centro (dentro del espacio disponible) \pausa \\
    :=  & Coloca el signo en la posición más a la izquierda \pausa \\
    \rowcolor{light-gray}
    :+  & Utilice un signo más para indicar si el resultado es positivo
          o negativo \pausa \\
    :-  & Utilice un signo menos solo para valores negativos \pausa \\
    \rowcolor{light-gray}
    :   & Use un espacio para insertar un espacio adicional antes de los \\
    \rowcolor{light-gray}
        & números positivos (y un signo menos antes de los números negativos) \pausa \\
    :,  & Usa una coma como separador de miles  \pausa \\
    \rowcolor{light-gray}
    :\_ & Utilice un guión bajo como separador de miles \\
  \end{tabular}
  \end{table}
\end{frame}

\begin{frame}[c]{Tipos de formato}

  \begin{table}[]
  \begin{tabular}{ll}
    \textbf{Formato} &  \textbf{Descripción} \\
    \rowcolor{light-gray}
    :b & Formato binario \pausa \\
    :c & Convierte el valor en el carácter Unicode correspondiente \pausa \\
    \rowcolor{light-gray}
    :d & Formato decimal \pausa \\
    :e & Formato científico, con minúscula e \pausa \\
    \rowcolor{light-gray}
    :E & Formato científico, con una E mayúscula \pausa \\
    :f & Formato de número de punto fijo \pausa \\
    \rowcolor{light-gray}
    :F & Formato de número de punto fijo, en mayúsculas \\
    \rowcolor{light-gray}
       & (muestra inf y nan como INF y NAN) \pausa \\
    :g & Formato general \pausa \\
    \rowcolor{light-gray}
    :G & Formato general (usando una E mayúscula para notaciones
         científicas) \\
  \end{tabular}
  \end{table}
\end{frame}

\begin{frame}[c]{Tipos de formato}

  \begin{table}[]
  \begin{tabular}{ll}
    \textbf{Formato} &  \textbf{Descripción} \\
    \rowcolor{light-gray}
    :o  & Formato octal \pausa \\
    :x  & Formato hexadecimal, minúsculas \pausa \\
    \rowcolor{light-gray}
    :X  & Formato hexadecimal, mayúsculas \pausa \\
    :n  & Formato de número \pausa \\
    \rowcolor{light-gray}
    :\% & Formato de porcentaje \\
  \end{tabular}
  \end{table}
\end{frame}
