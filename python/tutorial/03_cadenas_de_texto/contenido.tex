% ex: ts=2 sw=2 sts=2 et filetype=tex
% SPDX-License-Identifier: CC-BY-SA-4.0

\section{Cadenas ("\emph{Strings}")}

\begin{frame}[fragile]
  \frametitle{Cadenas de texto}

  Las cadenas en Python están rodeadas por comillas simples o comillas dobles.

  \vspace{\baselineskip}
  \textbf{'hola'} es lo mismo que \textbf{"hola"}.

  \vspace{\baselineskip}
  Se puede mostrar una cadena literal con la función
  \textcolor{codeKeyword}{print}():

  \vspace{\baselineskip}
  \begin{lstlisting}[language=Python]
  print("Hola")
  print('Hola')
  \end{lstlisting}
\end{frame}

\begin{frame}[fragile]
  \frametitle{Asignar una cadena a una variable}

  La asignación de una cadena a una variable se realiza con el nombre de
  la variable seguido de un signo igual y la cadena:

  \vspace{\baselineskip}
  \begin{lstlisting}[language=Python]
  a = "Hola"
  print(a)
  \end{lstlisting}
\end{frame}

\begin{frame}[fragile]
  \frametitle{Asignar una cadena a una variable}

  Puede asignar una cadena de varias líneas a una variable
  utilizando tres comillas:

  \begin{lstlisting}[language=Python]
a = """Lorem ipsum dolor sit amet,
consectetur adipiscing elit,
sed do eiusmod tempor incididunt
ut labore et dolore magna aliqua."""
print(a) 
  \end{lstlisting}
\end{frame}

\begin{frame}[fragile]
  \frametitle{Asignar una cadena a una variable}

  O tres comillas simples: 

  \begin{lstlisting}[language=Python]
a = '''Lorem ipsum dolor sit amet,
consectetur adipiscing elit,
sed do eiusmod tempor incididunt
ut labore et dolore magna aliqua.'''
print(a) 
  \end{lstlisting}

  \begin{alertblock}{Nota}
    en el resultado, los saltos de línea se insertan en la misma
    posición que en el código.
  \end{alertblock}
\end{frame}

\begin{frame}[fragile]
  \frametitle{Asignar una cadena a una variable}

  Como muchos otros lenguajes de programación populares, las cadenas
  en Python son arreglos de bytes que representan caracteres Unicode.

  \vspace{\baselineskip}
  Sin embargo, Python no tiene un tipo de datos de carácter, un solo
  carácter es simplemente una cadena con una longitud de 1.

  \vspace{\baselineskip}
  Se pueden usar corchetes para acceder a elementos de la cadena. 

  \begin{lstlisting}[language=Python]
  a = "Hola Mundo"
  #    0123456789
  print(a[1])
  \end{lstlisting}
\end{frame}

\begin{frame}[fragile]
  \frametitle{Recorriendo una cadena}

  Dado que las cadenas son arreglos, podemos recorrer los caracteres
  de una cadena, con un bucle \textcolor{codeKeyword}{for}.

  \vspace{\baselineskip}
  \begin{lstlisting}[language=Python]
  for x in "manzana":
    print(x)
  \end{lstlisting}
\end{frame}

\begin{frame}[fragile]
  \frametitle{Longitud de una cadena}

  Para obtener la longitud de una cadena, se usa la función 
  \textcolor{codeKeyword}{len}().

  \vspace{\baselineskip}
  \begin{lstlisting}[language=Python]
  a = "Hola Mundo!"
  print(len(a))
  \end{lstlisting}
\end{frame}

\begin{frame}[fragile]
  \frametitle{Comprobando cadenas}

  Para comprobar si una determinada frase o carácter está presente
  en una cadena, podemos usar la palabra clave
  \textcolor{codeKeyword}{in}.

  \begin{lstlisting}[language=Python]
  txt= "Las mejores cosas en la vida son gratis!"
  print("gratis" in txt)
  \end{lstlisting}

  \pausa
  Se puede usar con la sentencia \textcolor{codeKeyword}{if}:

  \begin{lstlisting}[language=Python]
  txt= "Las mejores cosas en la vida son gratis!"
  if "free" in txt:
    print("Si, 'gratis' esta en la cadena.")
  \end{lstlisting}
\end{frame}

\begin{frame}[fragile]
  \frametitle{Comprobando cadenas}

  Para comprobar si una determinada frase o carácter \textbf{no} 
  está presente en una cadena, podemos usar la palabra clave
  \textcolor{codeKeyword}{not in}.

  \begin{lstlisting}[language=Python]
  txt= "Las mejores cosas en la vida son gratis!"
  print("costoso" not in txt)
  \end{lstlisting}

  \pausa
  Se puede usar con la sentencia \textcolor{codeKeyword}{if}:

  \begin{lstlisting}[language=Python]
  txt= "Las mejores cosas en la vida son gratis!"
  if "costoso" not in txt:
    print("No, 'costoso' NO esta en la cadena.")
  \end{lstlisting}
\end{frame}


\section{Subcadenas}

\begin{frame}[fragile]
  \frametitle{Subcadenas}

  Puede devolver un rango de caracteres utilizando la sintaxis de
  corte/intervalo.

  \vspace{\baselineskip}
  Especifique el índice inicial y el índice final, separados por dos puntos,
  para devolver una parte de la cadena. 

  \vspace{\baselineskip}
  Ejemplo: Obtener los caracteres de la posición 2 a la posición 5 (no incluida):
  \begin{lstlisting}[language=Python]
  b = "Hola Mundo!"
  print(b[2:5])
  \end{lstlisting}

  \begin{exampleblock}{Nota}
    El primer caracter de una cadena tiene indice 0
  \end{exampleblock}
\end{frame}

\begin{frame}[fragile]
  \frametitle{Subcadena desde el principio}

  Al omitir el índice de inicio, el rango comenzará en el primer carácter:

  \vspace{\baselineskip}
  Ejemplo: Obtener los caracteres desde el principio hasta la posición 5 (no incluida):
  \begin{lstlisting}[language=Python]
  b = "Hola Mundo!"
  print(b[:5])
  \end{lstlisting}
\end{frame}

\begin{frame}[fragile]
  \frametitle{Subcadena hasta el final}

  Al omitir el índice final, el rango llegará al final:

  \vspace{\baselineskip}
  Ejemplo: Obtener los caracteres de la posición 2 hasta el final:
  \begin{lstlisting}[language=Python]
  b = "Hola Mundo!"
  print(b[2:])
  \end{lstlisting}
\end{frame}

\begin{frame}[fragile]
  \frametitle{Indices negativos}

  Use índices negativos para comenzar el segmento desde el final de la cadena:
  \vspace{\baselineskip}
  Ejemplo: Obtener los caracteres de la posición -5 hasta la posición -2 (no
  incluida)
  \begin{lstlisting}[language=Python]
  b = "Hola Mundo!"
  print(b[-5:-2])
  \end{lstlisting}
\end{frame}


\section{Modificando cadenas}

\begin{frame}[fragile]
  \frametitle{Mayúsculas}

  Python tiene un conjunto de métodos integrados que puede usar en cadenas.

  \vspace{\baselineskip}
  Ejemplo: El método \textcolor{codeKeyword}{upper}() devuelve la
  cadena en mayúsculas: 
  \begin{lstlisting}[language=Python]
  a = "Hola Mundo!"
  print(a.upper())
  \end{lstlisting}
\end{frame}

\begin{frame}[fragile]
  \frametitle{Minúsculas}

  El método \textcolor{codeKeyword}{lower}() devuelve la
  cadena en minúsculas: 

  \begin{lstlisting}[language=Python]
  a = "Hola Mundo!"
  print(a.lower())
  \end{lstlisting}
\end{frame}

\begin{frame}[fragile]
  \frametitle{Quitando los espacios en blanco}

  El espacio en blanco es el espacio antes y/o después del texto real,
  y muy a menudo se desea eliminar este espacio.

  \vspace{\baselineskip}
  El método \textcolor{codeKeyword}{strip}() quita cualquier espacio en
  blanco al principio y al final de la cadena:

  \begin{lstlisting}[language=Python]
  a = "  Hola Mundo!  "
  print(a.strip()) # regresa "Hola Mundo!"
  \end{lstlisting}
\end{frame}

\begin{frame}[fragile]
  \frametitle{Remplazando cadenas}

  El método \textcolor{codeKeyword}{replace}() remplaza una cadena por otra:

  \vspace{\baselineskip}
  \begin{lstlisting}[language=Python]
  a = "Hola Mundo!"
  print(a.replace("H","L"))
  \end{lstlisting}
\end{frame}

\begin{frame}[fragile]
  \frametitle{Dividiendo cadenas}

  El método \textcolor{codeKeyword}{split}() devuelve una lista donde
  el texto entre el separador especificado se convierte en los elementos
  de la lista.

  \vspace{\baselineskip}
  El método \textcolor{codeKeyword}{split}() divide la cadena en
  subcadenas si encuentra instancias del separador:
  \begin{lstlisting}[language=Python]
  a = "Hola,Mundo"
  print(a.split(","))
  \end{lstlisting}
\end{frame}


\section{Concatenación}

\begin{frame}[fragile]
  \frametitle{Mayúsculas}

  Para concatenar o combinar dos cadenas, puede usar el operador +.

  \begin{lstlisting}[language=Python]
  a = "Hola"
  b = "Mundo"
  c = a + b
  print(c)
  c = a +  " " + b
  print(c)
  \end{lstlisting}
\end{frame}


\section{Caracteres especiales}

\begin{frame}[fragile]
  \frametitle{Caracteres de escape}

  Para insertar caracteres que son ilegales en una cadena se
  usa un carácter de escape.

  \vspace{\baselineskip}
  Un carácter de escape es una diagonal invertida \textbackslash seguida del
  carácter que desea insertar.

  \vspace{\baselineskip}
  Un ejemplo de un carácter ilegal es una comilla doble dentro
  de una cadena que está rodeada por comillas dobles: 

  \vspace{\baselineskip}
  \begin{lstlisting}[language=Python]
  txt = "El se hace llamar "Hackerman", porque le gustan las computadoras"
  \end{lstlisting}
\end{frame}

\begin{frame}[fragile]
  \frametitle{Caracteres de escape}

  Para arreglar este problema se usa el carácter especial \textbackslash"

  \vspace{\baselineskip}
  \begin{lstlisting}[language=Python]
  txt = "El se hace llamar \"Hackerman\", porque le gustan las computadoras"
  \end{lstlisting}
\end{frame}

\begin{frame}[c]{Caracteres de escape}

  Otros caracteres de escape usados en Python son:

  \begin{table}[]
  \begin{tabular}{cl}
    \textbf{Código} &  \textbf{Resultado} \\
    \rowcolor{light-gray}
    \textbackslash'  & Comilla simple \pausa \\
    \textbackslash\textbackslash  & Diagonal invertida \pausa \\
    \rowcolor{light-gray}
    \textbackslash{n}  & Nueva linea \pausa \\
    \textbackslash{r}  & Retorno de carro \pausa \\
    \rowcolor{light-gray}
    \textbackslash{t}  & Tabulador \pausa \\
    \textbackslash{b}  & "Backspace" \pausa \\
    \rowcolor{light-gray}
    \textbackslash{f}  & "From Feed" \pausa \\
    \textbackslash{ooo}  & Valor octal \pausa \\
    \rowcolor{light-gray}
    \textbackslash{xhh}  & Valor hexadecimal \\
 \end{tabular}
  \end{table}
\end{frame}
