% ex: ts=2 sw=2 sts=2 et filetype=tex
% SPDX-License-Identifier: CC-BY-SA-4.0
\documentclass[aspectratio=169]{beamer}

\usetheme{Boadilla}

\setbeamertemplate{navigation symbols}{} % To remove the navigation symbols from the bottom of all slides

\graphicspath{{../../img/}{../../../styles/iteso/}}
% use the command "usepackage" instead of "usecolortheme" to use the path to styles
\usepackage{../../../styles/iteso/beamercolorthemeiteso}
\usepackage[utf8]{inputenc}
\usepackage{graphicx} % Allows including images
\usepackage{listings}

\title{Introducción a Python} % The short title appears at the bottom of every slide,
                                               % the full title is only on the title page
\author{Miguel Bernal Marin} % Your name
\institute[ITESO] % Your institution as it will appear on the bottom of every slide, may be shorthand to save space
{
 ITESO, Universidad \\
 Jesuita de Guadalajara \\% Your institution for the title page
\medskip
\textit{miguel.bernal@iteso.mx} % Your email address
}
\date{
    \today
} % Date, can be changed to a custom date

%New colors defined below
\definecolor{codegreen}{rgb}{0,0.6,0}
\definecolor{codegray}{rgb}{0.5,0.5,0.5}
\definecolor{codepurple}{rgb}{0.58,0,0.82}
\definecolor{backcolour}{rgb}{0.95,0.95,0.92}

%Code listing style named "mystyle"
\lstdefinestyle{mystyle}{
  backgroundcolor=\color{backcolour},   commentstyle=\color{codegreen},
  keywordstyle=\color{magenta},
  numberstyle=\tiny\color{codegray},
  stringstyle=\color{codepurple},
  basicstyle=\ttfamily\footnotesize,
  breakatwhitespace=false,
  breaklines=true,
  captionpos=b,
  keepspaces=true,
  numbers=left,
  numbersep=5pt,
  showspaces=false,
  showstringspaces=false,
  showtabs=false,
  tabsize=2
}

%"mystyle" code listing set
\lstset{style=mystyle}

%------------------------------------------------------------
%The next block of commands puts the table of contents at the
%beginning of each section and highlights the current section:
\AtBeginSection[]
{
  \begin{frame}
    \frametitle{Contenido}
    \tableofcontents[currentsection]
  \end{frame}
}
%------------------------------------------------------------

\begin{document}
{ % Portada de la universidad
  \setbeamertemplate{footline}{}
  \usebackgroundtemplate{\includegraphics[width=\paperwidth]{fondo-portada.png}}
  \begin{frame}
  \end{frame}
}

\begin{frame}
    \titlepage
\end{frame}

\usebackgroundtemplate{\includegraphics[width=\paperwidth]{fondo.png}}

% ex: ts=2 sw=2 sts=2 et filetype=tex
% SPDX-License-Identifier: CC-BY-SA-4.0
{
  \setbeamertemplate{footline}{}
  \setbeamercolor{frametitle}{fg=itesoprofundo}
  \usebackgroundtemplate{\includegraphics[width=\paperwidth]{04-dvps.png}}
\begin{frame}
    \frametitle{Contenido}
    \tableofcontents
\end{frame}
}

\section{Tipos de Datos}

\begin{frame}[c]{Tipo de datos integrados}

  En programación, el tipo de dato es un concepto importante.

  \vspace{\baselineskip}
  Las variables pueden almacenar datos de diferentes tipos y diferentes
  tipos pueden hacer cosas diferentes.

  \vspace{\baselineskip}
  Python tiene los siguientes tipos de datos integrados de forma
  predeterminada, en estas categorías:

  \begin{table}[]
  \begin{tabular}{ll}
    Texto & \textcolor{codeKeyword2}{str} \\
    \pausa
    Numérico & \textcolor{codeKeyword2}{int}, \textcolor{codeKeyword2}{float},
     \textcolor{codeKeyword2}{complex} \\
    \pausa
    Secuencia & \textcolor{codeKeyword2}{list}, \textcolor{codeKeyword2}{tuple},
     \textcolor{codeKeyword2}{range} \\
    \pausa
    Mapeo & \textcolor{codeKeyword2}{dict} \\
    \pausa
    Conjunto & \textcolor{codeKeyword2}{set},
     \textcolor{codeKeyword2}{frozenset} \\
    \pausa
    Boleano & \textcolor{codeKeyword2}{bool} \\
    \pausa
    Binario & \textcolor{codeKeyword2}{bytes},
     \textcolor{codeKeyword2}{bytearray}, \textcolor{codeKeyword2}{memoryview} \\
  \end{tabular}
  \end{table}
\end{frame}

\begin{frame}[fragile]
  \frametitle{Obteniendo el tipo de dato}

  Se puede obtener el tipo de dato con la función
  \textcolor{codeKeyword2}{type}()

  \vspace{\baselineskip}
  \begin{lstlisting}[language=Python]
  x = str(3)   # x guarda '3'
  y = int(3)   # y guarda 3
  z = float(3) # z guarda 3.0

  print( type(x) )
  print( type(y) )
  print( type(z) )
  \end{lstlisting}
\end{frame}

\begin{frame}[fragile]
  \frametitle{Configurando el tipo de dato}

  En Python, el tipo de dato es configurado/asociado cuando se asigna un
  valor a una variable:

  \begin{lstlisting}[language=Python]
x = "Hola Mundo"                      # str
x = 20                                # int
x = 20.5                              # float
x = 1j                                # complex
x = ["manzana", "platano", "naranja"] # list
x = ("manzana", "platano", "naranja") # tuple
x = range(6)                          # range
x = {"nombre" : "juan", "edad": 18}   # dict
x = {"manzana", "platano", "naranja"} # set
x = frozenset({"manzana", "platano", "naranja"}) # frozenset
x = True                              # bool
x = b"Hola"                           # bytes
x = bytearray(5)                      # bytearray
x = memoryview(bytes(5))              # memoryview
  \end{lstlisting}
\end{frame}

\begin{frame}[fragile]
  \frametitle{Configurando un tipo de dato especifico}

  Si se quiere especificar un tipo de dato especifico, se puede utilizar
  las siguientes funciones constructoras:

  \begin{lstlisting}[language=Python]
x = str("Hola Mundo")                        # str
x = int(20)                                  # int
x = float(20.5)                              # float
x = complex(1j)                              # complex
x = list(("manzana", "platano", "naranja"))  # list
x = tuple(("manzana", "platano", "naranja")) # tuple
x = range(6)                                 # range
x = dict(nombre = "juan", edad = 18)         # dict
x = set(("manzana", "platano", "naranja"))   # set
x = frozenset(("manzana", "platano", "naranja")) # frozenset
x = bool(5)                                  # bool
x = bytes(5)                                 # bytes
x = bytearray(5)                             # bytearray
x = memoryview(bytes(5))                     # memoryview
  \end{lstlisting}
\end{frame}

\section{Números en Python}

\begin{frame}[fragile]
  \frametitle{Números en Python}

  Hay tres tipos numéricos en Python:

  \begin{itemize}
    \item \textcolor{codeKeyword2}{int}
    \item \textcolor{codeKeyword2}{float}
    \item \textcolor{codeKeyword2}{complex}
  \end{itemize}

  Las variables numéricas son creadas cuando se le asigna su valor:

  \begin{lstlisting}[language=Python]
  x = 1    # int
  y = 2.8  # float
  z = j    # complex
  \end{lstlisting}

  \pausa
  Para verificar el tipo de dato de cualquier objeto en Python, se usa la
  función \textcolor{codeKeyword2}{type}():

  \begin{lstlisting}[language=Python]
  print( type(x) )
  print( type(y) )
  print( type(z) )
  \end{lstlisting}
\end{frame}

\begin{frame}[fragile]
  \frametitle{Números Enteros}

  Un entero (\textcolor{codeKeyword2}{int}) es un número positivo o
  negativo, sin decimales de longitud ilimitada.

  \begin{lstlisting}[language=Python]
  x = 1
  y = 35656222554887711
  z = -3255522

  print( type(x) )
  print( type(y) )
  print( type(z) )
  \end{lstlisting}
\end{frame}

\begin{frame}[fragile]
  \frametitle{Números de punto flotante}

  Un número de punto flotante (\textcolor{codeKeyword2}{float}), es un número
  positivo o negativo, que contiene uno o más decimales.

  \begin{lstlisting}[language=Python]
  x = 3.1416
  y = 1.0
  z = -35.59

  print( type(x) )
  print( type(y) )
  print( type(z) )
  \end{lstlisting}
\end{frame}

\begin{frame}[fragile]
  \frametitle{Números de punto flotante}

  Los número de punto flotante (\textcolor{codeKeyword2}{float}) pueden ser
  escritos como números en \emph{notación científica} con una "\textbf{e}"
  para indicar la potencia de 10.

  \begin{lstlisting}[language=Python]
  x = 35e3
  y = 12E4
  z = -87.7e100

  print( type(x) )
  print( type(y) )
  print( type(z) )
  \end{lstlisting}
\end{frame}

\begin{frame}[fragile]
  \frametitle{Números Complejos}

  Los números complejos son escritos con una "\textbf{j}" para indicar
  la parte imaginaria:

  \begin{lstlisting}[language=Python]
  x = 3+5j
  y = 5j
  z = -2.5j

  print( type(x) )
  print( type(y) )
  print( type(z) )
  \end{lstlisting}
\end{frame}

\begin{frame}[fragile]
  \frametitle{Conversión entre tipo de datos}

  Se pueden convertir de un tipo de dato a otro con las siguientes
  funciones \textcolor{codeKeyword2}{int}(), \textcolor{codeKeyword2}{float}()
  y \textcolor{codeKeyword2}{complex}():

  \begin{lstlisting}[language=Python]
  x = 1    # int
  y = 2.8  # float
  z = -1j  # complex

  # convertir de int a float:
  a = float(x)
  # convertir de float a int:
  b = int(y)
  # convertir de int a complex:
  c = complex(x)
  \end{lstlisting}

  \begin{alertblock}{Nota}
    No se puede convertir un \textbf{número complejo} en otro tipo de dato numérico.
  \end{alertblock}
\end{frame}

\begin{frame}[fragile]
  \frametitle{Números aleatorios}

  Python no tiene una función de \textbf{random}() para
  hacer números aleatorios, pero Python tiene integrado un modulo llamado
  \textbf{random} que puede ser usado para hacer números
  aleatorios:

  \vspace{\baselineskip}
  Importamos el modulo de \textbf{random}, y desplegamos un número aleatorio entre 1
  y 9:

  \begin{lstlisting}[language=Python]
  import random

  print( random.randrange(1,10) )
  \end{lstlisting}
\end{frame}

%\begin{frame}[c]{Créditos}
%  El material está basado en
%  \href{https://www.w3schools.com/python/default.asp}{https://www.w3schools.com/python/default.asp}
%\end{frame}


{ % Portada de la universidad
  \setbeamertemplate{footline}{}
  \usebackgroundtemplate{\includegraphics[width=\paperwidth]{fondo-portada.png}}
  \begin{frame}
  \end{frame}
}
\end{document}
