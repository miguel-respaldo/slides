% ex: ts=2 sw=2 sts=2 et filetype=tex
% SPDX-License-Identifier: CC-BY-SA-4.0

\section{PseInt a Python}

\begin{frame}[fragile]
  \frametitle{Nombre del algoritmo}

  Para hacer un algoritmo en pseudocódigo y su codificación en Python,
  el cuerpo del programa tiene la estructura:
  \vspace{\baselineskip}
  \begin{columns}
    \column{0.05\textwidth}
    \column{0.5\textwidth}
      \textbf{PseInt}
      \begin{lstlisting}[style=pseudocodigo]
Algoritmo nombre_del_algoritmo
  ...
  <instrucciones>
  ...
FinAlgoritmo
      \end{lstlisting}
    \column{0.05\textwidth}
    \pausa
    \column{0.4\textwidth}
      \textbf{Python}
      \begin{lstlisting}[language=Python]
...
<instrucciones>
...

      \end{lstlisting}
  \end{columns}
\end{frame}

\begin{frame}[fragile]
  \frametitle{Comentarios}

  Los comentarios se usan con:
  \vspace{\baselineskip}
  \begin{columns}
    \column{0.05\textwidth}
    \column{0.5\textwidth}
      \textbf{PseInt}
      \begin{lstlisting}[style=pseudocodigo]
// este es un comentario
      \end{lstlisting}
    \column{0.05\textwidth}
    \pausa
    \column{0.4\textwidth}
      \textbf{Python}
      \begin{lstlisting}[language=Python]
# este es un comentario
      \end{lstlisting}
  \end{columns}
\end{frame}

\begin{frame}[fragile]
  \frametitle{Salida a pantalla}

  Para mostrar algo en pantalla se usa:
  \vspace{\baselineskip}
  \begin{columns}
    \column{0.05\textwidth}
    \column{0.4\textwidth}
      \textbf{PseInt}
      \begin{lstlisting}[style=pseudocodigo]
Escribir "¡Hola Mundo!"
      \end{lstlisting}
    \column{0.05\textwidth}
    \pausa
    \column{0.5\textwidth}
      \textbf{Python}
      \begin{lstlisting}[language=Python]
print("¡Hola Mundo!")
      \end{lstlisting}
  \end{columns}
\end{frame}

\begin{frame}[fragile]
  \frametitle{Lectura de teclado}

  Para introducir datos por el teclado se usa:
  \vspace{\baselineskip}
  \begin{columns}
    \column{0.05\textwidth}
    \column{0.4\textwidth}
      \textbf{PseInt}
      \begin{lstlisting}[style=pseudocodigo]
Leer num1
      \end{lstlisting}
    \column{0.05\textwidth}
    \pausa
    \column{0.5\textwidth}
      \textbf{Python}
      \begin{lstlisting}[language=Python]
num1 = eval(input())
      \end{lstlisting}
  \end{columns}
\end{frame}

\begin{frame}[fragile]
  \frametitle{Para escribir y leer}

  Para mostrar un mensaje y leer un dato:
  \vspace{\baselineskip}
  \begin{columns}
    \column{0.02\textwidth}
    \column{0.44\textwidth}
      \textbf{PseInt}
      \begin{lstlisting}[style=pseudocodigo]
Escribir "Escribe un número" Sin Saltar
Leer num1
      \end{lstlisting}
    \column{0.02\textwidth}
    \pausa
    \column{0.56\textwidth}
      \textbf{Python}
      \begin{lstlisting}[language=Python]
num1 = eval(input("Escribe un número"))
      \end{lstlisting}
  \end{columns}
\end{frame}

\begin{frame}[fragile]
  \frametitle{Asignación}

  Para asignar un valor a una variable se usa:
  \vspace{\baselineskip}
  \begin{columns}
    \column{0.05\textwidth}
    \column{0.45\textwidth}
      \textbf{PseInt}
      \begin{lstlisting}[style=pseudocodigo]
num1 <- 5
num2 <- 3.4
suma <- num1 + num2
      \end{lstlisting}
    \column{0.05\textwidth}
    \pausa
    \column{0.55\textwidth}
      \textbf{Python}
      \begin{lstlisting}[language=Python]
num1 = 5
num2 = 3.4
suma = num1 + num2
      \end{lstlisting}
  \end{columns}
\end{frame}

\begin{frame}[fragile]
  \frametitle{Operaciones aritméticas}

  Ejemplo de operaciones aritméticas básicas:
  \vspace{\baselineskip}
  \begin{columns}
    \column{0.05\textwidth}
    \column{0.45\textwidth}
      \textbf{PseInt}
      \begin{lstlisting}[style=pseudocodigo]
suma <- num1 + num2
resta <- num1 - num2
multiplicacion <- num1 * num2
division <- num1 / num2
modulo <- num1 % num2
potencia <- num1 ^ 2
      \end{lstlisting}
    \column{0.05\textwidth}
    \pausa
    \column{0.55\textwidth}
      \textbf{Python}
      \begin{lstlisting}[language=Python]
suma = num1 + num2
resta = num1 - num2
multiplicacion = num1 * num2
division = num1 / num2
modulo = num1 % num2
potencia = num1 ** 2
      \end{lstlisting}
  \end{columns}
\end{frame}

\begin{frame}[fragile]
  \frametitle{Raíz cuadrada}

  La raíz cuadrada en Python necesita una biblioteca especial:

  \vspace{\baselineskip}
  \begin{columns}
    \column{0.05\textwidth}
    \column{0.45\textwidth}
      \textbf{PseInt}
      \begin{lstlisting}[style=pseudocodigo]
...
resultado <- raiz(num1)
      \end{lstlisting}
    \column{0.05\textwidth}
    \pausa
    \column{0.55\textwidth}
      \textbf{Python}
      \begin{lstlisting}[language=Python]
# se pone al principio del archivo
import math
...
resultado = math.sqrt(num1)
      \end{lstlisting}
  \end{columns}
\end{frame}

\begin{frame}[fragile]
  \frametitle{Formula general de las cuadráticas}

  La formula general para la solución de las ecuaciones cuadradas es:
  $x_{1,2} = \frac{-b \pm \sqrt{b^2 - 4ac}}{2a}$

  \vspace{\baselineskip}
  \textbf{PseInt}
  \begin{lstlisting}[style=pseudocodigo]
x1 <- (-b + raiz( b^2 - 4*a*c)) / (2*a)
x2 <- (-b - raiz( b^2 - 4*a*c)) / (2*a)
  \end{lstlisting}
  \pausa

  \textbf{Python}
  \begin{lstlisting}[language=Python]
import math
...
x1 = (-b + math.sqrt( b**2 - 4*a*c)) / (2*a)
x2 = (-b - math.sqrt( b**2 - 4*a*c)) / (2*a)
  \end{lstlisting}
\end{frame}
