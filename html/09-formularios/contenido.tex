% ex: ts=2 sw=2 sts=2 et filetype=tex
% SPDX-License-Identifier: CC-BY-SA-4.0

\section{Formularios HTML}

\begin{frame}[c]{Formularios}
  Se utiliza un formulario HTML para recopilar la entrada del usuario.
  La entrada del usuario se envía con mayor frecuencia a un servidor
  para su procesamiento.
\end{frame}

\begin{frame}[fragile]
  \frametitle{El elemento <form>}

  El elemento HTML \eti{<form>} se utiliza para crear un
  formulario HTML para la entrada del usuario:

  \vspace{\baselineskip}
  \begin{lstlisting}
<form>
.
elementos del formulario
.
</form>
  \end{lstlisting}

  \vspace{\baselineskip}
  El elemento \eti{<form>} es un contenedor para diferentes tipos de
  elementos de entrada, como: campos de texto, casillas de verificación,
  botones de opción, botones de envío, etc.
\end{frame}

\begin{frame}[c]{El elemento <input>}

  El elemento HTML \eti{<input>} es el elemento de formulario más utilizado.

  \vspace{\baselineskip}
  Un elemento \eti{<input>} se puede mostrar de muchas maneras,
  según el atributo \atri{type}.

  \vspace{\baselineskip}
  Aquí hay unos ejemplos:

  \begin{table}[]
  \begin{tabular}{ll}
    \textbf{Tipo} &  \textbf{Descripción} \\
    \rowcolor{light-gray}
    \eti{<input} \atri{type}="\propi{text}"\eti{>} &  Muestra un campo de entrada de texto de una sola línea \\
    \eti{<input} \atri{type}="\propi{radio}"\eti{>} & Muestra un botón de radio (para seleccionar una de muchas \\ 
                                                    & opciones)  \\
    \rowcolor{light-gray}
    \eti{<input} \atri{type}="\propi{checkbox}"\eti{>} & Muestra una casilla de verificación (para seleccionar cero \\
    \rowcolor{light-gray}
                                                     & o más de muchas opciones)  \\
    \eti{<input} \atri{type}="\propi{submit}"\eti{>} &  Muestra un botón de envío (para enviar el formulario) \\
    \rowcolor{light-gray}
    \eti{<input} \atri{type}="\propi{button}"\eti{>} &  Muestra un botón en el que se puede hacer clic \\
  \end{tabular}
  \end{table}
\end{frame}

\begin{frame}[fragile]
  \frametitle{Campos de texto}

  Define un campo de entrada de una sola línea para la entrada 
  de texto \eti{<input} \atri{type}="\propi{text}"\eti{>}

  \vspace{\baselineskip}
  Ejemplo: Un formulario con campos de entrada de texto:
  \begin{lstlisting}
<form>
  <label for="nombre">Nombre:</label><br>
  <input type="text" id="nombre" name="nombre"><br>
  <label for="apellidos">Apellidos:</label><br>
  <input type="text" id="apellidos" name="apellidos">
</form>
  \end{lstlisting}

  \begin{block}{Nota:}
    La etiqueta del formulario \eti{<form>} en sí no es visible.
    También tenga en cuenta que el ancho predeterminado de un
    campo de entrada es de 20 caracteres.
  \end{block}
\end{frame}

\begin{frame}[c]{El elemento <label>}

  Observe el uso del elemento \eti{<label>} en el ejemplo anterior.

  \begin{itemize}
    \item La etiqueta \eti{<label>} define una etiqueta para muchos
      elementos de formulario.
    \item El elemento \eti{<label>} es útil para los usuarios de lectores
      de pantalla (como Alexa), porque el lector de pantalla leerá en voz
      alta la etiqueta cuando el usuario se centre en el elemento de
      entrada.
    \item El elemento \eti{<label>} también ayuda a los usuarios que tienen
      dificultades para hacer clic en regiones muy pequeñas (como botones
      de radio o casillas de verificación), porque cuando el usuario hace
      clic en el texto dentro del elemento \eti{<label>}, alterna el botón
      de radio/casilla de verificación.
    \item El atributo \atri{for} de la etiqueta \eti{<label>} debe ser igual
      al atributo \atri{id} del elemento \eti{<input>} para unirlos.
  \end{itemize}
\end{frame}

\begin{frame}[fragile]
  \frametitle{Botones de radio}

  \eti{<input} \atri{type}="\propi{radio}"\eti{>}
  define un botón de opción.

  \vspace{\baselineskip}
  Los botones de opción le permiten al usuario seleccionar UNA de un
  número limitado de opciones.

  \vspace{\baselineskip}
  Ejemplo: Un formulario con botones de radio:
  \begin{lstlisting}
<p>Escoje tu lenguaje Web favorito:</p>
<form>
  <input type="radio" id="html" name="leng_fav" value="HTML">
  <label for="html">HTML</label><br>
  <input type="radio" id="css" name="leng_fav" value="CSS">
  <label for="css">CSS</label><br>
  <input type="radio" id="javascript" name="leng_fav" value="JavaScript">
  <label for="javascript">JavaScript</label>
</form>
  \end{lstlisting}
\end{frame}

\begin{frame}[fragile]
  \frametitle{}

  \vspace{\baselineskip}
  \begin{lstlisting}
  \end{lstlisting}
\end{frame}

\begin{frame}[fragile]
  \frametitle{}

  \vspace{\baselineskip}
  \begin{lstlisting}
  \end{lstlisting}
\end{frame}

\begin{frame}[fragile]
  \frametitle{}

  \vspace{\baselineskip}
  \begin{lstlisting}
  \end{lstlisting}
\end{frame}

\begin{frame}[fragile]
  \frametitle{}

  \vspace{\baselineskip}
  \begin{lstlisting}
  \end{lstlisting}
\end{frame}
