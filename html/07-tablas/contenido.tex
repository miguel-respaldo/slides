% ex: ts=2 sw=2 sts=2 et filetype=tex
% SPDX-License-Identifier: CC-BY-SA-4.0

\section{Definir una tabla}

\begin{frame}[fragile]
  \frametitle{Tablas HTML}

  Las tablas HTML permiten a los desarrolladores web organizar
  los datos en filas y columnas.

  Una tabla en HTML consta de celdas de tabla dentro de filas y columnas

  \begin{lstlisting}
<table>
  <tr>
    <th>Compañía</th>
    <th>Contacto</th>
    <th>País</th>
  </tr>
  <tr>
    <td>Robots industriales</td>
    <td>María Pérez</td>
    <td>Alemania</td>
  </tr>
  <tr>
    <td>Centro comercial Moctezuma</td>
    <td>Francisco Hernández</td>
    <td>México</td>
  </tr>
</table> 
  \end{lstlisting}
\end{frame}

\begin{frame}[fragile]
  \frametitle{Celdas de tabla}

  Cada celda de la tabla está definida por una etiqueta \eti{<td>}
  y \eti{</td>}.

  \begin{exampleblock}{td}
    Significa "\emph{table data}" en inglés, que es datos de la tabla
    en español.
  \end{exampleblock}

  Todo lo que está entre \eti{<td>} y \eti{</td>} es el contenido
  de la celda de la tabla.

  \begin{lstlisting}
<table>
  <tr>
    <td>Miguel</td>
    <td>Hidalgo</td>
    <td>y Costilla</td>
  </tr>
</table> 
  \end{lstlisting}

  \begin{block}{Nota:}
    Los elementos de datos de la tabla son los contenedores de datos
    de la tabla. Pueden contener todo tipo de elementos HTML; texto,
    imágenes, listas, otras tablas, etc.
  \end{block}
\end{frame}

\begin{frame}[fragile]
  \frametitle{Filas de tabla}

  Cada fila de la tabla comienza con una etiqueta \eti{<tr>} y
  termina con \eti{</tr>}.

  \begin{exampleblock}{tr}
    Significa "\emph{table row}" en inglés, que es fila de la tabla
    en español.
  \end{exampleblock}

  \begin{lstlisting}
<table>
  <tr>
    <td>Miguel</td>
    <td>Hidalgo</td>
    <td>y Costilla</td>
  </tr>
  <tr>
    <td>José</td>
    <td>María</td>
    <td>Morelos</td>
  </tr>
</table> 
  \end{lstlisting}
\end{frame}

\begin{frame}[c]{Filas de tabla}

  Puede tener tantas filas como desee en una tabla, solo asegúrese
  de que la cantidad de celdas sea la misma en cada fila.

  \vspace{\baselineskip}
  \begin{block}{Nota:}
    hay ocasiones en las que una fila puede tener menos o más celdas
    que otra. Aprenderá sobre eso en una sección posterior.
  \end{block}
\end{frame}

\begin{frame}[fragile]
  \frametitle{Encabezados de tabla}

  A veces desea que sus celdas sean encabezados, en esos casos use la
  etiqueta \eti{<th>} en lugar de la etiqueta \eti{<td>}. Hagamos que
  la primera fila sean encabezados de tabla:

  \begin{lstlisting}
<table>
  <tr>
    <th>Nombre</th>
    <th>Apellido paterno</th>
    <th>Apellido materno</th>
  </tr>
  <tr>
    <td>Miguel</td>
    <td>Hidalgo</td>
    <td>y Costilla</td>
  </tr>
  <tr>
    <td>José María</td>
    <td>Morelos</td>
    <td>y Pavón</td>
  </tr>
</table> 
  \end{lstlisting}
\end{frame}

\begin{frame}[c]{Encabezados de tabla}
  De forma predeterminada, el texto de los elementos \eti{<th>}
  está en negrita y centrado, pero puedes cambiar eso con CSS.
\end{frame}

\begin{frame}[c]{Resumen de etiquetas de tabla HTML}
  \begin{table}[]
  \begin{tabular}{cll}
    \textbf{Etiqueta} &  \textbf{Descripción} \\
    \rowcolor{light-gray}
    \eti{<table>} & Define una tabla \\
    \eti{<th>} & Define una celda de encabezado en una tabla \\
    \rowcolor{light-gray}
    \eti{<tr>} & Define una fila en una tabla \\
    \eti{<td>} & Define una celda en una tabla \\
    \rowcolor{light-gray}
    \eti{<caption>} & Define un título de tabla \\
    \eti{<colgroup>} & Especifica un grupo de una o más columnas \\
                     & en una tabla para formatear \\
    \rowcolor{light-gray}
    \eti{<col>} & Especifica las propiedades de columna para cada \\
                & columna dentro de un elemento <colgroup> \\
    \eti{<thead>} & Agrupa el contenido del encabezado en una tabla \\
    \rowcolor{light-gray}
    \eti{<tbody>} & Agrupa el contenido del cuerpo en una tabla \\
    \eti{<tfoot>} & Agrupa el contenido del pie de página en una tabla \\
  \end{tabular}
  \end{table}
\end{frame}

\section{Bordes de la tabla}

\begin{frame}[c]{Cómo agregar un borde}
  Las tablas HTML pueden tener bordes de diferentes estilos y formas.

  \vspace{\baselineskip}
  Cuando agrega un borde a una tabla, también agrega bordes alrededor
  de cada celda de la tabla.

  \vspace{\baselineskip}
  Para agregar un borde, use la propiedad \atri{border} CSS en los
  elementos \eti{table}, \eti{th} y \eti{td}.
\end{frame}

\begin{frame}[fragile]
  \frametitle{Ejemplo 1 de bordes}
  \lstinputlisting{0701-bordes01.html}
  Continua en la siguiente página
\end{frame}

\begin{frame}[fragile]
  \frametitle{Ejemplo 1 de bordes (cont.)}
  \lstinputlisting[firstnumber=14]{0701-bordes02.html}
\end{frame}

\begin{frame}[fragile]
  \frametitle{Bordes de tabla contraídos}

  Para evitar tener bordes dobles como en el ejemplo anterior,
  establezca la propiedad \atri{border-collapse} CSS en \propi{collapse}.

  \vspace{\baselineskip}
  Esto hará que los bordes colapsen en un solo borde:

  \vspace{\baselineskip}
  \begin{lstlisting}
table, th, td {
  border: 1px solid black;
  border-collapse: collapse;
}
  \end{lstlisting}
\end{frame}

\begin{frame}[fragile]
  \frametitle{Bordes de la tabla de estilo}

  Si establece un color de fondo para cada celda y le da al
  borde un color blanco (el mismo que el fondo del documento)
  obtendrá la impresión de un borde invisible:

  \vspace{\baselineskip}
  \begin{lstlisting}
table, th, td {
  border: 1px solid white;
  border-collapse: collapse;
}
th, td {
  background-color: #96D4D4;
}
  \end{lstlisting}
\end{frame}

\begin{frame}[fragile]
  \frametitle{Bordes de mesa redonda}

  Con la propiedad \atri{border-radius}, los bordes obtienen
  esquinas redondeadas:

  \vspace{\baselineskip}
  \begin{lstlisting}
table, th, td {
  border: 1px solid black;
  border-radius: 10px;
}
  \end{lstlisting}
\end{frame}

\begin{frame}[fragile]
  \frametitle{Bordes de mesa redonda}

  Omita el borde alrededor de la tabla omitiendo \atri{table}
  del selector css:

  \vspace{\baselineskip}
  \begin{lstlisting}
th, td {
  border: 1px solid black;
  border-radius: 10px;
}
  \end{lstlisting}
\end{frame}

\begin{frame}[fragile]
  \frametitle{Bordes de tabla punteados}
  Con la propiedad \atri{border-style}, puede establecer la
  apariencia del borde.

  \vspace{\baselineskip}
  Se permiten los siguientes valores:

  \begin{columns}
    \column{0.5\textwidth}
  \begin{itemize}
    \item dotted
    \item dashed
    \item solid
    \item double
    \item groove
  \end{itemize}
    \column{0.6\textwidth}
  \begin{itemize}
    \item ridge
    \item inset
    \item outset
    \item none
    \item hidden
  \end{itemize}
  \end{columns}

  \vspace{\baselineskip}
  Ejemplo:
  \begin{lstlisting}
th, td {
  border-style: dotted;
}
  \end{lstlisting}
\end{frame}

\begin{frame}[fragile]
  \frametitle{Color del borde}

  Con la propiedad \atri{border-color}, puede establecer el color del borde.

  \vspace{\baselineskip}
  \begin{lstlisting}
th, td {
  border-color: red;
}
  \end{lstlisting}
\end{frame}

\section{Tamaños de tabla}

\begin{frame}[c]{Tamaños de tabla}
  Las tablas HTML pueden tener diferentes tamaños para cada columna,
  fila o la tabla completa.

  \vspace{\baselineskip}
  Utilice el atributo \atri{style} con las propiedades \propi{width} o
  \propi{height} para especificar el tamaño de una tabla, fila o columna.
\end{frame}

\begin{frame}[fragile]
  \frametitle{Ancho de la tabla HTML}

  Para establecer el ancho de una tabla, agregue el atributo \atri{style}
  al elemento \eti{<table>}:

  Establezca el ancho de la tabla al 100\%:
  \begin{lstlisting}
<table style="width:100%">
  <tr>
    <th>Nombre</th>
    <th>Apellido paterno</th>
    <th>Apellido materno</th>
  </tr>
  <tr>
    <td>Miguel</td>
    <td>Hidalgo</td>
    <td>y Costilla</td>
  </tr>
</table> 
  \end{lstlisting}

  \begin{exampleblock}{Nota:}
    Usar un porcentaje como unidad de tamaño para un ancho significa
    qué tan ancho será este elemento en comparación con su elemento
    principal, que en este caso es el elemento \eti{<body>}.
  \end{exampleblock}
\end{frame}

\begin{frame}[fragile]
  \frametitle{Ancho de columna de la tabla HTML}

  Para establecer el tamaño de una columna específica,
  agregue el atributo \atri{style} en un elemento \eti{<th>}
  o \eti{<td>}:

  \vspace{\baselineskip}
  Establezca el ancho de la primera columna al 70%:
  \begin{lstlisting}
<table style="width:100%">
  <tr>
    <th style="width:70%">Nombre</th>
    <th>Apellido paterno</th>
    <th>Apellido materno</th>
  </tr>
  <tr>
    <td>Miguel</td>
    <td>Hidalgo</td>
    <td>y Costilla</td>
  </tr>
</table> 
  \end{lstlisting}
\end{frame}

\begin{frame}[fragile]
  \frametitle{Altura de fila de la tabla HTML}

  Para establecer la altura de una fila específica, agregue el
  atributo \atri{style} en un elemento de fila de la tabla:

  \vspace{\baselineskip}
  Establezca la altura de la segunda fila en 200 píxeles:
  \begin{lstlisting}
<table style="width:100%">
  <tr>
    <th>Nombre</th>
    <th>Apellido paterno</th>
    <th>Apellido materno</th>
  </tr>
  <tr style="height:200px">
    <td>Miguel</td>
    <td>Hidalgo</td>
    <td>y Costilla</td>
  </tr>
</table> 
  \end{lstlisting}
\end{frame}

\section{Encabezados de una tabla}

\begin{frame}[fragile]
  \frametitle{Encabezados de una tabla}

  Las tablas HTML pueden tener encabezados para cada columna o
  fila, o para muchas columnas/filas.

  Los encabezados de las tablas se definen con elementos \eti{th},
  cada elemento \eti{th} representa una celda de la tabla.
  \vspace{\baselineskip}
  \begin{lstlisting}
<table>
  <tr>
    <th>Nombre</th>
    <th>Apellido paterno</th>
    <th>Apellido materno</th>
  </tr>
  <tr>
    <td>Miguel</td>
    <td>Hidalgo</td>
    <td>y Costilla</td>
  </tr>
</table> 
  \end{lstlisting}
\end{frame}

\begin{frame}[fragile]
  \frametitle{Encabezados de tabla verticales}

  Para usar la primera columna como encabezados de tabla,
  defina la primera celda de cada fila como un elemento \eti{th}:

  \begin{lstlisting}
<table>
  <tr>
    <th>Nombre</th>
    <td>Miguel</td>
    <td>José María</td>
  </tr>
  <tr>
    <th>Apellido paterno</th>
    <td>Hidalgo</td>
    <td>Morelos</td>
  </tr>
  <tr>
    <th>Apellido materno</th>
    <td>y Costilla</td>
    <td>y Pavón</td>
  </tr>
</table> 
  \end{lstlisting}
\end{frame}

\begin{frame}[fragile]
  \frametitle{Alinear encabezados de tabla}

  De forma predeterminada, los encabezados de las tablas están
  en negrita y centrados:

  Para alinear a la izquierda los encabezados de la tabla,
  use la propiedad \atri{text-align} CSS:

  \vspace{\baselineskip}
  \begin{lstlisting}
<style>
th {
  text-align: left;
}
</style>
  \end{lstlisting}
\end{frame}

\begin{frame}[fragile]
  \frametitle{Encabezado para varias columnas}

  Puede tener un encabezado que abarque dos o más columnas.

  Para hacer esto, use el atributo \atri{colspan} en el elemento
  \eti{<th>}:

  \vspace{\baselineskip}
  \begin{lstlisting}
<table>
  <tr>
    <th>Nombre</th>
    <th colspan="2">Apellidos</th>
  </tr>
  <tr>
    <td>Miguel</td>
    <td>Hidalgo</td>
    <td>y Costilla</td>
  </tr>
</table> 
  \end{lstlisting}
\end{frame}

\begin{frame}[fragile]
  \frametitle{Título de la tabla}

  Puede agregar un título que sirva como encabezado para toda la tabla.
  Para agregar un título a una tabla, use la etiqueta \eti{<caption>}:

  \vspace{\baselineskip}
  \begin{lstlisting}
<table style="width:100%">
  <caption>Ahorros del mes</caption>
  <tr>
    <th>Mes</th>
    <th>Ahorro</th>
  </tr>
  <tr>
    <td>Enero</td>
    <td>$100</td>
  </tr>
  <tr>
    <td>Febrero</td>
    <td>$100</td>
  </tr>
</table> 
  \end{lstlisting}
\end{frame}

\section{Relleno y espaciado}

\begin{frame}[c]{}
\end{frame}

\begin{frame}[fragile]
  \frametitle{}

  \vspace{\baselineskip}
  \begin{lstlisting}

  \end{lstlisting}
\end{frame}

\section{Celdas de varias filas o columnas}

\begin{frame}[c]{}
\end{frame}

\begin{frame}[fragile]
  \frametitle{}

  \vspace{\baselineskip}
  \begin{lstlisting}

  \end{lstlisting}
\end{frame}

\section{Estilos de las tablas}

\begin{frame}[c]{}
\end{frame}

\begin{frame}[fragile]
  \frametitle{}

  \vspace{\baselineskip}
  \begin{lstlisting}

  \end{lstlisting}
\end{frame}

\section{Agrupar columnas}

\begin{frame}[c]{}
\end{frame}

\begin{frame}[fragile]
  \frametitle{}

  \vspace{\baselineskip}
  \begin{lstlisting}

  \end{lstlisting}
\end{frame}
