% ex: ts=2 sw=2 sts=2 et filetype=tex
% SPDX-License-Identifier: CC-BY-SA-4.0

\section{Formato de texto}

\begin{frame}[c]{Elementos de formato HTML}

  HTML contiene varios elementos para definir texto con
  un significado especial.

  \vspace{\baselineskip}
  Los elementos de formato se diseñaron para mostrar tipos
  especiales de texto:

  \vspace{\baselineskip}
  \begin{enumerate}
    \item \eti{<b>} Texto en negrita
    \item \eti{<strong>} Texto importante
    \item \eti{<i>} Texto en cursiva/itálica
    \item \eti{<em>} Texto enfatizado
    \item \eti{<mark>} Texto marcado
    \item \eti{<small>} Texto más pequeño
    \item \eti{<del>} Texto eliminado
    \item \eti{<ins>} Texto insertado
    \item \eti{<sub>} Texto de subíndice
    \item \eti{<sup>} Texto en superíndice
  \end{enumerate}
\end{frame}

\begin{frame}[fragile]
  \frametitle{Elementos HTML <b> y <strong>}

  El elemento HTML \eti{<b>} define texto en negrita,
  sin ninguna importancia adicional.

  \begin{lstlisting}
<b> Este texto esta en negritas </b>
  \end{lstlisting}

  \vspace{\baselineskip}
  El elemento HTML \eti{<strong>} define texto con gran importancia.
  El contenido interior normalmente se muestra en negrita.

  \begin{lstlisting}
<strong> Este texto es importante </strong>
  \end{lstlisting}

  \vspace{\baselineskip}
  La diferencia esta en el sentido de como se lee el texto por un programa
  de texto a voz, ejemplo: "Alexa". Alexa leerá el texto con un tono más
  fuerte cuando lea una etiqueta \eti{<strong>}. \eti{<b>} es mas un aspecto
  visual.
\end{frame}

\begin{frame}[fragile]
  \frametitle{Elementos HTML <i> y <em>}

  El elemento HTML \eti{<i>} define una parte del texto en una voz
  o estado de ánimo alternativo. El contenido interior normalmente
  se muestra en cursiva/itálica.

  \vspace{\baselineskip}
  \textbf{Sugerencia}: la \eti{<i>} etiqueta se usa a menudo para
  indicar un término técnico, una frase de otro idioma, un
  pensamiento, el nombre de un barco, etc.

  \vspace{\baselineskip}
  \begin{lstlisting}
<i>Este es un texto en cursiva/itálica</i>
  \end{lstlisting}
\end{frame}

\begin{frame}[fragile]
  \frametitle{Elementos HTML <i> y <em>}

  El elemento HTML \eti{<em>} define texto enfatizado.
  El contenido interior normalmente se muestra en cursiva/itálica.

  \vspace{\baselineskip}
  \textbf{Sugerencia}: un lector de pantalla como "Alexa"
  pronunciará las palabras \eti{<em>} con énfasis, usando acento verbal.

  \vspace{\baselineskip}
  \begin{lstlisting}
<em>Este texto esta enfatizado</em>
  \end{lstlisting}
\end{frame}

\begin{frame}[fragile]
  \frametitle{Elemento HTML <small>}

  El elemento HTML \eti{<small>} define texto más pequeño:

  \vspace{\baselineskip}
  \begin{lstlisting}
<small>Este es un texto mas pequeño</small>
  \end{lstlisting}
\end{frame}

\begin{frame}[fragile]
  \frametitle{Elemento HTML <mark>}

  El elemento HTML \eti{<mark>} define el texto que debe
  marcarse o resaltarse:

  \vspace{\baselineskip}
  \begin{lstlisting}
<p>No se te olvide comprar <mark>leche</mark> el día de hoy.</p>
  \end{lstlisting}
\end{frame}

\begin{frame}[fragile]
  \frametitle{Elemento HTML <del>}

  El elemento HTML \eti{<del>} define el texto que se ha eliminado
  de un documento. Los navegadores generalmente marcarán una línea
  a través del texto eliminado:

  \vspace{\baselineskip}
  \begin{lstlisting}
<p>Mi color favorito es el <del>azul</del> rojo.</p>
  \end{lstlisting}
\end{frame}

\begin{frame}[fragile]
  \frametitle{Elemento HTML <ins>}

  El elemento HTML \eti{<ins>} define un texto que se ha insertado
  en un documento. Los navegadores normalmente subrayarán el texto
  insertado:

  \vspace{\baselineskip}
  \begin{lstlisting}
<p>Mi color favorito es el <del>azul</del> <ins>rojo</ins>.</p>
  \end{lstlisting}
\end{frame}

\begin{frame}[fragile]
  \frametitle{Elemento HTML <sub>}

  El elemento HTML \eti{<sub>} define el texto del subíndice.
  El texto del subíndice aparece medio carácter por debajo de la
  línea normal y, a veces, se representa en una fuente más pequeña.
  El texto de subíndice se puede utilizar para fórmulas químicas,
  como $H_2O$:

  \vspace{\baselineskip}
  \begin{lstlisting}
<p>Este es un <sub>texto como subíndice</sub>.</p>
  \end{lstlisting}
\end{frame}

\begin{frame}[fragile]
  \frametitle{Elemento HTML <sup>}

  El elemento HTML \eti{<sup>} define texto en superíndice.
  El texto en superíndice aparece medio carácter por encima de
  la línea normal y, a veces, se representa en una fuente más
  pequeña. El texto en superíndice se puede usar para notas al
  pie, como $WWW^{[1]}$:

  \vspace{\baselineskip}
  \begin{lstlisting}
<p>Este es un <sub>texto como superíndice</sub>.</p>
  \end{lstlisting}
\end{frame}

\begin{frame}[c]{Resumen de unos formatos de texto en HTML}
  \begin{table}[]
  \begin{tabular}{cll}
    \textbf{Etiqueta} &  \textbf{Descripción} \\
    \rowcolor{light-gray}
    \eti{<b>} & Texto en negrita \\
    \eti{<strong>} & Texto importante \\
    \rowcolor{light-gray}
    \eti{<i>} & Texto en cursiva/itálica \\
    \eti{<em>} & Texto enfatizado \\
    \rowcolor{light-gray}
    \eti{<mark>} & Texto marcado \\
    \eti{<small>} & Texto más pequeño \\
    \rowcolor{light-gray}
    \eti{<del>} & Texto eliminado \\
    \eti{<ins>} & Texto insertado \\
    \rowcolor{light-gray}
    \eti{<sub>} & Texto de subíndice \\
    \eti{<sup>} & Texto en superíndice \\
  \end{tabular}
  \end{table}
\end{frame}

\section{Citas de texto}


\begin{frame}[c]{}

\end{frame}

\begin{frame}[fragile]
  \frametitle{}

  \vspace{\baselineskip}
  \begin{lstlisting}
  \end{lstlisting}
\end{frame}

