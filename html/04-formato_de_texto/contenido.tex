% ex: ts=2 sw=2 sts=2 et filetype=tex
% SPDX-License-Identifier: CC-BY-SA-4.0

\section{Formato de texto}

\begin{frame}[c]{Elementos de formato HTML}

  HTML contiene varios elementos para definir texto con
  un significado especial.

  Los elementos de formato se diseñaron para mostrar tipos
  especiales de texto:

  \vspace{\baselineskip}
  \begin{enumerate}
    \item \eti{<b>} Texto en negrita
    \item \eti{<strong>} Texto importante
    \item \eti{<i>} Texto en cursiva/itálica
    \item \eti{<em>} Texto enfatizado
    \item \eti{<mark>} Texto marcado
    \item \eti{<small>} Texto más pequeño
    \item \eti{<del>} Texto eliminado
    \item \eti{<ins>} Texto insertado
    \item \eti{<sub>} Texto de subíndice
    \item \eti{<sup>} Texto en superíndice
  \end{enumerate}
\end{frame}

\begin{frame}[fragile]
  \frametitle{Elementos HTML <b> y <strong>}

  El elemento HTML \eti{<b>} define texto en negrita,
  sin ninguna importancia adicional.

  \begin{lstlisting}
<b> Este texto esta en negritas </b>
  \end{lstlisting}

  \vspace{\baselineskip}
  El elemento HTML \eti{<strong>} define texto con gran importancia.
  El contenido interior normalmente se muestra en negrita.

  \begin{lstlisting}
<strong> Este texto es importante </strong>
  \end{lstlisting}

  \begin{block}{La diferencia}
    esta en el sentido de como se lee el texto por un programa
    de texto a voz, ejemplo: "Alexa". Alexa leerá el texto con un tono más
    fuerte cuando lea una etiqueta \eti{<strong>}. \eti{<b>} es mas un aspecto
    visual.
  \end{block}
\end{frame}

\begin{frame}[fragile]
  \frametitle{Elementos HTML <i> y <em>}

  El elemento HTML \eti{<i>} define una parte del texto en una voz
  o estado de ánimo alternativo. El contenido interior normalmente
  se muestra en cursiva/itálica.

  \vspace{\baselineskip}
  \textbf{Sugerencia}: la \eti{<i>} etiqueta se usa a menudo para
  indicar un término técnico, una frase de otro idioma, un
  pensamiento, el nombre de un barco, etc.

  \vspace{\baselineskip}
  \begin{lstlisting}
<i> Este es un texto en cursiva/itálica </i>
  \end{lstlisting}
\end{frame}

\begin{frame}[fragile]
  \frametitle{Elementos HTML <i> y <em>}

  El elemento HTML \eti{<em>} define texto enfatizado.
  El contenido interior normalmente se muestra en cursiva/itálica.

  \vspace{\baselineskip}
  \textbf{Sugerencia}: un lector de pantalla como "Alexa"
  pronunciará las palabras \eti{<em>} con énfasis, usando acento verbal.

  \vspace{\baselineskip}
  \begin{lstlisting}
<em> Este texto esta enfatizado</em>
  \end{lstlisting}
\end{frame}

\begin{frame}[fragile]
  \frametitle{Elemento HTML <small>}

  El elemento HTML \eti{<small>} define texto más pequeño:

  \vspace{\baselineskip}
  \begin{lstlisting}
<small> Este es un texto mas pequeño </small>
  \end{lstlisting}
\end{frame}

\begin{frame}[fragile]
  \frametitle{Elemento HTML <mark>}

  El elemento HTML \eti{<mark>} define el texto que debe
  marcarse o resaltarse:

  \vspace{\baselineskip}
  \begin{lstlisting}
<p> No se te olvide comprar <mark>leche</mark> el día de hoy.</p>
  \end{lstlisting}
\end{frame}

\begin{frame}[fragile]
  \frametitle{Elemento HTML <del>}

  El elemento HTML \eti{<del>} define el texto que se ha eliminado
  de un documento. Los navegadores generalmente marcarán una línea
  a través del texto eliminado:

  \vspace{\baselineskip}
  \begin{lstlisting}
<p> Mi color favorito es el <del>azul</del> rojo.</p>
  \end{lstlisting}
\end{frame}

\begin{frame}[fragile]
  \frametitle{Elemento HTML <ins>}

  El elemento HTML \eti{<ins>} define un texto que se ha insertado
  en un documento. Los navegadores normalmente subrayarán el texto
  insertado:

  \vspace{\baselineskip}
  \begin{lstlisting}
<p>Mi color favorito es el <del>azul</del> <ins>rojo</ins>.</p>
  \end{lstlisting}
\end{frame}

\begin{frame}[fragile]
  \frametitle{Elemento HTML <sub>}

  El elemento HTML \eti{<sub>} define el texto del subíndice.
  El texto del subíndice aparece medio carácter por debajo de la
  línea normal y, a veces, se representa en una fuente más pequeña.
  El texto de subíndice se puede utilizar para fórmulas químicas,
  como $H_2O$:

  \vspace{\baselineskip}
  \begin{lstlisting}
<p>Este es un <sub>texto como subíndice</sub>.</p>
  \end{lstlisting}
\end{frame}

\begin{frame}[fragile]
  \frametitle{Elemento HTML <sup>}

  El elemento HTML \eti{<sup>} define texto en superíndice.
  El texto en superíndice aparece medio carácter por encima de
  la línea normal y, a veces, se representa en una fuente más
  pequeña. El texto en superíndice se puede usar para notas al
  pie, como $WWW^{[1]}$:

  \vspace{\baselineskip}
  \begin{lstlisting}
<p>Este es un <sup>texto como superíndice</sup>.</p>
  \end{lstlisting}
\end{frame}

\begin{frame}[c]{Resumen de unos formatos de texto en HTML}
  \begin{table}[]
  \begin{tabular}{cll}
    \textbf{Etiqueta} &  \textbf{Descripción} \\
    \rowcolor{light-gray}
    \eti{<b>} & Texto en negrita \\
    \eti{<strong>} & Texto importante \\
    \rowcolor{light-gray}
    \eti{<i>} & Texto en cursiva/itálica \\
    \eti{<em>} & Texto enfatizado \\
    \rowcolor{light-gray}
    \eti{<mark>} & Texto marcado \\
    \eti{<small>} & Texto más pequeño \\
    \rowcolor{light-gray}
    \eti{<del>} & Texto eliminado \\
    \eti{<ins>} & Texto insertado \\
    \rowcolor{light-gray}
    \eti{<sub>} & Texto de subíndice \\
    \eti{<sup>} & Texto en superíndice \\
  \end{tabular}
  \end{table}
\end{frame}

\section{Citas de texto}


\begin{frame}[c]{Elementos de cita de texto}
  En esta sección repasaremos los elementos
  \eti{<blockquote>}, \eti{<q>}, \eti{<abbr>},
  \eti{<address>}, \eti{<cite>} y
  \eti{<bdo>} del HTML.
\end{frame}

\begin{frame}[fragile]
  \frametitle{HTML <blockquote> para citas}

  El elemento HTML \eti{<blockquote>} define una sección
  que se cita de otra fuente.

  \vspace{\baselineskip}
  Los navegadores suelen sangrar (poner una sangría) los
  elementos \eti{<blockquote>}.

  \vspace{\baselineskip}
  \begin{lstlisting}
<p>Aquí hay una cita del sitio del INEGI:</p>
<blockquote cite="https://inegi.org.mx/inegi/quienes_somos.html">
Somos un organismo público autónomo responsable de normar y coordinar
el Sistema Nacional de Información Estadística y Geográfica, así como
de captar y difundir información de México en cuanto al territorio,
los recursos, la población y economía, que permita dar a conocer las
características de nuestro país y ayudar a la toma de decisiones.
</blockquote>
  \end{lstlisting}
\end{frame}

\begin{frame}[fragile]
  \frametitle{HTML <q> para citas cortas}

  La etiqueta \eti{<q>} de HTML define una cita corta.

  \vspace{\baselineskip}
  Los navegadores normalmente insertan comillas alrededor de la cita.

  \vspace{\baselineskip}
  \begin{lstlisting}
<p>El INEGI <q> norma y coordina el Sistema Nacional de
   Información Estadística y Geográfica</q> </p>
  \end{lstlisting}
\end{frame}

\begin{frame}[fragile]
  \frametitle{HTML <abbr> para abreviaturas}

  La etiqueta HTML \eti{<abbr>} define una abreviatura o un
  acrónimo, como "HTML", "CSS", "Sr.", "Dr.", "OMS", "ATM".

  \vspace{\baselineskip}
  Marcar abreviaturas puede brindar información útil a los navegadores,
  sistemas de traducción y motores de búsqueda.

  \vspace{\baselineskip}
  \textbf{Sugerencia}: use el atributo de \atri{title} global para mostrar
  la descripción de la abreviatura/acrónimo cuando pase el mouse sobre
  el elemento.

  \vspace{\baselineskip}
  \begin{lstlisting}
<p>La <abbr title="Organización Mundial de la Salud">OMS</abbr>
  fue fundada en 1948.</p>
  \end{lstlisting}
\end{frame}

\begin{frame}[fragile]
  \frametitle{HTML <address> para información de contacto}

  La etiqueta HTML \eti{<address>} define la información de
  contacto del autor/propietario de un documento o artículo.

  \vspace{\baselineskip}
  La información de contacto puede ser una dirección de correo
  electrónico, URL, dirección física, número de teléfono,
  identificador de redes sociales, etc.

  \vspace{\baselineskip}
  El texto del elemento \eti{<address>} generalmente se presenta
  en cursiva/itálica y los navegadores siempre agregarán un salto
  de línea antes y después del elemento \eti{<address>}.

  \vspace{\baselineskip}
  \begin{lstlisting}
<address>
Blvd. del Rodeo 401 <br>
El Vigía <br>
C.P. 45140 <br>
Zapopan, Jal.
</address>
  \end{lstlisting}
\end{frame}

\begin{frame}[fragile]
  \frametitle{HTML <cite> para título de trabajo}

  La etiqueta HTML \eti{<cite>} define el título de un trabajo
  creativo (por ejemplo, un libro, un poema, una canción,
  una película, una pintura, una escultura, etc.).

  \begin{exampleblock}{Nota:}
    El nombre de una persona no es el título de una obra.
  \end{exampleblock}

  \vspace{\baselineskip}
  El texto del elemento \eti{<cite>} normalmente se muestra en
  cursiva/itálica.

  \vspace{\baselineskip}
  \begin{lstlisting}
<p><cite> El grito </cite> por Edvard Munch. Pintado en 1893.</p>
  \end{lstlisting}
\end{frame}

\begin{frame}[fragile]
  \frametitle{HTML <bdo> para anulación bidirecciona}

  BDO son las iniciales del ingles \emph{Bi-Directional Override},
  que significa anulación bidireccional.

  \vspace{\baselineskip}
  La etiqueta HTML \eti{<bdo>} se utiliza para anular la
  dirección del texto actual:

  \vspace{\baselineskip}
  \begin{lstlisting}
<bdo dir="rtl"> Este texto se escribirá de derecha a izquierda </bdo>
  \end{lstlisting}

  \vspace{\baselineskip}
  \textbf{rtl} significa right-to-left, que significa de derecha-a-izquierda

\end{frame}

\begin{frame}[c]{Resumen de la sección de citas de texto}
  \begin{table}[]
  \begin{tabular}{cll}
    \textbf{Etiqueta} &  \textbf{Descripción} \\
    \rowcolor{light-gray}
      \eti{<abbr>} & Define una abreviatura o acrónimo \\
      \eti{<dirección>}  & Define la información de contacto del \\
                   & autor/propietario de un documento \\
    \rowcolor{light-gray}
      \eti{<bdo>} & Define la dirección del texto \\
      \eti{<blockquote>}  & Define una sección que se cita de otra fuente \\
    \rowcolor{light-gray}
      \eti{<cite>} & Define el título de una obra \\
      \eti{<q>} & Define una cita corta en línea \\
  \end{tabular}
  \end{table}

\end{frame}

\section{Comentarios en HTML}

\begin{frame}[fragile]
  \frametitle{Etiqueta de comentario de HTML}

  Los comentarios HTML no se muestran en el navegador,
  pero pueden ayudar a documentar su código fuente HTML.

  \vspace{\baselineskip}
  Puede agregar comentarios a su fuente HTML usando la siguiente sintaxis:

  \vspace{\baselineskip}
  \begin{lstlisting}
<!-- Escribe tu comentario aquí -->
  \end{lstlisting}

  \vspace{\baselineskip}
  Observe que hay un signo de exclamación (!) en la etiqueta inicial,
  pero no en la etiqueta final.

  \begin{alertblock}{Nota:}
    el navegador no muestra los comentarios, pero pueden
    ayudar a documentar su código fuente HTML.
  \end{alertblock}
\end{frame}

\begin{frame}[fragile]
  \frametitle{Añadir comentarios}

  Con los comentarios puedes colocar notificaciones y
  recordatorios en tu código HTML:

  \vspace{\baselineskip}
  \begin{lstlisting}
<!-- Este es un comentario -->

<p> Este es un párrafo. </p>

<!-- Acuerdate agregar más información aquí -->
  \end{lstlisting}
\end{frame}

\begin{frame}[fragile]
  \frametitle{Ocultar contenido}

  Los comentarios se pueden utilizar para ocultar contenido.

  \vspace{\baselineskip}
  Lo que puede ser útil si oculta contenido temporalmente:

  \vspace{\baselineskip}
  \begin{lstlisting}
<p> Este es un párrafo. </p>

<!-- <p> Este es otro párrafo. </p> -->

<p> Este es un párrafo también. </p>
  \end{lstlisting}
\end{frame}

\begin{frame}[fragile]
  \frametitle{Ocultar contenido}

  También puede ocultar más de una línea, todo lo que esté
  entre \eti{<!--} y se ocultará \eti{-->} de la pantalla.

  \vspace{\baselineskip}
  \begin{lstlisting}
<p> Este es un párrafo. </p>

<!--
<p> En la siguiente imagen: </p>
<img border="0" src="la_imagen.jpg" alt="Esta imagen">
-->

<p> Este es un párrafo también. </p>
  \end{lstlisting}

  \vspace{\baselineskip}
  Los comentarios también son excelentes para depurar HTML, ya
  que puede comentar líneas de código HTML, una a la vez, para
  buscar errores.
\end{frame}

\begin{frame}[fragile]
  \frametitle{Ocultar contenido en línea}

  Los comentarios se pueden usar para ocultar partes en
  medio del código HTML.

  \vspace{\baselineskip}
  \begin{lstlisting}
<p> Este <!-- texto de aquí --> es un párrafo. </p>
  \end{lstlisting}
\end{frame}
