% ex: ts=2 sw=2 sts=2 et filetype=tex
% SPDX-License-Identifier: CC-BY-SA-4.0

\section{Listas en HTML}

\begin{frame}[c]{Listas}

  Las listas HTML permiten a los desarrolladores web agrupar
  un conjunto de elementos relacionados en listas.

  \vspace{\baselineskip}
  \begin{columns}
    \column{0.05\textwidth}
    \column{0.45\textwidth}
      Esta es una lista sin ordenar/numerar:
      \begin{itemize}
        \item un elemento
        \item otro elemento
        \item otro elemento más 
      \end{itemize}
    \column{0.5\textwidth}
      Esta es una lista ordenada/numerada:
      \begin{enumerate}
        \item Primer elemento
        \item Segundo elemento
        \item Tercer elemento
      \end{enumerate}
  \end{columns}
\end{frame}

\begin{frame}[fragile]
  \frametitle{Listas sin numerar o sin orden (desordenada)}

  Una lista desordenada (sin orden) comienza con la
  etiqueta \eti{<ul>} (que en inglés
  es: \textbf{u}norder \textbf{l}ist).
  Cada elemento de la lista comienza con la etiqueta \eti{<li>} (que en
  inglés es: \textbf{l}ist \textbf{i}tem).

  \vspace{\baselineskip}
  Los elementos de la lista se marcarán con viñetas
  (pequeños círculos negros) de forma predeterminada:

  \vspace{\baselineskip}
  \begin{lstlisting}
<ul>
  <li>Café</li>
  <li>Te</li>
  <li>Leche</li>
</ul>  
  \end{lstlisting}
\end{frame}

\begin{frame}[fragile]
  \frametitle{Lista ordenada}

  Una lista ordenada comienza con la etiqueta \eti{<ol>} (que en ingés
  es: \textbf{o}rdered \textbf{l}ist). Cada elemento de la lista
  comienza con la etiqueta \eti{<li>} ( que en
  inglés es: \textbf{l}ist \textbf{i}tem).

  \vspace{\baselineskip}
  Los elementos de la lista se marcarán con números de
  forma predeterminada:
  \vspace{\baselineskip}
  \begin{lstlisting}
<ol>
  <li>Café</li>
  <li>Te</li>
  <li>Leche</li>
</ol>  
  \end{lstlisting}
\end{frame}

\begin{frame}[fragile]
  \frametitle{Listas de descripción}

  HTML también admite listas de descripción.

  \vspace{\baselineskip}
  Una lista de descripción es una lista de términos,
  con una descripción de cada término.

  \vspace{\baselineskip}
  La etiqueta \eti{<dl>} define la lista de descripciones,
  la etiqueta \eti{<dt>} define el término (nombre) y
  la etiqueta \eti{<dd>} describe cada término:

  \vspace{\baselineskip}
  \begin{lstlisting}
<dl>
  <dt>Café</dt>
  <dd>- Bebida caliente color negro</dd>
  <dt>Leche</dt>
  <dd>- Bebida fría color blanco</dd>
</dl> 
  \end{lstlisting}
\end{frame}

\begin{frame}[c]{Resumen de etiquetas de listas en HTML}
  \begin{table}[]
  \begin{tabular}{cll}
    \textbf{Etiqueta} &  \textbf{Descripción} \\
    \rowcolor{light-gray}
    \eti{<ul>} & Define una lista sin orden/numerar \\
    \eti{<ol>} & Define una lista ordenada/numerada \\
    \rowcolor{light-gray}
    \eti{<li>} & Define un elemento de una lista \\
    \eti{<dl>} & Define una lista de descripción \\
    \rowcolor{light-gray}
    \eti{<dt>} & Define un término en una lista de descripción \\
    \eti{<dd>} & Describe el término en una lista de descripción \\
  \end{tabular}
  \end{table}
\end{frame}
