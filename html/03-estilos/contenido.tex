% ex: ts=2 sw=2 sts=2 et filetype=tex
% SPDX-License-Identifier: CC-BY-SA-4.0

\section{Encabezados HTML}

\begin{frame}[fragile]
  \frametitle{Encabezados HTML}

  Los encabezados HTML son títulos o subtítulos que desea
  mostrar en una página web.

  \vspace{\baselineskip}
  Los encabezados (heading en ingles) del HTML se definen con
  las etiquetas \eti{<h1>} a la \eti{<h6>}.

  \vspace{\baselineskip}
  \eti{<h1>} define el encabezado más importante.
  \eti{<h6>} define el encabezado menos importante.

  \vspace{\baselineskip}
  \begin{lstlisting}
<h1>Encabezado 1</h1>
<h2>Encabezado 2</h2>
<h3>Encabezado 3</h3>
<h4>Encabezado 4</h4>
<h5>Encabezado 5</h5>
<h6>Encabezado 6</h6>
  \end{lstlisting}

  \begin{exampleblock}{Nota:}
    Los navegadores agregan automáticamente un espacio en blanco
    (un margen) antes y después de un encabezado.
  \end{exampleblock}
\end{frame}

\begin{frame}[c]{Los encabezados son importantes}

  Los motores de búsqueda utilizan los encabezados para indexar
  la estructura y el contenido de sus páginas web.

  \vspace{\baselineskip}
  Los usuarios a menudo hojean una página por sus títulos.
  Es importante utilizar encabezados para mostrar la estructura
  del documento.

  \vspace{\baselineskip}
  La etiqueta \eti{<h1>} deben usarse para los encabezados principales,
  seguidos de \eti{<h2>}, luego los menos importantes
  \eti{<h3>}, y así sucesivamente.

  \begin{block}{Nota:}
    Use encabezados HTML solo para encabezados.
    No use encabezados para hacer que el texto sea \textbf{GRANDE}
    o en \textbf{negrita}.
  \end{block}
\end{frame}

\begin{frame}[fragile]
  \frametitle{Encabezados más grandes}

  Cada encabezado HTML tiene un tamaño predeterminado.
  Sin embargo, puede especificar el tamaño de cualquier encabezado
  con el tributo \atri{style}, utilizando la propiedad CSS \atri{font-size}:

  \vspace{\baselineskip}
  \begin{lstlisting}
<h1 style="font-size:60px;">Un gran título 1</h1>
  \end{lstlisting}
\end{frame}

\section{Párrafos en HTML}

\begin{frame}[fragile]
  \frametitle{Párrafos en HTML}

  Un párrafo siempre comienza en una nueva línea y suele ser un
  bloque de texto.

  \vspace{\baselineskip}
  El elemento HTML \eti{<p>} define un párrafo.

  \vspace{\baselineskip}
  Un párrafo siempre comienza en una nueva línea y los navegadores
  agregan automáticamente un espacio en blanco (un margen) antes
  y después de un párrafo.

  \vspace{\baselineskip}
  \begin{lstlisting}
<p>Este es un párrafo.</p>
<p>Este es otro párrafo.</p>
  \end{lstlisting}
\end{frame}

\begin{frame}[c]{Mostrando HTML en pantalla}

  No puede estar seguro de cómo se mostrará HTML.

  \vspace{\baselineskip}
  Las pantallas grandes o pequeñas y las ventanas redimensionadas
  crearán resultados diferentes.

  \vspace{\baselineskip}
  Con HTML, no puede cambiar la visualización agregando espacios
  adicionales o líneas adicionales en su código HTML.

  \vspace{\baselineskip}
  El navegador eliminará automáticamente cualquier espacio y
  línea adicional cuando se muestre la página

  \vspace{\baselineskip}
  El siguiente ejemplo muestra que los espacios del código no se ven en un
  navegador web (así es como funciona HTML).
\end{frame}

\begin{frame}[fragile]
  \frametitle{Ejemplo de espacios en párrafos}
  \begin{lstlisting}
<p>
Este es un párrafo
contiene varias líneas
en el código pero el
navegador web
las ignora
</p>

<p>
Este es un     párrafo
contiene         varios     espacios
en el      código   pero    el
      navegador     web
las     ignora
</p>
  \end{lstlisting}
\end{frame}

\begin{frame}[fragile]
  \frametitle{Líneas horizontales HTML}

  La etiqueta \eti{<hr>} (horizontal rule) define una ruptura
  temática en una página HTML y, con mayor frecuencia, se muestra
  como una regla/línea horizontal.

  \vspace{\baselineskip}
  El elemento \eti{<hr>} se usa para separar el contenido (o definir
  un cambio) en una página HTML:

  \vspace{\baselineskip}
  \begin{lstlisting}
<h1>Este es un título 1</h1>
<p>Aquí algo de texto.</p>
<hr>
<h2>Este es un encabezado 2</h2>
<p>Aqúi más texto.</p>
<hr>
  \end{lstlisting}

  \vspace{\baselineskip}
  La etiqueta \eti{<hr>} es una etiqueta vacía, lo que significa
  que no tiene una etiqueta final.
\end{frame}

\begin{frame}[fragile]
  \frametitle{Saltos de línea HTML}

  El elemento HTML \eti{<br>} define un salto de línea.

  \vspace{\baselineskip}
  Puedes usar \eti{<br>} si deseas un salto de línea
  (una nueva línea) sin comenzar un nuevo párrafo:

  \vspace{\baselineskip}
  \begin{lstlisting}
<p>Este es un<br>párrafo que tiene<br>saltos de línea.</p>
  \end{lstlisting}

  \vspace{\baselineskip}
  La etiqueta \eti{<br>} es una etiqueta vacía, lo que significa
  que no tiene una etiqueta final.
\end{frame}

\begin{frame}[fragile]
  \frametitle{El problema del Poeta}

  Este poema se mostrará en una sola línea:

  \vspace{\baselineskip}
  \begin{lstlisting}
<p>
    ¿Mi tierra?
    Mi tierra eres tú.

    ¿Mi gente?
    Mi gente eres tú.

    El destierro y la muerte
    para mi están adonde
    no estés tú.

    ¿Y mi vida?
    Dime, mi vida,
    ¿qué es, si no eres tú?
</p>
  \end{lstlisting}
\end{frame}

\begin{frame}[fragile]
  \frametitle{Solución: el elemento HTML <pre>}

  El elemento HTML \eti{<pre>} define texto con formato previo.

  \vspace{\baselineskip}
  El texto dentro de un elemento \eti{<pre>} se muestra en una fuente
  (tipo de letra) de ancho fijo (generalmente Courier) y conserva tanto
  los espacios como los saltos de línea:

  \vspace{\baselineskip}
  \begin{lstlisting}
<pre>
    ¿Mi tierra?
    Mi tierra eres tú.

    ¿Mi gente?
    Mi gente eres tú.

    El destierro y la muerte
    para mi están adonde
    no estés tú.

    ...
</pre>
  \end{lstlisting}
\end{frame}

\begin{frame}[c]{Resumen de la sección}
  \begin{table}[]
  \begin{tabular}{cll}
    \textbf{Etiqueta} &  \textbf{Descripción} \\
    \rowcolor{light-gray}
    <p>  &  Define un párrafo \\
    <hr> &  Crea una linea horizontal \\
    \rowcolor{light-gray}
    <br> &  Inserta un salto de línea \\
    <pre> &  Define un texto pre-formateado \\
  \end{tabular}
  \end{table}
\end{frame}

\section{Estilos HTML}

\begin{frame}[fragile]
  \frametitle{El atributo de estilo en HTML}

  El atributo HTML \atri{style} se usa para agregar estilos a un
  elemento, como color, fuente, tamaño y más.

  \vspace{\baselineskip}
  La configuración del estilo de un elemento HTML se puede hacer con
  el atributo \atri{style}.

  \vspace{\baselineskip}
  El atributo HTML \atri{style} tiene la siguiente sintaxis:

  \vspace{\baselineskip}
  \eti{<etiqueta} \atri{style=}\propi{"propiedad:varlor;"}\eti{>}

  \vspace{\baselineskip}
  La \propi{propiedad} es una propiedad CSS.
  El \propi{valor} es un valor CSS.

  \begin{block}{}
    Aprenderemos más sobre CSS más adelante en este curso.
  \end{block}
\end{frame}

\begin{frame}[fragile]
  \frametitle{Color de fondo}

  La propiedad CSS \propi{background-color} define el color de
  fondo de un elemento HTML.

  \vspace{\baselineskip}
  Ejemplo: Establezca el color de fondo de una página en azul polvo:
  \begin{lstlisting}
<body style="background-color:powderblue;">

<h1>Este es un encabezado</h1>
<p>Este es un párrafo.</p>

</body> 
  \end{lstlisting}
\end{frame}

\begin{frame}[fragile]
  \frametitle{Color de fondo}

  \vspace{\baselineskip}
  Ejemplo: Establezca el color de fondo para dos elementos diferentes:
  \begin{lstlisting}
<body>

<h1 style="background-color:powderblue;">Este es un encabezado</h1>
<p  style="background-color:tomato;">Este es un párrafo.</p>

</body> 
  \end{lstlisting}
\end{frame}

\begin{frame}[fragile]
  \frametitle{Color de texto}

  La propiedad CSS \propi{color} define el color del
  texto de un elemento HTML:

  \vspace{\baselineskip}
  \begin{lstlisting}
<h1 style="color:blue;">Este es un encabezado</h1>
<p  style="color:red;">Este es un párrafo.</p>
  \end{lstlisting}
\end{frame}

\begin{frame}[fragile]
  \frametitle{Fuentes}

  La propiedad CSS \propi{font-family} define la fuente
  que se utilizará para un elemento HTML:

  \vspace{\baselineskip}
  \begin{lstlisting}
<h1 style="font-family:verdana;">Este es un encabezado</h1>
<p  style="font-family:courier;">Este es un párrafo.</p>
  \end{lstlisting}
\end{frame}

\begin{frame}[fragile]
  \frametitle{Tamaño del texto}

  La propiedad CSS \propi{font-size} define el
  tamaño del texto para un elemento HTML:

  \vspace{\baselineskip}
  \begin{lstlisting}
<h1 style="font-size:300%;">Este es un encabezado</h1>
<p  style="font-size:160%;">Este es un párrafo.</p>
  \end{lstlisting}
\end{frame}

\begin{frame}[fragile]
  \frametitle{Alineación del texto}

  La propiedad CSS \propi{text-align} define la
  alineación horizontal del texto para un elemento HTML:

  \vspace{\baselineskip}
  \begin{lstlisting}
<h1 style="text-align:center;">Este es un encabezado</h1>
<p  style="text-align:right;">Este es un párrafo.</p>
<p  style="text-align:center;">Este es otro párrafo.</p>
  \end{lstlisting}
\end{frame}

\begin{frame}[c]{Resumen de estilos}
  \begin{itemize}
    \item Use el atributo \atri{style} para diseñar elementos HTML
    \item Utilizar \propi{background-color} para el color de fondo
    \item Usar \propi{color} para colores de texto
    \item Usar \propi{font-family} para fuentes de texto
    \item Usar \propi{font-size} para tamaños de texto
    \item Usar \propi{text-align} para alineación de texto
  \end{itemize}
\end{frame}
