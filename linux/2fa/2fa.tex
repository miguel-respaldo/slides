% ex: ts=2 sw=2 sts=2 et filetype=tex
% SPDX-License-Identifier: CC-BY-SA-4.0

\documentclass[aspectratio=169]{beamer}

\usetheme{Boadilla}

\setbeamertemplate{navigation symbols}{} % To remove the navigation symbols from the bottom of all slides

\graphicspath{{../img/}}
\usepackage[utf8]{inputenc}
\usepackage[T1]{fontenc} %% https://tex.stackexchange.com/questions/664/why-should-i-use-usepackaget1fontenc
\usepackage{graphicx}
\usepackage{listings}
\usepackage{colortbl}
\usepackage{tikz}
\usetikzlibrary{trees}

\title{Autenticación de dos factores (2FA)}

% ex: ts=4 sw=4 sts=4 et filetype=tex
% SPDX-License-Identifier: CC-BY-SA-4.0

\author{Nancy Ruiz Monroy} % Your name

\institute[EMMA]
{
 Escuela Militar de \\% Your institution for the title page
 Mantenimiento y Abastesimiento\\
\medskip
\textit{docentecivilemma.36@gmail.com} % Your email address
}

%\logo{\includegraphics[height=1cm]{logo.png}}

%\titlegraphic{
%    \includegraphics[width=4cm]{logo_full.png}
%}

% ex: ts=4 sw=4 sts=4 et filetype=tex
% SPDX-License-Identifier: CC-BY-SA-4.0

% Date, can be changed to a custom date
\date{
Primavera 2024
}

% ex: ts=2 sw=2 sts=2 et filetype=tex
% SPDX-License-Identifier: CC-BY-SA-4.0

%New colors defined below
\definecolor{editorNumeros}{rgb}{0.5,0.5,0.5}
\definecolor{editorGray}{rgb}{0.95, 0.95, 0.95}
\definecolor{editorOcher}{rgb}{1, 0.5, 0} % #FF7F00 -> rgb(239, 169, 0)
\definecolor{editorGreen}{rgb}{0, 0.5, 0} % #007C00 -> rgb(0, 124, 0)
\definecolor{editorBrown}{rgb}{0.69,0.31,0.31}

%Code listing style named "griscmd"
\lstdefinestyle{griscmd}{
  % General design
  backgroundcolor=\color{editorGray},
  basicstyle={\ttfamily},
  numberstyle=\tiny\color{editorNumeros},
  xleftmargin={0.2cm},
  numbersep=5pt,
  numbers=left,
  stepnumber=1,
  firstnumber=1,
  numberfirstline=true,
  % Code design
  identifierstyle=\color{black},
  keywordstyle=\color{blue}\bfseries,
  ndkeywordstyle=\color{editorGreen}\bfseries,
  stringstyle=\color{editorOcher}\ttfamily,
  commentstyle=\color{editorBrown}\ttfamily,
  % Code
  alsodigit={.:;},
  tabsize=2,
  showtabs=false,
  showspaces=false,
  showstringspaces=false,
  extendedchars=true,
  breaklines=true,
  % Español
  literate=%
  {á}{{\'a}}1 {é}{{\'e}}1 {í}{{\'i}}1 {ó}{{\'o}}1 {ú}{{\'u}}1
  {Á}{{\'A}}1 {É}{{\'E}}1 {Í}{{\'I}}1 {Ó}{{\'O}}1 {Ú}{{\'U}}1
  {¡}{{!`}}1  {¿}{{?`}}1
  {ñ}{{\~n}}1 {Ñ}{{\~N}}1
}

%"griscmd" code listing set
\lstset{style=griscmd}



%------------------------------------------------------------
%The next block of commands puts the table of contents at the
%beginning of each section and highlights the current section:
%\AtBeginSection[]
%{
%  \begin{frame}
%    \frametitle{Contenido}
%    \tableofcontents[currentsection]
%  \end{frame}
%}
%------------------------------------------------------------

\newcommand{\nologo}{\setbeamertemplate{logo}{}} % command to set the logo to nothing
\newcommand{\pausa}{\pause} % Para usar una pausa en las presentaciones
%\newcommand{\pausa}{}      % Para que NO salgan las pausas

\begin{document}

{
\nologo
\begin{frame}
    \titlepage
\end{frame}
}

\begin{frame}
    \frametitle{Contenido}
    \tableofcontents
\end{frame}

% ex: ts=2 sw=2 sts=2 et filetype=tex
% SPDX-License-Identifier: CC-BY-SA-4.0
\begin{frame}
    \frametitle{Contenido}
    \tableofcontents
\end{frame}

\section{Tipos de Datos}

\begin{frame}[c]{Tipo de datos integrados}

  \vspace{\baselineskip}
  En programación, el tipo de dato es un concepto importante.

  \vspace{\baselineskip}
  Las variables pueden almacenar datos de diferentes tipos y diferentes
  tipos pueden hacer cosas diferentes.

  \vspace{\baselineskip}
  Python tiene los siguientes tipos de datos integrados de forma
  predeterminada, en estas categorías:

  \begin{table}[]
  \begin{tabular}{ll}
    Texto & \textcolor{codeKeyword2}{str} \\
    \pausa
    Numérico & \textcolor{codeKeyword2}{int}, \textcolor{codeKeyword2}{float},
     \textcolor{codeKeyword2}{complex} \\
    \pausa
    Secuencia & \textcolor{codeKeyword2}{list}, \textcolor{codeKeyword2}{tuple},
     \textcolor{codeKeyword2}{range} \\
    \pausa
    Mapeo & \textcolor{codeKeyword2}{dict} \\
    \pausa
    Conjunto & \textcolor{codeKeyword2}{set},
     \textcolor{codeKeyword2}{frozenset} \\
    \pausa
    Boleano & \textcolor{codeKeyword2}{bool} \\
    \pausa
    Binario & \textcolor{codeKeyword2}{bytes},
     \textcolor{codeKeyword2}{bytearray}, \textcolor{codeKeyword2}{memoryview} \\
  \end{tabular}
  \end{table}
\end{frame}

\begin{frame}[fragile]
  \frametitle{Obteniendo el tipo de dato}

  Se puede obtener el tipo de dato con la función
  \textcolor{codeKeyword2}{type}()

  \vspace{\baselineskip}
  \begin{lstlisting}[language=Python]
  x = str(3)   # x guarda '3'
  y = int(3)   # y guarda 3
  z = float(3) # z guarda 3.0

  print( type(x) )
  print( type(y) )
  print( type(z) )
  \end{lstlisting}
\end{frame}

\begin{frame}[fragile]
  \frametitle{Configurando el tipo de dato}

  \vspace{\baselineskip}
  En Python, el tipo de dato es configurado/asociado cuando se asigna un
  valor a una variable:

  \begin{lstlisting}[language=Python]
x = "Hola Mundo"                      # str
x = 20                                # int
x = 20.5                              # float
x = 1j                                # complex
x = ["manzana", "platano", "naranja"] # list
x = ("manzana", "platano", "naranja") # tuple
x = range(6)                          # range
x = {"nombre" : "juan", "edad": 18}   # dict
x = {"manzana", "platano", "naranja"} # set
x = frozenset({"manzana", "platano", "naranja"}) # frozenset
x = True                              # bool
x = b"Hola"                           # bytes
x = bytearray(5)                      # bytearray
x = memoryview(bytes(5))              # memoryview
  \end{lstlisting}
\end{frame}

\begin{frame}[fragile]
  \frametitle{Configurando un tipo de dato especifico}

  \vspace{\baselineskip}
  Si se quiere especificar un tipo de dato especifico, se puede utilizar
  las siguientes funciones constructoras:

  \begin{lstlisting}[language=Python]
x = str("Hola Mundo")                        # str
x = int(20)                                  # int
x = float(20.5)                              # float
x = complex(1j)                              # complex
x = list(("manzana", "platano", "naranja"))  # list
x = tuple(("manzana", "platano", "naranja")) # tuple
x = range(6)                                 # range
x = dict(nombre = "juan", edad = 18)         # dict
x = set(("manzana", "platano", "naranja"))   # set
x = frozenset(("manzana", "platano", "naranja")) # frozenset
x = bool(5)                                  # bool
x = bytes(5)                                 # bytes
x = bytearray(5)                             # bytearray
x = memoryview(bytes(5))                     # memoryview
  \end{lstlisting}
\end{frame}

\section{Números}

\begin{frame}[fragile]
  \frametitle{Números en Python}

  \vspace{\baselineskip}
  Hay tres tipos numéricos en Python:

  \begin{itemize}
    \item \textcolor{codeKeyword2}{int}
    \item \textcolor{codeKeyword2}{float}
    \item \textcolor{codeKeyword2}{complex}
  \end{itemize}

  Las variables numéricas son creadas cuando se le asigna su valor:

  \begin{lstlisting}[language=Python]
  x = 1    # int
  y = 2.8  # float
  z = 1j   # complex
  \end{lstlisting}

  \pausa
  Para verificar el tipo de dato de cualquier objeto en Python, se usa la
  función \textcolor{codeKeyword2}{type}():

  \begin{lstlisting}[language=Python]
  print( type(x) )
  print( type(y) )
  print( type(z) )
  \end{lstlisting}
\end{frame}

\begin{frame}[fragile]
  \frametitle{Números Enteros}

  Un entero (\textcolor{codeKeyword2}{int}) es un número positivo o
  negativo, sin decimales de longitud ilimitada.

  \vspace{\baselineskip}
  \begin{lstlisting}[language=Python]
  x = 1
  y = 35656222554887711
  z = -3255522

  print( type(x) )
  print( type(y) )
  print( type(z) )
  \end{lstlisting}
\end{frame}

\begin{frame}[fragile]
  \frametitle{Números de punto flotante}

  Un número de punto flotante (\textcolor{codeKeyword2}{float}), es un número
  positivo o negativo, que contiene uno o más decimales.

  \vspace{\baselineskip}
  \begin{lstlisting}[language=Python]
  x = 3.1416
  y = 1.0
  z = -35.59

  print( type(x) )
  print( type(y) )
  print( type(z) )
  \end{lstlisting}
\end{frame}

\begin{frame}[fragile]
  \frametitle{Números de punto flotante}

  Los número de punto flotante (\textcolor{codeKeyword2}{float}) pueden ser
  escritos como números en \emph{notación científica} con una "\textbf{e}"
  para indicar la potencia de 10.

  \vspace{\baselineskip}
  \begin{lstlisting}[language=Python]
  x = 35e3
  y = 12E4
  z = -87.7e100

  print( type(x) )
  print( type(y) )
  print( type(z) )
  \end{lstlisting}
\end{frame}

\begin{frame}[fragile]
  \frametitle{Números Complejos}

  Los números complejos son escritos con una "\textbf{j}" para indicar
  la parte imaginaria:

  \vspace{\baselineskip}
  \begin{lstlisting}[language=Python]
  x = 3+5j
  y = 5j
  z = -2.5j

  print( type(x) )
  print( type(y) )
  print( type(z) )
  \end{lstlisting}
\end{frame}

\begin{frame}[fragile]
  \frametitle{Conversión entre tipo de datos}

  \vspace{\baselineskip}
  Se pueden convertir de un tipo de dato a otro con las siguientes
  funciones \textcolor{codeKeyword2}{int}(), \textcolor{codeKeyword2}{float}()
  y \textcolor{codeKeyword2}{complex}():

  \begin{lstlisting}[language=Python]
  x = 1    # int
  y = 2.8  # float
  z = -1j  # complex

  # convertir de int a float:
  a = float(x)
  # convertir de float a int:
  b = int(y)
  # convertir de int a complex:
  c = complex(x)
  \end{lstlisting}

  \pausa
  \begin{alertblock}{Nota}
    No se puede convertir un \textbf{número complejo} en otro tipo de dato numérico.
  \end{alertblock}
\end{frame}

\begin{frame}[fragile]
  \frametitle{Números aleatorios}

  Python no tiene una función de \textbf{random}() para
  hacer números aleatorios, pero Python tiene integrado un modulo llamado
  \textbf{random} que puede ser usado para hacer números
  aleatorios:

  \vspace{\baselineskip}
  Importamos el modulo de \textbf{random}, y desplegamos un número aleatorio entre 1
  y 9:

  \begin{lstlisting}[language=Python]
  import random

  print( random.randrange(1,10) )
  \end{lstlisting}
\end{frame}

\section{Casting}

\begin{frame}[c]{Especificando el tipo de variable}

  \vspace{\baselineskip}
  Puede haber ocasiones en las que se desee especificar un tipo de dato en una
  variable. Esto se puede hacer con el "\textbf{casting}". Python es un
  lenguaje orientado a objetos y, como tal, usa clases para definir tipos de
  datos, incluidos sus tipos primitivos.

  \pausa
  \vspace{\baselineskip}
  Por lo tanto, la conversión (\textbf{casting}) en python se realiza
  mediante funciones constructoras:

  \vspace{\baselineskip}
  \begin{description}
    \item[int()] construye un número entero a partir de una literal entero,
      una literal flotante (eliminando todos los decimales) o una literal
      de cadena (siempre que la cadena represente un número entero).
    \pausa
    \item[float()] construye un número flotante a partir de una literal
      entero, una literal flotante o una literal de cadena (siempre que la
      cadena represente un número flotante o entero)
    \pausa
    \item[str()] construye una cadena a partir de una amplia variedad de
      tipos de datos, incluidas cadenas, literales enteros y literales
      flotantes
  \end{description}
\end{frame}

\begin{frame}[fragile]
  \frametitle{Especificando el tipo de variable}
  \begin{lstlisting}[language=Python]
  # Integers (enteros)
  x = int(1)   # x sera 1
  y = int(2.8) # y sera 2
  z = int("3") # z sera 3

  # Floats (enteros con decimales)
  x = float(1)     # x sera 1.0
  y = float(2.8)   # y sera 2.8
  z = float("3")   # z sera 3.0
  w = float("4.2") # w sera 4.2

  # Strings (cadenas de texto)
  x = str("s1") # x sera 's1'
  y = str(2)    # y sera '2'
  z = str(3.0)  # z sera '3.0'
  \end{lstlisting}
\end{frame}

\section{Lógicos}

\begin{frame}[fragile]
  \frametitle{Valores Lógicos (\emph{Boleanos})}
  Los booleanos representan uno de dos valores: Verdadero
  (\textcolor{codeKeyword2}{True}) o Falso
  (\textcolor{codeKeyword2}{False}).

  \pausa
  \vspace{\baselineskip}
  En programación, a menudo es necesario saber si una expresión es verdadera
  o falsa. Se puede evaluar cualquier expresión en Python y obtener una de
  dos respuestas, Verdadero (\textcolor{codeKeyword2}{True})
  o Falso (\textcolor{codeKeyword2}{False}).

  \pausa
  \vspace{\baselineskip}
  Cuando se comparan dos valores, la expresión se evalúa y Python
  devuelve la respuesta booleana: 

  \begin{lstlisting}[language=Python]
  print(10 > 9)
  print(10 == 9)
  print(10 < 9)
  \end{lstlisting}
\end{frame}

\begin{frame}[fragile]
  \frametitle{Valores Lógicos (\emph{Boleanos})}

  Cuando se corre una condición en una sentencia
  \textcolor{codeKeyword}{if}, Python regresa Verdadero
  (\textcolor{codeKeyword2}{True}) o Falso
  (\textcolor{codeKeyword2}{False}):

  \vspace{\baselineskip}
  Ejemplo: imprimir un mensaje en función de si la condición es Verdadero
  (\textcolor{codeKeyword2}{True}) o Falso
  (\textcolor{codeKeyword2}{False}):
  \begin{lstlisting}[language=Python]
a = 200
b = 33

if b > a:
  print("b es mas grande que a")
else:
  print("b No es mas grande que a") 
  \end{lstlisting}
\end{frame}

\begin{frame}[fragile]
  \frametitle{Evaluando valores y variables}

  La función \textcolor{codeKeyword2}{bool}() permite evaluar cualquier
  valor y a cambio, regresa el valor Verdadero o Falso.

  \vspace{\baselineskip}
  \begin{lstlisting}[language=Python]
print(bool("Hola"))
print(bool(15))

x = "Hola"
y = 15

print(bool(x))
print(bool(y))
  \end{lstlisting}
\end{frame}

\begin{frame}[fragile]
  \frametitle{La mayoría de los valores son verdadero}

  \vspace{\baselineskip}
  La función \textcolor{codeKeyword2}{bool}() permite evaluar cualquier
  valor y a cambio, regresa el valor Verdadero o Falso.

  \pausa
  \vspace{\baselineskip}
  Casi cualquier valor se evalúa como \textbf{Verdadero} si tiene algún
  tipo de contenido.

  \pausa
  \vspace{\baselineskip}
  Cualquier cadena es verdadera, excepto las cadenas vacías.

  \pausa
  \vspace{\baselineskip}
  Cualquier número es verdadero, excepto 0.

  \pausa
  \vspace{\baselineskip}
  Cualquier lista, tupla, conjunto y diccionario son verdaderos, excepto
  los vacíos.
  \begin{lstlisting}[language=Python]
bool("abc")
bool(123)
bool(["manzana", "naranja", "platano"])
  \end{lstlisting}
\end{frame}

\begin{frame}[fragile]
  \frametitle{Algunos valores son falsos}

  De hecho, no hay muchos valores que se evalúen como \textbf{Falso},
  excepto los valores vacíos, como \textbf{()}, \textbf{[]}, \textbf{\{\}},
  \textbf{""}, el número \textbf{0} y el
  valor \textbf{None}. Y, por supuesto, el valor \textbf{False} se evalúa
  como falso.

  \vspace{\baselineskip}
  \begin{lstlisting}[language=Python]
bool(False)
bool(None)
bool(0)
bool("")
bool(())
bool([])
bool({}) 
  \end{lstlisting}
\end{frame}

\begin{frame}[fragile]
  \frametitle{Algunos valores son falsos}

  Un valor más, u objeto en este caso, se evalúa como Falso, y eso es si
  se tiene un objeto que está hecho a partir de una clase con una función
  \emph{\_\_len\_\_} que devuelve 0 o Falso: 

  \vspace{\baselineskip}
  \begin{lstlisting}[language=Python]
class myclass():
  def __len__(self):
    return 0

myobj = myclass()
print(bool(myobj))
  \end{lstlisting}
\end{frame}

\begin{frame}[fragile]
  \frametitle{Las funciones pueden regresar valores lógicos (boleanos)}

  Se puede crear una función que regrese un valor lógico (boleano)

  \vspace{\baselineskip}
  \begin{lstlisting}[language=Python]
def miFuncion() :
  return True

print(miFuncion())
  \end{lstlisting}
\end{frame}

\begin{frame}[fragile]
  \frametitle{Las funciones pueden regresar valores lógicos (boleanos)}

  Se puede ejecutar código en base al valor de retorno de una función:

  \vspace{\baselineskip}
  \begin{lstlisting}[language=Python]
def miFuncion():
  return True

if miFuncion():
  print("Si!")
else:
  print("No!") 
  \end{lstlisting}
\end{frame}

\begin{frame}[fragile]
  \frametitle{Las funciones pueden regresar valores lógicos (boleanos)}

  Python también tiene muchas funciones integradas que devuelven un valor
  booleano, como la función \textcolor{codeKeyword2}{isinstance}(),
  que se puede usar para determinar si un objeto es de un determinado
  tipo de dato: 

  \vspace{\baselineskip}
  \begin{lstlisting}[language=Python]
x = 200
print(isinstance(x, int)) 
  \end{lstlisting}
\end{frame}

\section{Operadores}

\begin{frame}[fragile]
  \frametitle{Operadores en Python}

  Los operadores son usados para realizar operaciones sobre las variables y
  sus valores.

  \vspace{\baselineskip}
  En el ejemplo de abajo, se usa el operador \textbf{+} para hacer la suma
  de dos valores:

  \vspace{\baselineskip}
  \begin{lstlisting}[language=Python]
  print(10 + 5)
  \end{lstlisting}
\end{frame}

\begin{frame}[c]{Operadores en Python}
  Python divide los operadores en los siguientes grupos:
  \pausa
  \begin{itemize}
    \item Aritméticos
    \pausa
    \item Asignación
    \pausa
    \item Comparación
    \pausa
    \item Lógicos
    \pausa
    \item Identidad
    \pausa
    \item Pertenencia (\emph{Membership})
    \pausa
    \item A nivel de bits (\emph{Bitwise})
  \end{itemize}
\end{frame}

\begin{frame}[c]{Operadores Aritméticos}

  Los operadores aritméticos son usados con valores numéricos para realizar
  operaciones matemáticas comunes:

  \begin{table}[]
  \begin{tabular}{cll}
    \textbf{Operador} &  \textbf{Nombre} & \textbf{Ejemplo} \\
    \rowcolor{light-gray}
    + & Adición        & $x + y$ \\
    \pausa
    - & Sustracción    & $x - y$ \pausa \\
    \rowcolor{light-gray}
    * & Multiplicación & $x * y$ \\
    \pausa
    / & División       & $x / y$ \pausa \\
    \rowcolor{light-gray}
    \% & Modulo        & $x \% y$ \\
    \pausa
    ** & Exponente     & $x ** y$ \pausa \\
    \rowcolor{light-gray}
    // & División entera & $x // y$ \\
  \end{tabular}
  \end{table}
\end{frame}

\begin{frame}[c]{Operadores de Asignación}

  \vspace{\baselineskip}
  Los operadores de asignación son usados para asignar valores a las
  variables:

  \begin{table}[]
  \begin{tabular}{cll}
    \textbf{Operador} &  \textbf{Ejemplo} & \textbf{lo mismo que} \\
    \rowcolor{light-gray}
    =  & x = 5  & x = 5  \pausa \\
    += & x += 3 & x = x + 3 \pausa \\
    \rowcolor{light-gray}
    -= & x -= 3 & x = x - 3 \pausa \\
    *= & x *= 3 & x = x * 3 \pausa \\
    \rowcolor{light-gray}
    /= & x /= 3 & x = x / 3 \pausa \\
    \%= & x \%= 3 & x = x \% 3 \pausa \\
    \rowcolor{light-gray}
    //= & x //= 3 & x = x // 3 \pausa \\
    **= & x **= 3 & x = x ** 3 \pausa \\
    \rowcolor{light-gray}
    \&= & x \&= 3 & x = x \& 3 \pausa \\
    \textbar= & x \textbar= 3 & x = x \textbar  3 \pausa \\
    \rowcolor{light-gray}
    \^{}= & x \^{}= 3 & x = x \^{} 3 \pausa \\
    $>>=$ & $x >>= 3$ & $x = x >> 3$ \pausa \\
    \rowcolor{light-gray}
    $<<=$ & $x <<= 3$ & $x = x << 3$ \\
  \end{tabular}
  \end{table}
\end{frame}

\begin{frame}[c]{Operadores de Comparación}

  Los operadores de comparación son usados para comparar dos valores:

  \begin{table}[]
  \begin{tabular}{cll}
    \textbf{Operador} &  \textbf{Nombre} & \textbf{Ejemplo} \\
    \rowcolor{light-gray}
    ==  & Igual  & x == y  \pausa \\
    !=  & No igual (Diferente)  & x != y  \pausa \\
    \rowcolor{light-gray}
     >  & Mayor que  & x > y  \pausa \\
     <  & Menor que  & x < y  \pausa \\
    \rowcolor{light-gray}
    >=  & Mayor o igual que  & x >= y  \pausa \\
    <=  & Menor o igual que  & x <= y  \\
  \end{tabular}
  \end{table}
\end{frame}

\begin{frame}[c]{Operadores Lógicos}

  Los operadores lógicos son usados para combinar sentencias condicionales:

  \begin{table}[]
  \begin{tabular}{cll}
    \textbf{Operador} &  \textbf{Descripción} & \textbf{Ejemplo} \\
    \rowcolor{light-gray}
    and & Regresa Verdadero si ambas & calif > 5 and calif < 11  \\
    \rowcolor{light-gray}
        & sentencias son verdaderas  & \pausa \\
    or  & Regresa Verdadero si una de &  calif < 6 or faltas > 6  \\
        & las sentencias es verdadera & \pausa \\
    \rowcolor{light-gray}
    not & Revierte el resultado, & not(calif > 5 and calif < 11) \\
    \rowcolor{light-gray}
        & regresa Falso si el    & \\
    \rowcolor{light-gray}
        & resultado es verdadero & \\
  \end{tabular}
  \end{table}
\end{frame}

\begin{frame}[c]{Operadores de Identidad}

  Los operadores de identidad se utilizan para comparar los objetos,
  no si son iguales, sino si en realidad son el mismo objeto, con la
  misma ubicación de memoria:

  \begin{table}[]
  \begin{tabular}{cll}
    \textbf{Operador} &  \textbf{Descripción} & \textbf{Ejemplo} \\
    \rowcolor{light-gray}
    is  & Regresa Verdadero si ambas & x is y  \\
    \rowcolor{light-gray}
        & variables son el mismo objeto  & \pausa \\
    is not  & Regresa Verdadero si ambas &  x is not y \\
        & variables No son el mismo objeto & \\
  \end{tabular}
  \end{table}
\end{frame}

\begin{frame}[c]{Operadores de Pertenencia}

  Los operadores de pertenencia se utilizan para probar si una secuencia
  esta presente en un objeto:

  \begin{table}[]
  \begin{tabular}{cll}
    \textbf{Operador} &  \textbf{Descripción} & \textbf{Ejemplo} \\
    \rowcolor{light-gray}
    in  & Regresa Verdadero si una secuencia & x in y  \\
    \rowcolor{light-gray}
        & con un valor especifico esta & \\
    \rowcolor{light-gray}
        & presente en el objeto & \pausa \\
    not in  & Regresa Verdadero si una secuancia &  x not in y \\
        & con un valor especifico No esta & \\
        & presente en el objeto & \\
  \end{tabular}
  \end{table}
\end{frame}

\begin{frame}[c]{Operadores a nivel de Bits (\emph{Bitwise})}

  Los operadores bit a bit (\emph{Bitwise}) se utilizan para
  comparar números a nivel binario:

  \begin{table}[]
  \begin{tabular}{cll}
    \textbf{Operador} &  \textbf{Nombre} & \textbf{Descripción} \\
    \rowcolor{light-gray}
    \&  & AND & Establece cada bit en 1 si ambos bits son 1 \pausa \\
    \textbar & OR & Establece cada bit en 1 si uno de los dos bits son 1 \pausa \\
    \rowcolor{light-gray}
    \^{} & XOR & Establece cada bit en 1 si sólo uno de los dos bits es 1 \pausa \\
    \~{} & NOT & Invierte todos los bits \pausa \\
    \rowcolor{light-gray}
    $<<$ & Desplazamiento & Desplaza a la izquierda agregando ceros \\
    \rowcolor{light-gray}
         & a la izquierda & a la derecha y deja que los bits más a \\
    \rowcolor{light-gray}
         &  & la izquierda se caigan/pierdan \pausa \\
    $>>$ & Desplazamiento & Desplaza a la derecha poniendo copias del \\
         & a la derecha &  bit más significativo a la izquierda y \\
         & con signo & deja que los bits de la derecha se caigan/pierdan \\
 \end{tabular}
  \end{table}
\end{frame}

\begin{frame}[fragile]
  \frametitle{Operadores a nivel de Bits (\emph{Bitwise})}

  \begin{lstlisting}[language=Python]
x = 22

print("x      =       {:>6b}".format(x))
print("x & 4  = {:>3d} = {:>6b}".format(x & 4, x & 4))
print("x | 1  = {:>3d} = {:>6b}".format(x | 1, x | 1))
print("x ^ 4  = {:>3d} = {:>6b}".format(x ^ 4, x ^ 4))
print("~x     = {:>3d} = {:>6b}".format(~x , ~x))
print("x << 1 = {:>3d} = {:>6b}".format(x << 1, x << 1))
print("x >> 2 = {:>3d} = {:>6b}".format(x >> 1, x >> 1))
  \end{lstlisting}
\end{frame}


\begin{frame}[c]{Créditos}
  El material está basado en
  \begin{itemize}
    \item \href{https://es.wikipedia.org/wiki/Autenticaci\%C3\%B3n_de_m\%C3\%BAltiples_factores}
      {https://es.wikipedia.org/wiki/Autenticaci\%C3\%B3n\_de\_m\%C3\%BAltiples\_factores}
  \end{itemize}
\end{frame}

\end{document}
