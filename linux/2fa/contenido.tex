% ex: ts=2 sw=2 sts=2 et filetype=tex
% SPDX-License-Identifier: CC-BY-SA-4.0

\section{Autenticación de múltiples factores (MFA)}

\begin{frame}[c]{La autenticación de múltiples factores (AMF)}

  \begin{itemize}
    \item Más comúnmente conocida por sus siglas en inglés \textbf{MFA}
      (\emph{Multi Factor Authentication})

    \pausa
    \item Es un método de control de acceso informático en el que a un
      usuario se le concede acceso al sistema solo después de que
      presente dos o más pruebas diferentes de que es quien dice ser.

    \pausa
    \item Estas pruebas pueden ser diversas, como una contraseña, que
      posea una \textbf{clave secundaria rotativa}, o un \textbf{certificado
      digital} instalado en el equipo, \emph{biometría}, entre otros.
  \vspace{\baselineskip}
\end{frame}

\begin{frame}[c]{}
  \vspace{\baselineskip}
\end{frame}

\begin{frame}[fragile]
  \frametitle{}
  \vspace{\baselineskip}
  \begin{lstlisting}[language=Bash,numbers=none]
  \end{lstlisting}
\end{frame}
