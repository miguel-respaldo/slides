% ex: ts=2 sw=2 sts=2 et filetype=tex
% SPDX-License-Identifier: CC-BY-SA-4.0

\section{La filosofía de las herramientas de Unix}

\begin{frame}[c]{La filosofía de las herramientas de Unix}
  \begin{itemize}
    \item Una herramienta es un programa simple, generalmente diseñado para un
      propósito específico, a veces se le denomina como un comando.

    \pausa
    \item La filosofía de herramientas de Unix, surge durante la creación del
      sistema operativo UNIX, después de la revolucionaria invención del
      tubo (o tubería) '|'.

    \pausa
    \item La tubería permitía enviar la \textbf{salida} de un programa a la
      \textbf{entrada} de otro. La filosofía de las herramientas era tener
      \textbf{pequeños programas} para realizar una \textbf{tarea
      particular} en lugar de tratar de desarrollar grandes
      programas monolíticos para realizar una gran cantidad de tareas.

      Para realizar tareas más complejas, las herramientas simplemente se
      conectarían entre sí mediante tuberías.
  \end{itemize}
\end{frame}

\begin{frame}[c]{La filosofía de las herramientas de Unix}
  \begin{itemize}
    \item Todas las herramientas básicas del sistema UNIX se diseñaron para
      que pudieran operar juntas. Los editores originales basados en texto (e
      incluso TeX y LaTeX) usaban ASCII y ahora usan UTF-8 y puede usar
      herramientas como: \textbf{sed, awk, vi, grep, cat, more, tr} y
      varias otras herramientas basadas en texto junto con estos editores.

    \pausa
    \item Usando esta filosofía, los programadores evitaron escribir un
      programa (dentro de su programa más grande) que ya había sido escrito
      por otra persona (esto podría considerarse una forma de reciclaje de
      código). Por ejemplo, varias aplicaciones diferentes utilizan los
      correctores ortográficos de la línea de comandos en lugar de que cada
      aplicación cree su propio corrector ortográfico.

    \pausa
    \item Esta filosofía vive hoy en GNU/Linux y varios otros sistemas
      operativos basados en el sistema UNIX (FreeBSD, NetBSD, OpenBSD, etc.).
  \end{itemize}
\end{frame}

\section{Manuales y Ayuda}

\begin{frame}[c]{Comando man}
  \begin{description}
    \item[Nombre]
      \textbf{man} - interfaz de los manuales de referencia del sistema

    \vspace{\baselineskip}
    \item[Sinopsis]
      man [opciones de man] [[sección] página ...] ... \\
      man -K [opciones de man] [sección] term ... \\
      man -f [opciones de whatis] página ... \\

    \vspace{\baselineskip}
    \item[Descripción]
      man es el paginador de manuales del sistema.
      Cada argumento de página dado a man normalmente es el nombre de un
      programa, utilidad o función. La página de manual asociada con cada
      uno de estos argumentos es, pues, encontrada y mostrada.

    \vspace{\baselineskip}
    \item[Ejemplos]
      \begin{itemize}
        \item man ls
        \item man -K ssh
        \item man -f scp
      \end{itemize}
  \end{description}
\end{frame}

\begin{frame}[c]{Comando info}
  \begin{description}
    \item[Nombre]
      \textbf{info} - lee los documentos Info

    \vspace{\baselineskip}
    \item[Sinopsis]
      info [opciones] ... [menu-item ... ]

    \vspace{\baselineskip}
    \item[Descripción]
      Lee la documentación que se encuentra en formato Info.

    \vspace{\baselineskip}
    \item[Ejemplos]
      \begin{itemize}
        \item info emacs
        \item info -f ./archivo.info
      \end{itemize}
  \end{description}
\end{frame}

\begin{frame}[c]{Comando whatis}
  \begin{description}
    \item[Nombre]
      \textbf{whatis} - muestra descripciones de una línea de las páginas
      de manual

    \vspace{\baselineskip}
    \item[Sinopsis]
      whatis [-dlv?V] [-r|-w] [-s lista] [-m sistema[,...]] [-M ruta] [-L localización] [-C archivo] nombre ...
    \vspace{\baselineskip}
    \item[Descripción]
      Cada página de manual despone de una descripción breve. whatis busca
      nombres de página de manual y muestra las descripciones de página de
      cualquier nombre coincidente.

      El nombre puede contener comodines (-w) o puede ser una expresión
      regular (-r). Con estas opciones puede ser necesario entrecomillar el
      nombre o escapar (\textbackslash{}) los caracteres especiales para impedir que la
      shell los interprete.

    \vspace{\baselineskip}
    \item[Ejemplos]
      \begin{itemize}
        \item whatis ls
        \item whatis ssh
      \end{itemize}
  \end{description}
\end{frame}

\begin{frame}[c]{Comando apropos}
  \begin{description}
    \item[Nombre]
      \textbf{apropos} - busca nombres y descripciones de páginas de manual

    \vspace{\baselineskip}
    \item[Sinopsis]
      apropos [-dalv?V] [-e|-w|-r] [-s listado] [-m sistema[,...]] [-M ruta] [-L local] [-C archivo] palabraclave ...

    \vspace{\baselineskip}
    \item[Descripción]
      Cada página de manual contiene una breve descripción.
      apropos busca las descripciones de las instancias de palabra clave.

      Una palabraclave normalmente es una expresión regular, como si (-r)
      fuera empleada, o quizá contenga comodines (-w), o coincida con la
      palabra clave exacta (-e). Utilizando estas opciones, quizá sea
      necesario entrecomillar la palabraclave o escapar (\textbackslash{})
      los caracteres especiales para impedir que la shell los interprete.

    \vspace{\baselineskip}
    \item[Ejemplos]
      \begin{itemize}
        \item apropos ssh
      \end{itemize}
  \end{description}
\end{frame}


\section{Gestión de usuarios}

\begin{frame}[c]{}
\end{frame}

\begin{frame}[fragile]
  \frametitle{}

  \begin{lstlisting}[language=Bash]
  \end{lstlisting}
\end{frame}

\section{Herramientas relacionadas con el texto}

\begin{frame}[c]{}
\end{frame}

\begin{frame}[fragile]
  \frametitle{}

  \begin{lstlisting}[language=Bash]
  \end{lstlisting}
\end{frame}

\section{Comandos de red}
\section{Seguridad}
\section{Archivar archivos}

