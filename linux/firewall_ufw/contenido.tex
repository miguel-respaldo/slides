% ex: ts=2 sw=2 sts=2 et filetype=tex
% SPDX-License-Identifier: CC-BY-SA-4.0

\section{Cortafuegos "Firewalls" con UFW}

\begin{frame}[c]{Introducción}
  El kernel de Linux incluye el subsistema \textbf{Netfilter}, que se
  utiliza para manipular o decidir el destino del tráfico de red que se
  dirige hacia o a través de su servidor. Todas las soluciones modernas
  de cortafuegos de Linux utilizan este sistema para el filtrado de paquetes.
\end{frame}

\begin{frame}[c]{Introducción}
  El sistema de filtrado de paquetes del kernel sería de poca utilidad para
  los administradores sin una interfaz de espacio de usuario para
  administrarlo. Este es el propósito de \textbf{iptables}: cuando un paquete
  llega a su servidor, se entregará al subsistema \textbf{Netfilter} para su
  aceptación, manipulación o rechazo según las reglas proporcionadas desde
  el espacio de usuario a través de \emph{iptables}. Por lo tanto, iptables
  es todo lo que necesita para administrar su firewall, si está familiarizado
  con él, pero hay muchas interfaces disponibles para simplificar la tarea.
\end{frame}

\begin{frame}[c]{ufw - Cortafuegos sin complicaciones}
  La herramienta de configuración de firewall predeterminada para Ubuntu es
  \textbf{ufw}. Desarrollado para facilitar la configuración del cortafuegos
  de \emph{iptables}, \emph{ufw} proporciona una forma sencilla de crear un
  cortafuegos basado en host IPv4 o IPv6.

  \vspace{\baselineskip}
  \begin{exampleblock}{ufw}
    por defecto está inicialmente deshabilitado.
  \end{exampleblock}
\end{frame}

\begin{frame}[c]{ufw - Cortafuegos sin complicaciones}
  De la página de manual de ufw:

  \vspace{\baselineskip}
  “ufw no pretende proporcionar una funcionalidad de firewall completa a
  través de su interfaz de comandos, sino que proporciona una manera fácil
  de agregar o eliminar reglas simples.

  \vspace{\baselineskip}
  Actualmente se utiliza principalmente para cortafuegos basados en host”.
\end{frame}

\begin{frame}[fragile]
  \frametitle{ufw - Cortafuegos sin complicaciones}
  Los siguientes son algunos ejemplos de cómo usar ufw:
  \begin{itemize}
    \item Primero, ufw debe estar habilitado. Desde un indicador de terminal,
      ingrese:
      \begin{lstlisting}[language=Bash,numbers=none]
sudo ufw enable
      \end{lstlisting}
    \item Para abrir un puerto (SSH en este ejemplo):
      \begin{lstlisting}[language=Bash,numbers=none]
sudo ufw allow ssh
      \end{lstlisting}
      UFW registra el significado del puerto \textbf{allow ssh} porque está
      enumerado como servicio en el archivo \textbf{/etc/services}.

      Sin embargo, podemos escribir la regla equivalente especificando el
      puerto en vez del nombre del servicio. Por ejemplo, este comando
      funciona como el anterior:
      \begin{lstlisting}[language=Bash,numbers=none]
sudo ufw allow 22
      \end{lstlisting}
  \end{itemize}
\end{frame}

\begin{frame}[fragile]
  \frametitle{ufw - Cortafuegos sin complicaciones}
  \begin{itemize}
    \item Las reglas también se pueden agregar usando un formato
      \emph{numerado}:
      \begin{lstlisting}[language=Bash,numbers=none]
sudo ufw insert 1 allow 80
      \end{lstlisting}
    \item Del mismo modo, para cerrar un puerto abierto:
      \begin{lstlisting}[language=Bash,numbers=none]
sudo ufw deny 22
      \end{lstlisting}
    \item Para eliminar una regla, use eliminar seguido de la regla:
      \begin{lstlisting}[language=Bash,numbers=none]
sudo ufw delete deny 22
      \end{lstlisting}
  \end{itemize}
\end{frame}

\begin{frame}[fragile]
  \frametitle{ufw - Cortafuegos sin complicaciones}
  \begin{itemize}
    \item También es posible permitir el acceso desde hosts o redes
      específicas a un puerto. El siguiente ejemplo permite el acceso SSH
      desde el host \textbf{192.168.0.2} a cualquier dirección IP en
      este host:
      \begin{lstlisting}[language=Bash,numbers=none]
sudo ufw allow proto tcp from 192.168.0.2 to any port 22
      \end{lstlisting}
      Reemplace 192.168.0.2 con 192.168.0.0/24 para permitir el acceso
      SSH desde toda la subred.
  \end{itemize}
\end{frame}

\begin{frame}[fragile]
  \frametitle{ufw - Cortafuegos sin complicaciones}
  \begin{exampleblock}{Nota:}
    Si el puerto que desea abrir o cerrar está definido en
    \textbf{/etc/services},
    puede usar el nombre del puerto en lugar del número.
  \end{exampleblock}
  \begin{itemize}
    \item Agregar la opción –dry-run a un comando ufw generará las reglas
      resultantes, pero no las aplicará. Por ejemplo, lo siguiente es lo
      que se aplicaría si se abriera el puerto HTTP:
      \begin{lstlisting}[language=Bash,numbers=none]
sudo ufw --dry-run allow http
      \end{lstlisting}
  \end{itemize}
\end{frame}

\begin{frame}[fragile]
  \frametitle{ufw - Cortafuegos sin complicaciones}
  \begin{lstlisting}[language=Bash]
*filter
:ufw-user-input - [0:0]
:ufw-user-output - [0:0]
:ufw-user-forward - [0:0]
:ufw-user-limit - [0:0]
:ufw-user-limit-accept - [0:0]
### RULES ###

### tuple ### allow tcp 80 0.0.0.0/0 any 0.0.0.0/0
-A ufw-user-input -p tcp --dport 80 -j ACCEPT

### END RULES ###
-A ufw-user-input -j RETURN
-A ufw-user-output -j RETURN
-A ufw-user-forward -j RETURN
-A ufw-user-limit -m limit --limit 3/minute -j LOG --log-prefix "[UFW LIMIT]: "
-A ufw-user-limit -j REJECT
-A ufw-user-limit-accept -j ACCEPT
COMMIT
Rules updated
  \end{lstlisting}
\end{frame}

\begin{frame}[fragile]
  \frametitle{ufw - Cortafuegos sin complicaciones}
  \begin{itemize}
    \item ufw puede ser deshabilitado por:
      \begin{lstlisting}[language=Bash,numbers=none]
sudo ufw disable
      \end{lstlisting}
    \item Para ver el estado del firewall, ingrese:
      \begin{lstlisting}[language=Bash,numbers=none]
sudo ufw status
      \end{lstlisting}
    \item Y para obtener información de estado más detallada, use:
      \begin{lstlisting}[language=Bash,numbers=none]
sudo ufw status verbose
      \end{lstlisting}
    \item Para ver el formato numerado :
      \begin{lstlisting}[language=Bash,numbers=none]
sudo ufw status numbered
      \end{lstlisting}
  \end{itemize}
\end{frame}

\begin{frame}[c]{Integración de aplicaciones en ufw}
  Las aplicaciones que abren puertos pueden incluir un perfil ufw,
  que detalla los puertos necesarios para que la aplicación funcione
  correctamente. Los perfiles se mantienen en
  \textbf{/etc/ufw/applications.d} y
  se pueden editar si se han cambiado los puertos predeterminados.
\end{frame}

\begin{frame}[fragile]
  \frametitle{Integración de aplicaciones en ufw}
  \begin{itemize}
    \item Para ver qué aplicaciones han instalado un perfil,
      ingrese lo siguiente en una terminal:
      \begin{lstlisting}[language=Bash,numbers=none]
sudo ufw app list
      \end{lstlisting}
    \item Similar a permitir el tráfico a un puerto, el uso de un
      perfil de aplicación se logra ingresando:
      \begin{lstlisting}[language=Bash,numbers=none]
sudo ufw allow Samba
      \end{lstlisting}
    \item También está disponible una sintaxis extendida:
      \begin{lstlisting}[language=Bash,numbers=none]
sudo ufw allow from 192.168.0.0/24 to any app Samba
      \end{lstlisting}
      Reemplace Samba y 192.168.0.0/24 con el perfil de aplicación que
      está utilizando y el rango de IP para su red.
  \end{itemize}
\end{frame}

\begin{frame}[fragile]
  \frametitle{Integración de aplicaciones en ufw}

  \begin{block}{Nota:}
    No es necesario especificar el \emph{protocolo} para la aplicación,
    porque esa información se detalla en el perfil. Además, tenga en cuenta
    que el nombre de la \emph{aplicación} reemplaza el número
    de\emph{puerto}.
  \end{block}

  \begin{itemize}
    \item Para ver detalles sobre qué puertos, protocolos, etc.,
      están definidos para una aplicación, ingrese:
      \begin{lstlisting}[language=Bash,numbers=none]
sudo ufw app info Samba
      \end{lstlisting}
  \end{itemize}
\end{frame}

