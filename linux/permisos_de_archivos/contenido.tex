% ex: ts=2 sw=2 sts=2 et filetype=tex
% SPDX-License-Identifier: CC-BY-SA-4.0

\section{Permisos de archivos}

\begin{frame}[c]{Permisos de archivos}
  Aunque ya existen muchas funciones de seguridad buenas integradas en los
  sistemas basados en Linux, puede existir una \textbf{vulnerabilidad}
  potencial muy importante cuando se otorga el acceso local, es decir,
  problemas basados en \textbf{permisos de archivos} que resultan de un
  usuario que no asigna los \emph{permisos correctos} a los archivos y
  directorios.

  \vspace{\baselineskip}
  Entonces, en función de la necesidad de los permisos adecuados,
  veremos las formas de asignar permisos y se mostrará algunos ejemplos en
  los que puede ser necesaria la modificación.
\end{frame}

\begin{frame}[c]{Grupos de permisos}

  Cada archivo y directorio tiene tres grupos de permisos basados en usuarios:

  \vspace{\baselineskip}
  \begin{description}
    \item [propietario] Los permisos de propietario se aplican solo al
      propietario del archivo o directorio, no afectarán las acciones de
      otros usuarios.
    \pausa
    \item [grupo] Los permisos de grupo se aplican solo al grupo que se ha
      asignado al archivo o directorio, no afectarán las acciones de
      otros usuarios.
    \pausa
    \item [otros] Los permisos de otros o también llamados todos los usuarios
      se aplican a todos los demás usuarios del sistema; este es el grupo de
      permisos que más desea ver.
  \end{description}
\end{frame}

\begin{frame}[c]{Tipos de permisos}

  Cada archivo o directorio tiene tres tipos de permisos básicos:

  \vspace{\baselineskip}
  \begin{description}
    \item [lectura] El permiso de lectura se refiere a la capacidad de un
      usuario para leer el contenido del archivo.
    \pausa
    \item [escritura] Los permisos de escritura se refieren a la capacidad
      de un usuario para escribir o modificar un archivo o directorio.
    \pausa
    \item [ejecutar] Afecta la capacidad de un usuario
      para ejecutar un archivo o ver el contenido de un directorio.
  \end{description}
\end{frame}

\begin{frame}[fragile]
  \frametitle{Ver los permisos}

  Puede ver los permisos verificando los "permisos de archivo o directorio"
  en su Administrador de archivos GUI favorito o revisando el resultado del
  comando "\textbf{ls -l}" mientras está en la terminal y mientras trabaja
  en el directorio que contiene el archivo o la carpeta.

  \begin{lstlisting}[language=Bash]
usuario@linux ~ $ ls -l
total 0
-rw-r--r--. 1 usuario grupo  0 Aug 19 10:49 un-archivo.txt
-rw-rw-rw-. 1 usuario grupo  0 Aug 19 10:49 Un-archivo.txt
drwxr-xr-x. 2 usuario grupo 40 Aug 19 10:49 un-directorio
usuario@linux ~ $ ls -l /dev/null
crw-rw-rw-. 1 root root 1, 3 Aug 19 07:14 /dev/null
usuario@linux ~ $ ls -l /dev/sda
brw-rw----. 1 root disk 8, 0 Aug 19 07:14 /dev/sda
usuario@linux ~ $
  \end{lstlisting}
\end{frame}

\begin{frame}[c]{Ver los permisos}
  El permiso en la línea de comando se muestra como: \textbf{\_rwxrwxrwx 1
  propietario:grupo}

  \begin{enumerate}
    \item Derechos de usuario/Permisos
      \begin{enumerate}
        \item El primer carácter que se marco con un guión bajo es el indicador
          de tipo de archivo.
          \begin{description}
            \item [-] es un archivo ordinario
            \item [d] es un directorio
            \item [d] es un enlace simbólico
            \item [b] es un archivo/dispositivo de bloques
            \item [c] es un archivo/dispositivo de caracteres
            \item [s] es un archivo de socket
          \end{description}
        \item El siguiente conjunto de tres caracteres (rwx) es para los
          permisos de \textbf{propietario}.
        \item El segundo conjunto de tres caracteres (rwx) es para los
          permisos de \textbf{grupo}.
        \item El tercer conjunto de tres caracteres (rwx) es para los
          permisos de \textbf{otros} o \textbf{todos los usuarios}.
      \end{enumerate}
  \end{enumerate}
\end{frame}

\begin{frame}[c]{Ver los permisos}
  El permiso en la línea de comando se muestra como: \textbf{\_rwxrwxrwx 1
  propietario:grupo}

  \begin{enumerate}
    \setcounter{enumi}{1}
    \item Siguiendo esa agrupación ya que el número entero/número muestra el
      número de enlaces duros al archivo.
    \item La última parte es la asignación de propietario y grupo con el
      formato \textbf{Propietario:Grupo}.
  \end{enumerate}
\end{frame}

\begin{frame}[c]{Modificación de los permisos}
  Cuando está en la línea de comando, los permisos se editan usando el
  comando \textbf{chmod}. Puede asignar los permisos explícitamente o
  mediante una referencia binaria como se describe a continuación.
\end{frame}

\begin{frame}[c]{Definición explícita de permisos}
  Para definir permisos de forma explícita, deberá hacer referencia al
  grupo de permisos y los tipos de permisos.

  \vspace{\baselineskip}
  Los grupos de permisos utilizados son:
  \begin{description}
    \item [u] Propietario
    \item [g] Grupo
    \item [o] Otros
    \item [a] Todos los usuarios
  \end{description}

  \vspace{\baselineskip}
  Los operadores de asignación potenciales son \textbf{+} (más) y \textbf{–}
  (menos); estos se utilizan para decirle al sistema si debe agregar o
  eliminar los permisos específicos.
\end{frame}

\begin{frame}[c]{Definición explícita de permisos}

  Los tipos de permisos que se utilizan son:

  \begin{description}
    \item [r] Leer
    \item [w] Escribir
    \item [x] Ejecutar
  \end{description}

  \vspace{\baselineskip}
  Entonces, por ejemplo, supongamos que tengo un archivo llamado "archivo1"
  que actualmente tiene los permisos establecidos en \textbf{-rw-rw-rw-},
  lo que significa que el propietario, el grupo y otros tienen
  permiso de lectura y escritura.
\end{frame}

\begin{frame}[fragile]
  \frametitle{Definición explícita de permisos}
  Ahora queremos eliminar los permisos de lectura y escritura del grupo de
  todos los usuarios.

  \vspace{\baselineskip}
  Para realizar esta modificación, invocaría el comando:

  \begin{lstlisting}[language=Bash]
chmod a-rw archivo1
  \end{lstlisting}

  \pausa
  \vspace{\baselineskip}
  Para agregar los permisos anteriores, invocaría el comando:

  \begin{lstlisting}[language=Bash]
chmod a+rw archivo1
  \end{lstlisting}

  \vspace{\baselineskip}
  Como puede ver, si desea otorgar esos permisos, cambiaría el carácter
  menos por un signo más para agregar esos permisos.
\end{frame}

\begin{frame}[fragile]
  \frametitle{Uso de referencias binarias para establecer permisos}
  Para establecer el permiso usando referencias binarias, primero debe
  comprender que la entrada se realiza ingresando tres enteros/números.

  \vspace{\baselineskip}
  Una cadena de permiso de muestra sería
  \begin{lstlisting}[language=Bash]
chmod 640 archivo1
  \end{lstlisting}
  lo que significa que el propietario tiene permisos de lectura y escritura,
  el grupo tiene permisos de lectura y otros (todos los demás usuarios) no
  tienen derechos sobre el archivo.

  \vspace{\baselineskip}
  El primer número representa el permiso del propietario; el segundo
  representa los permisos del Grupo; y el último número representa los
  permisos para todos los demás usuarios.
\end{frame}

\begin{frame}[fragile]
  \frametitle{Definición explícita de permisos}
  Los números son una representación binaria de la cadena rwx.
  \begin{description}
    \item [r] = 4
    \item [w] = 2
    \item [x] = 1
  \end{description}

  \vspace{\baselineskip}
  Agrega los números para obtener el número entero representa los
  permisos que desea establecer. Deberá incluir los permisos binarios para
  cada uno de los tres grupos de permisos.

  Entonces, para configurar un archivo llamado arhivo1 con permisos
  \textbf{-rwxr-----} , debe ingresar el comando
  \begin{lstlisting}[language=Bash]
chmod 740 archivo1
  \end{lstlisting}
\end{frame}

\begin{frame}[fragile]
  \frametitle{Propietarios y Grupos}
  El propietario como su nombre lo indica es el usuario al que pertenece el
  archivo y el grupo es el nombre del grupo al que pertenece el archivo o
  directorio.

  Utiliza el comando \textbf{chown} para cambiar las asignaciones de
  propietario y grupo, la sintaxis es simple
  \begin{lstlisting}[language=Bash]
chown  propietario:grupo  nombre-de-archivo
  \end{lstlisting}
  por lo tanto, para cambiar el propietario del archivo1 a usuario1 y el grupo
  a familia, debe ingresar
  \begin{lstlisting}[language=Bash]
chown  usuario1:familia  archivo1
  \end{lstlisting}
\end{frame}

\begin{frame}[c]{Permisos especiales de Setuid/Setgid}
  Los permisos setuid/setguid se utilizan para indicarle al sistema que
  ejecute un ejecutable como propietario con los permisos del propietario
  del archivo.

  \vspace{\baselineskip}
  Tenga cuidado al usar los bits setuid/setgid en los permisos. Si asigna
  permisos incorrectamente a un archivo propiedad de root con el conjunto
  de bits setuid/setgid, entonces puede abrir su sistema a la intrusión.
\end{frame}

\begin{frame}[fragile]
  \frametitle{Permisos especiales de Setuid/Setgid}
  Solo puede asignar el bit setuid/setgid definiendo explícitamente los
  permisos. El carácter del bit setuid/setguid es "\textbf{s}" en la parte de
  ejecución de los permisos de propietario o grupo.

  \vspace{\baselineskip}
  Para que configure el bit setuid/setguid en "script1.sh" 

  \begin{lstlisting}[language=Bash]
chmod g+s script1.sh
  \end{lstlisting}
\end{frame}

\begin{frame}[fragile]
  \frametitle{Permisos especiales de Sticky Bit}
  El sticky bit puede ser muy útil en un entorno compartido porque cuando
  se ha asignado a los permisos en un directorio, lo configura para que solo
  el propietario del archivo pueda cambiar el nombre o eliminar dicho archivo.

  \vspace{\baselineskip}
  Solo puede asignar el sticky bit definiendo explícitamente los permisos.
  El carácter del sticky bit es "\textbf{t}" en la parte ejecutable de los
  permisos de otros usuarios.

  \vspace{\baselineskip}
  Para establecer el sticky bit en un directorio llamado dir1, debe ejecutar
  el comando
  \begin{lstlisting}[language=Bash]
chmod +t dir1
  \end{lstlisting}
\end{frame}

\begin{frame}[c]{Cuando los permisos son importantes}

  Para algunos usuarios de computadoras basadas en Mac o Windows,
  usted no piensa en los permisos, pero esos entornos no se enfocan tan
  agresivamente en los derechos basados en el usuario sobre los archivos,
  a menos que se encuentre en un entorno corporativo. Pero ahora está
  ejecutando un sistema basado en Linux y la seguridad basada en permisos
  se simplifica y se puede usar fácilmente para restringir el acceso a su
  gusto.

  \vspace{\baselineskip}
  Entonces, se mostrarán algunos documentos y carpetas en los que desea
  enfocarse y se mostrarán cómo se deben configurar los permisos óptimos.
\end{frame}

\begin{frame}[fragile]
  \frametitle{Cuando los permisos son importantes}
  \begin{itemize}
    \item \textbf{Directorios de inicio}/\textbf{home directory}: los
      directorios de inicio de los usuarios son importantes, porque no desea
      que otros usuarios puedan ver y modificar los archivos en los documentos
      de escritorio de otro usuario. Para remediar esto, querrá que el
      directorio tenga los permisos \textbf{drwx------} (700), así que
      digamos que queremos aplicar los permisos correctos en el directorio de
      inicio del "usuario1", lo que se puede hacer emitiendo el comando
      \begin{lstlisting}[language=Bash]
chmod 700 /home/usuario1
      \end{lstlisting}

    \item \textbf{Archivos de configuración del gestor de arranque}: si decide
      implementar una contraseña para arrancar sistemas operativos específicos,
      entonces querrá eliminar los permisos de lectura y escritura del archivo
      de configuración de todos los usuarios excepto root. Para ello puedes
      cambiar los permisos del archivo a \textbf{700}.
  \end{itemize}
\end{frame}

\begin{frame}[c]{Cuando los permisos son importantes}
  \begin{itemize}
    \item \textbf{Archivos de configuración del sistema y de demonios}: es muy
      importante restringir los derechos a los archivos de configuración del
      sistema y del demonio para impedir que los usuarios editen el contenido.
      Puede que no sea aconsejable restringir los permisos de lectura, pero es
      obligatorio restringir los permisos de escritura. En estos casos puede
      ser mejor modificar los derechos a \textbf{644}.

    \item \textbf{scripts de firewall}: puede que no siempre sea necesario
      bloquear a todos los usuarios para que no lean el archivo de firewall,
      pero es recomendable restringir que los usuarios escriban en el archivo
      En este caso, el usuario raíz ejecuta automáticamente la secuencia de
      comandos del cortafuegos en el arranque, por lo que todos los demás
      usuarios no necesitan derechos, por lo que puede asignar los
      \textbf{700} permisos.
  \end{itemize}
\end{frame}
