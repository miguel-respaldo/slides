% ex: ts=2 sw=2 sts=2 et filetype=tex
% SPDX-License-Identifier: CC-BY-SA-4.0


\section{BIOS, UEFI/EFI}

\begin{frame}[c]{¿Qué es la BIOS?}
  \begin{columns}
    \column{0.3\textwidth}
      \begin{center}
        \includegraphics[scale=1.0]{lfs/amiBIOS.jpg}
      \end{center}
    \column{0.7\textwidth}
    \begin{itemize}
      \item El sistema básico de entrada-salida o \textbf{BIOS}
        (del inglés \textbf{Basic Input/Output System}) es un estándar de
        facto que define la interfaz de \textbf{firmware} para computadoras
        IBM PC compatibles.
      \pausa
      \item El firmware del BIOS es instalado dentro de la computadora
        personal (PC), y es el primer programa que se ejecuta cuando se
        \emph{enciende la computadora}.
      \pausa
      \item El propósito fundamental del BIOS es \textbf{iniciar}, y
        \textbf{probar el hardware} del sistema y cargar un \emph{gestor de
        arranque} o un sistema operativo desde un dispositivo de
        almacenamiento de datos. 
    \end{itemize}
  \end{columns}
\end{frame}

\begin{frame}[c]{¿Qué es un firmware?}
  \begin{itemize}
    \item El \textbf{firmware} o \textbf{soporte lógico inalterable} es u
      programa informático que establece la lógica de más \emph{bajo nivel}
      que controla los \emph{circuitos electrónicos} de un dispositivo de
      cualquier tipo.
    \pausa
    \item Está fuertemente integrado con la electrónica del dispositivo.
    \pausa
    \item Un \emph{firmware} es un software que maneja físicamente al
      \textbf{hardware}.
  \end{itemize}
\end{frame}

\begin{frame}[c]{¿Qué es la UEFI/EFI?}
  \begin{columns}
    \column{0.4\textwidth}
      \begin{center}
        \includegraphics[scale=0.45]{lfs/uefi_pila.png}
      \end{center}
    \column{0.6\textwidth}
    \begin{itemize}
      \item La \textbf{Unified Extensible Firmware Interface} (UEFI, "interfaz
        unificada de firmware extensible" en español) es una especificación
        que define una interfaz entre el sistema operativo y el firmware.
    \end{itemize}
  \end{columns}
\end{frame}

\begin{frame}[c]{¿Qué es la UEFI/EFI?}
  \begin{columns}
    \column{0.3\textwidth}
      \begin{center}
        \includegraphics[scale=0.5]{lfs/uefi.png}
      \end{center}
    \column{0.7\textwidth}
    \begin{itemize}
      \item UEFI reemplaza la antigua interfaz del Sistema Básico de Entrada
        y Salida (BIOS) estándar presentado en las computadoras personales
        IBM PC como IBM PC ROM BIOS. 
      \pausa
      \item La interfaz UEFI incluye \textbf{bases de datos} con información
        de la plataforma, \textbf{inicio y tiempo de ejecución} de los
        servicios disponibles listos para cargar el sistema operativo.
    \end{itemize}
  \end{columns}
\end{frame}

\begin{frame}[c]{Unified Extensible Firmware Interface}
  UEFI destaca principalmente por:
    Diseño modular.
  \begin{itemize}
    \item Compatibilidad y emulación del BIOS para los sistemas operativos
      solo compatibles con esta última.
    \pausa
    \item Soporte completo para la \textbf{Tabla de particiones GUID} (GPT),
      se pueden crear hasta 128 particiones por disco, con una capacidad
      total de 8 ZB.
    \pausa
    \item Capacidad de arranque desde unidades de almacenamiento grandes,
      dado que no sufren de las limitaciones del MBR.
    \pausa
    \item Independiente de la arquitectura y controladores de la CPU.
    \pausa
    \item Entorno amigable y flexible Pre-Sistema Operativo, incluyendo
      capacidades de red.
  \end{itemize}
\end{frame}

\begin{frame}[c]{Unified Extensible Firmware Interface}
  La EFI hereda las nuevas características avanzadas del BIOS como
  \textbf{ACPI} (Interfaz Avanzada de Configuración y Energía) y el
  \textbf{SMBIOS} (Sistema de Gestión de BIOS), y se le pueden añadir
  muchas otras, ya que el entorno se ejecuta en \textbf{64 bits} y no en
  16 bits, como su predecesora.
\end{frame}

\section{Tabla de particiones}

\begin{frame}[c]{MBR: Master Boot Record}
\end{frame}

\section{Sistemas de almacenamiento}

\begin{frame}[fragile]
  \frametitle{Sintaxis}
  %\begin{lstlisting}[language=Bash]
  %\end{lstlisting}

  \vspace{\baselineskip}
\end{frame}
