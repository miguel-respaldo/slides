% ex: ts=2 sw=2 sts=2 et filetype=tex
% SPDX-License-Identifier: CC-BY-SA-4.0

\section{Introducción a JavaScript}

\begin{frame}[c]{¿Qué es JavaScript?}
  \textbf{JavaScript} (abreviado comúnmente \textbf{JS}) es un lenguaje de
  programación interpretado, dialecto del estándar ECMAScript. Se define
  como orientado a objetos, basado en prototipos, imperativo, débilmente
  tipado y dinámico. 

  \vspace{\baselineskip}
  Se utiliza principalmente del lado del cliente, implementado como parte
  de un navegador web permitiendo mejoras en la interfaz de usuario y
  páginas web dinámicas y JavaScript del lado del servidor (Server-side
  JavaScript o SSJS).
\end{frame}

\begin{frame}[c]{¿Qué es JavaScript?}

  JavaScript se diseñó con una sintaxis similar a C, aunque adopta nombres
  y convenciones del lenguaje de programación Java. Sin embargo, Java y
  JavaScript tienen semánticas y propósitos diferentes.

  \vspace{\baselineskip}
  Tradicionalmente se venía utilizando en páginas web HTML para realizar
  operaciones y únicamente en el marco de la aplicación cliente, sin acceso
  a funciones del servidor.

  \vspace{\baselineskip}
  Actualmente es ampliamente utilizado para enviar y recibir información del
  servidor junto con ayuda de otras tecnologías como \textbf{AJAX}.

\end{frame}

\begin{frame}[c]{¿Qué es AJAX?}

  \textbf{AJAX}, acrónimo de \textbf{A}synchronous \textbf{J}avaScript
  \textbf{A}nd \textbf{X}ML (JavaScript asíncrono y XML), es una técnica
  de desarrollo web para crear aplicaciones web asíncronas.

  \vspace{\baselineskip}
  Estas aplicaciones se ejecutan en el cliente, es decir, en el navegador
  de los usuarios mientras se mantiene la comunicación asíncrona con el
  servidor en segundo plano.

  \vspace{\baselineskip}
  De esta forma es posible interactuar con el
  servidor sin necesidad de recargar la página web, mejorando la
  interactividad, velocidad y usabilidad en las aplicaciones.
\end{frame}
