% ex: ts=2 sw=2 sts=2 et filetype=tex
% SPDX-License-Identifier: CC-BY-SA-4.0

\section{Arreglos en JavaScript}

\begin{frame}[fragile]
  \frametitle{Arreglos/Matrices de JavaScript} 

  Una arreglo o matriz es una variable especial,
  que puede contener más de un valor:

  \vspace{\baselineskip}
  \begin{lstlisting}
const carros = ["Saab", "Volvo", "BMW"];
document.getElementById("demo").innerHTML = carros;
  \end{lstlisting}
\end{frame}

\begin{frame}[fragile]
  \frametitle{¿Por qué utilizar matrices?}

  Si tiene una lista de elementos (una lista de nombres de automóviles,
  por ejemplo), almacenar los automóviles en variables individuales podría
  tener este aspecto:

  \vspace{\baselineskip}
  \begin{lstlisting}
let carro1 = "Saab";
let carro2 = "Volvo";
let carro3 = "BMW"; 
  \end{lstlisting}

  \vspace{\baselineskip}
  Sin embargo, ¿qué sucede si desea recorrer los autos y encontrar uno
  específico? ¿Y si no tuvieras 3 coches, sino 300?

  \vspace{\baselineskip}
  ¡La solución es un arreglo/matriz!

  \vspace{\baselineskip}
  Una matriz puede contener muchos valores bajo un solo nombre y puede
  acceder a los valores haciendo referencia a un número de índice.
\end{frame}

\begin{frame}[fragile]
  \frametitle{Crear un arreglo/matriz}

  El uso de un literal de matriz es la forma más fácil de
  crear una matriz de JavaScript.

  \vspace{\baselineskip}
  Sintaxis:
  \begin{lstlisting}
const nombre_arreglo = [ elemento1, elemento2, ... ];
  \end{lstlisting}

  \begin{block}{}
    Es una práctica común declarar arreglos con la palabra clave const.
  \end{block}
\end{frame}

\begin{frame}[fragile]
  \frametitle{Crear un arreglo/matriz}

  Los espacios y los saltos de línea no son importantes.
  Una declaración puede abarcar varias líneas:

  \vspace{\baselineskip}
  \begin{lstlisting}
const carros = [
  "Saab",
  "Volvo",
  "BMW"
]; 
  \end{lstlisting}

  \vspace{\baselineskip}
  También puede crear una matriz y luego proporcionar los elementos:

  \vspace{\baselineskip}
  \begin{lstlisting}
const carros = [];
carros[0]= "Saab";
carros[1]= "Volvo";
carros[2]= "BMW";
  \end{lstlisting}
\end{frame}

\begin{frame}[fragile]
  \frametitle{Uso de la palabra clave JavaScript new}

  El siguiente ejemplo también crea un \textbf{Array} y le asigna valores:

  \vspace{\baselineskip}
  \begin{lstlisting}
const carros = new Array("Saab", "Volvo", "BMW");
document.getElementById("demo").innerHTML = carros;
  \end{lstlisting}

  \begin{exampleblock}{}
    Los dos ejemplos anteriores hacen exactamente lo mismo.

    \vspace{\baselineskip}
    No hay necesidad de usar new Array().

    \vspace{\baselineskip}
    Por simplicidad, legibilidad y velocidad de ejecución,
    utilice el método literal de matriz.
  \end{exampleblock}
\end{frame}

\begin{frame}[fragile]
  \frametitle{Acceso a los elementos de la matriz}

  Accede a un elemento de matriz haciendo referencia al
  \textbf{número de índice}:

  \vspace{\baselineskip}
  \begin{lstlisting}
const carros = ["Saab", "Volvo", "BMW"];
document.getElementById("demo").innerHTML = carros[0];
  \end{lstlisting}

  \begin{exampleblock}{Nota:}
    Los índices de matriz comienzan con \textf{0}.

    \vspace{\baselineskip}
    [0] es el primer elemento. [1] es el segundo elemento.
  \end{exampleblock}
\end{frame}

\begin{frame}[fragile]
  \frametitle{Cambio de un elemento de matriz}

  Esta declaración cambia el valor del primer elemento en carros:

  \vspace{\baselineskip}
  \begin{lstlisting}
const cars = ["Saab", "Volvo", "BMW"];
cars[0] = "Opel";
document.getElementById("demo").innerHTML = cars;
  \end{lstlisting}
\end{frame}

\begin{frame}[fragile]
  \frametitle{Acceda a la matriz completa}

  Con JavaScript, se puede acceder a la matriz completa
  haciendo referencia al nombre de la matriz:

  \vspace{\baselineskip}
  \begin{lstlisting}
const cars = ["Saab", "Volvo", "BMW"];
document.getElementById("demo").innerHTML = cars;
  \end{lstlisting}
\end{frame}

\begin{frame}[fragile]
  \frametitle{Las matrices son objetos}

  Las matrices son un tipo especial de objetos.
  El operador \textbf{typeof} en JavaScript devuelve "objeto" para
  arreglos/matrices.

  \vspace{\baselineskip}
  Pero, los arreglos de JavaScript se describen mejor como lista de
  elementos.

  \vspace{\baselineskip}
  Las matrices usan números para acceder a sus "elementos".
  En este ejemplo, persona[0] devuelve Juan:

  \vspace{\baselineskip}
  \begin{lstlisting}
const persona = ["Juan", "Pérez", 16];
document.getElementById("demo").innerHTML = persona[0];
  \end{lstlisting}
\end{frame}

\begin{frame}[fragile]
  \frametitle{Las matrices son objetos}

  Los objetos usan nombres para acceder a sus "miembros".
  En este ejemplo, persona.nombre devuelve Juan:

  \vspace{\baselineskip}
  \begin{lstlisting}
const persona = {nombre:"Juan", apellido:"Pérez", edad:16};
document.getElementById("demo").innerHTML = person.nombre;
  \end{lstlisting}
\end{frame}

\begin{frame}[fragile]
  \frametitle{Los elementos de matriz pueden ser objetos}

  Las variables de JavaScript pueden ser objetos.
  Las arreglos/matrices son tipos especiales de objetos.

  Debido a esto, puedes tener variables de diferentes
  tipos en el mismo \textbf{Array}.

  Puedes tener objetos en un \textbf{Array}. Puedes tener
  funciones en un \textbf{Array}. Puedes tener arreglos en
  un \textbf{Array}:

  \vspace{\baselineskip}
  \begin{lstlisting}
miArreglo[0] = Date.now;
miArreglo[1] = miFuncion;
miArreglo[2] = misCarros;
  \end{lstlisting}

  Después haremos ejercicios de este ejemplo.
\end{frame}

\begin{frame}[fragile]
  \frametitle{Propiedades y métodos de matriz}

  La verdadera fortaleza de las matrices de JavaScript
  son las propiedades y los métodos de matriz integrados:

  \vspace{\baselineskip}
  \begin{lstlisting}
carros.length   // Regresa el número de elementos
carros.sort()   // Ordena el arreglo
  \end{lstlisting}
\end{frame}

\begin{frame}[fragile]
  \frametitle{La propiedad de longitud}

  La propiedad \textbf{length} de una matriz/arreglo devuelve la
  longitud de una matriz/arreglo (el número de elementos de
  la matriz/arreglo).

  \vspace{\baselineskip}
  \begin{lstlisting}
const frutas = ["Banana", "Orange", "Apple", "Mango"];
document.getElementById("demo").innerHTML = frutas.length;
  \end{lstlisting}


  \begin{block}{}
    La propiedad \textbf{length} siempre es uno más que el índice
    de matriz más alto.
  \end{block}
\end{frame}

\begin{frame}[fragile]
  \frametitle{Acceso al primer elemento del arreglo}

  El primer elemento es el indice 0.

  \vspace{\baselineskip}
  \begin{lstlisting}
const frutas = ["Banana", "Orange", "Apple", "Mango"];
document.getElementById("demo").innerHTML = frutas[0];
  \end{lstlisting}
\end{frame}

\begin{frame}[fragile]
  \frametitle{Acceso al último elemento del arreglo}

  El último elemento es el tamaño del arreglo menos uno.

  \vspace{\baselineskip}
  \begin{lstlisting}
const frutas = ["Banana", "Orange", "Apple", "Mango"];
document.getElementById("demo").innerHTML = frutas[frutas.length-1];
  \end{lstlisting}
\end{frame}

\begin{frame}[fragile]
  \frametitle{Elementos de matriz en un ciclo/bucle}

  Una forma de recorrer una matriz es usando un ciclo/bucle
  \textbf{for}:

  \vspace{\baselineskip}
  \begin{lstlisting}
<script>
const frutas = ["Banana", "Orange", "Apple", "Mango"];
let fLen = frutas.length;

let text = "<ul>";
for (let i = 0; i < fLen; i++) {
  text += "<li>" + frutas[i] + "</li>";
}
text += "</ul>";

document.getElementById("demo").innerHTML = text;
</script>
  \end{lstlisting}
\end{frame}

\begin{frame}[fragile]
  \frametitle{Elementos de matriz en un ciclo/bucle}

  También puedes usar la función \textbf{Array.forEach()}:

  \vspace{\baselineskip}
  \begin{lstlisting}
<script>
const frutas = ["Banana", "Orange", "Apple", "Mango"];

let text = "<ul>";
frutas.forEach(miFuncion);
text += "</ul>";
document.getElementById("demo").innerHTML = text;

function miFuncion(value) {
  text += "<li>" + value + "</li>";
} 
</script>
  \end{lstlisting}
\end{frame}

\begin{frame}[fragile]
  \frametitle{}

  \vspace{\baselineskip}
  \begin{lstlisting}
  \end{lstlisting}
\end{frame}

\begin{frame}[fragile]
  \frametitle{}

  \vspace{\baselineskip}
  \begin{lstlisting}
  \end{lstlisting}
\end{frame}

\begin{frame}[fragile]
  \frametitle{}

  \vspace{\baselineskip}
  \begin{lstlisting}
  \end{lstlisting}
\end{frame}

\begin{frame}[fragile]
  \frametitle{}

  \vspace{\baselineskip}
  \begin{lstlisting}
  \end{lstlisting}
\end{frame}

\begin{frame}[fragile]
  \frametitle{}

  \vspace{\baselineskip}
  \begin{lstlisting}
  \end{lstlisting}
\end{frame}

\begin{frame}[fragile]
  \frametitle{}

  \vspace{\baselineskip}
  \begin{lstlisting}
  \end{lstlisting}
\end{frame}

\begin{frame}[fragile]
  \frametitle{}

  \vspace{\baselineskip}
  \begin{lstlisting}
  \end{lstlisting}
\end{frame}

\begin{frame}[fragile]
  \frametitle{}

  \vspace{\baselineskip}
  \begin{lstlisting}
  \end{lstlisting}
\end{frame}
