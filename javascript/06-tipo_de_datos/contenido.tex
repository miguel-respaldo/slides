% ex: ts=2 sw=2 sts=2 et filetype=tex
% SPDX-License-Identifier: CC-BY-SA-4.0

\section{Tipos de datos de JavaScript}

\begin{frame}[fragile]
  \frametitle{El concepto de tipos de datos}

  Las variables de JavaScript pueden contener diferentes tipos de datos:
  números, cadenas, objetos y más:

  \vspace{\baselineskip}
  \begin{lstlisting}
let largo = 16;                                 // Numero
let nombre = "Juan";                            // Cadena de texto
let objeto = {nombre:"Juan", apellido:"Pérez"}; // Objeto 
  \end{lstlisting}

  \begin{itemize}
    \item En programación, los tipos de datos son un concepto importante.
    \item Para poder operar sobre variables, es importante saber algo sobre el tipo.
    \item Sin tipos de datos, una computadora no puede resolver esto de manera segura:
  \end{itemize}
\end{frame}

\begin{frame}[fragile]
  \frametitle{El concepto de tipos de datos}

  \begin{lstlisting}
let x = 16 + "Volvo";
  \end{lstlisting}

  \vspace{\baselineskip}
  ¿Tiene algún sentido agregar "Volvo" a dieciséis?,
  ¿Producirá un error o producirá un resultado?

  JavaScript tratará el ejemplo anterior como:

  \vspace{\baselineskip}
  \begin{lstlisting}
let x = "16" + "Volvo";
  \end{lstlisting}

  \begin{block}{}
    Al agregar un número y una cadena, JavaScript tratará el
    número como una cadena.
  \end{block}
\end{frame}

\begin{frame}[fragile]
  \frametitle{El concepto de tipos de datos}
  \begin{lstlisting}
<!DOCTYPE html>
<html>
<body>

<h2>JavaScript</h2>

<p>Cuando se suman un número y una cadena de texto,
   JavaScript los tratará como texto</p>

<p id="demo"></p>

<script>
let x = 16 + "Volvo";
document.getElementById("demo").innerHTML = x;
</script>

</body>
</html>
  \end{lstlisting}
\end{frame}

\begin{frame}[fragile]
  \frametitle{El concepto de tipos de datos}

  ¿Cuál es el resultado de la siguiente expresión?

  \begin{lstlisting}
let x = "Volvo" + 16;
  \end{lstlisting}

  JavaScript evalúa las expresiones de izquierda a derecha.
  Diferentes secuencias pueden producir diferentes resultados:

  \vspace{\baselineskip}
¿Cuál es el resultado de la siguiente expresión?
  \begin{lstlisting}
let x = 16 + 4 + "Volvo";
  \end{lstlisting}

  \vspace{\baselineskip}
¿Cuál es el resultado de la siguiente expresión?
  \begin{lstlisting}
let x = "Volvo" + 16 + 4;
  \end{lstlisting}

  En el segundo ejemplo, JavaScript trata el 16 y el 4 como números,
  hasta que llega a "Volvo".

  \vspace{\baselineskip}
  En el tercer ejemplo, dado que el primer operando es una cadena,
  todos los operandos se tratan como cadenas.
\end{frame}

\begin{frame}[fragile]
  \frametitle{Los tipos de JavaScript son dinámicos}

  JavaScript tiene tipos dinámicos. Esto significa que la misma
  variable se puede utilizar para contener diferentes tipos de datos:

  \vspace{\baselineskip}
  \begin{lstlisting}
let x;           // ahora x es sin definir (undefined)
x = 5;           // ahora x es un Numero
x = "Juan";      // ahora x es una Cadena de Texto (String)
  \end{lstlisting}
\end{frame}

\begin{frame}[fragile]
  \frametitle{Cadenas JavaScript}

  Una cadena (o una cadena de texto) es una serie de
  caracteres como \textbf{"Juan Pérez"}.

  \vspace{\baselineskip}
  Las cadenas se escriben con comillas. Puede utilizar
  comillas simples o dobles:

  \vspace{\baselineskip}
  \begin{lstlisting}
let nombreDeCarro1 = "Volvo XC60";   // Usando comillas dobles
let nombreDeCarro2 = 'Volvo XC60';   // Usando comillas simples
  \end{lstlisting}
\end{frame}

\begin{frame}[fragile]
  \frametitle{Cadenas JavaScript}

  Puede usar comillas dentro de una cadena, siempre que no
  coincidan con las comillas que rodean la cadena:

  \vspace{\baselineskip}
  \begin{lstlisting}
let respuesta2 = "El se llama 'Juan'"; // Comilla simple dentro de dobles
let respuesta3 = 'El se llama "Juan"'; // Comilla doble dentro de simples
  \end{lstlisting}

  \vspace{\baselineskip}
  Aprenderá más sobre las cadenas más adelante.
\end{frame}

\begin{frame}[fragile]
  \frametitle{Números JavaScript}

  JavaScript tiene un solo tipo de números.

  \vspace{\baselineskip}
  Los números se pueden escribir con o sin decimales:

  \vspace{\baselineskip}
  \begin{lstlisting}
let x1 = 34.00;     // escrito con decimales
let x2 = 34;        // escrito sin decimales
  \end{lstlisting}
\end{frame}

\begin{frame}[fragile]
  \frametitle{Números JavaScript}

  Los números extra grandes o extra pequeños se pueden escribir
  con notación científica (exponencial):

  \vspace{\baselineskip}
  \begin{lstlisting}
let y = 123e5;      // 12300000 =  123 x 10^5
let z = 123e-5;     // 0.00123  = 123 x 10^(-5)
  \end{lstlisting}
\end{frame}

\begin{frame}[fragile]
  \frametitle{Booleanos de JavaScript}

  Los valores booleanos solo pueden tener dos valores:
  \textbf{true} o \textbf{false}.

  \vspace{\baselineskip}
  \begin{lstlisting}
let x = 5;
let y = 5;
let z = 6;
(x == y)       // regresa true
(x == z)       // regresa false 
  \end{lstlisting}

  Los booleanos se utilizan a menudo en pruebas condicionales.
\end{frame}

\begin{frame}[fragile]
  \frametitle{Booleanos de JavaScript}
  \begin{lstlisting}
<!DOCTYPE html>
<html>
<body>

<h2>Booleanos JavaScript</h2>
<p>Los valores booleanos solo pueden tener dos valores:
   true o false:</p>

<p id="demo"></p>
<script>
let x = 5;
let y = 5;
let z = 6;

document.getElementById("demo").innerHTML = (x == y) + "<br>" + (x == z);
</script>
</body>
</html>
  \end{lstlisting}
\end{frame}

\begin{frame}[fragile]
  \frametitle{Matrices/Arreglos de JavaScript}

  \begin{itemize}
    \item Las matrices/arreglos de JavaScript se escriben con corchetes.
    \item Los elementos de la matriz están separados por comas.
  \end{itemize}

  \vspace{\baselineskip}
  El siguiente código declara (crea) una matriz/arreglo llamada cars,
  que contiene tres elementos (nombres de automóviles):
  \begin{lstlisting}
const cars = ["Saab", "Volvo", "BMW"];
document.getElementById("demo").innerHTML = cars[0];
  \end{lstlisting}

  \vspace{\baselineskip}
  Los índices de matriz se basan en cero, lo que significa que el
  primer elemento es [0], el segundo es [1] y así sucesivamente.
\end{frame}

\begin{frame}[fragile]
  \frametitle{Objetos de JavaScript}

  \begin{itemize}
    \item Los objetos de JavaScript se escriben con llaves {}.
    \item Las propiedades de los objetos se escriben como pares
          nombre:valor, separados por comas.
  \end{itemize}

  \vspace{\baselineskip}
  \begin{lstlisting}
const persona = {nombre:"Juan", apellido:"Pérez", edad:50, color:"azul"};
  \end{lstlisting}

  El objeto (persona) en el ejemplo anterior tiene 4 propiedades:
  nombre, apellido, edad y color.
\end{frame}

\begin{frame}[fragile]
  \frametitle{Objetos de JavaScript}
  \begin{lstlisting}
<!DOCTYPE html>
<html>
<body>
<h2>Objetos de JavaScript</h2>

<p id="demo"></p>
<script>
const persona = {
  nombre : "Juan",
  apellido : "Pérez",
  edad     : 50,
  color    : "azul"
};

document.getElementById("demo").innerHTML =
persona.nombre + " tiene " + persona.edad + " años.";
</script>
</body>
</html>
  \end{lstlisting}
\end{frame}

\begin{frame}[fragile]
  \frametitle{El operador "typeof"}

  Puede utilizar el operador \textbf{typeof} de JavaScript para
  encontrar el tipo de una variable de JavaScript.

  El operador \textbf{typeof} devuelve el tipo de una variable o
  una expresión:

  \vspace{\baselineskip}
  \begin{lstlisting}
document.getElementById("demo").innerHTML = 
typeof "" + "<br>" +
typeof "Juan" + "<br>" + 
typeof "Juan Pérez";
  \end{lstlisting}

  \vspace{\baselineskip}
  \begin{lstlisting}
document.getElementById("demo").innerHTML = 
typeof 0 + "<br>" + 
typeof 314 + "<br>" +
typeof 3.14 + "<br>" +
typeof (3) + "<br>" +
typeof (3 + 4);
  \end{lstlisting}
\end{frame}

\begin{frame}[fragile]
  \frametitle{Indefinido}

  En JavaScript, una variable sin valor, tiene el valor \textbf{undefined}.
  El tipo también es \textbf{undefined}.

  \vspace{\baselineskip}
  \begin{lstlisting}
let car;  // El valor es sin definir y el tipo es sin definir
  \end{lstlisting}

  \vspace{\baselineskip}
  Cualquier variable se puede vaciar, estableciendo el valor en
  \textbf{undefined}. El tipo también será \textbf{undefined}.

  \vspace{\baselineskip}
  \begin{lstlisting}
car = undefined; // El valor ees sin definir y el tipo es sin definir
  \end{lstlisting}

\end{frame}

\begin{frame}[fragile]
  \frametitle{Valores vacíos}

  \begin{itemize}
    \item Un valor vacío no tiene nada que ver con undefined.
    \item Una cadena vacía tiene tanto un valor legal como un tipo.
  \end{itemize}

  \vspace{\baselineskip}
  \begin{lstlisting}
let car = "";    // El valor es "", el tipo de dato (typeof) es "string"
  \end{lstlisting}
\end{frame}
