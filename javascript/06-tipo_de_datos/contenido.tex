% ex: ts=2 sw=2 sts=2 et filetype=tex
% SPDX-License-Identifier: CC-BY-SA-4.0

\section{Tipos de datos de JavaScript}

\begin{frame}[fragile]
  \frametitle{El concepto de tipos de datos}

  Las variables de JavaScript pueden contener diferentes tipos de datos:
  números, cadenas, objetos y más:

  \vspace{\baselineskip}
  \begin{lstlisting}
let largo = 16;                                 // Numero
let nombre = "Juan";                            // Cadena de texto
let objeto = {nombre:"Juan", apellido:"Pérez"}; // Objeto 
  \end{lstlisting}

  En programación, los tipos de datos son un concepto importante.

  Para poder operar sobre variables, es importante saber algo sobre el tipo.

  Sin tipos de datos, una computadora no puede resolver esto de manera segura:

\end{frame}

\begin{frame}[fragile]
  \frametitle{El concepto de tipos de datos}

  \begin{lstlisting}
let x = 16 + "Volvo";
  \end{lstlisting}

  \vspace{\baselineskip}
  ¿Tiene algún sentido agregar "Volvo" a dieciséis?,
  ¿Producirá un error o producirá un resultado?

  JavaScript tratará el ejemplo anterior como:

  \vspace{\baselineskip}
  \begin{lstlisting}
let x = "16" + "Volvo";
  \end{lstlisting}

  \begin{block}{}
    Al agregar un número y una cadena, JavaScript tratará el
    número como una cadena.
  \end{block}
\end{frame}

\begin{frame}[fragile]
  \frametitle{El concepto de tipos de datos}
  \begin{lstlisting}
<!DOCTYPE html>
<html>
<body>

<h2>JavaScript</h2>

<p>Cuando se suman un número y una cadena de texto,
   JavaScript los tratará como texto</p>

<p id="demo"></p>

<script>
let x = 16 + "Volvo";
document.getElementById("demo").innerHTML = x;
</script>

</body>
</html>
  \end{lstlisting}
\end{frame}

\begin{frame}[fragile]
  \frametitle{El concepto de tipos de datos}

  \begin{lstlisting}
let x = "Volvo" + 16;
  \end{lstlisting}

  \vspace{\baselineskip}
  JavaScript evalúa las expresiones de izquierda a derecha.
  Diferentes secuencias pueden producir diferentes resultados:

  \vspace{\baselineskip}
¿Cuál es el resultado de la siguiente expresión?
  \begin{lstlisting}
let x = 16 + 4 + "Volvo";
  \end{lstlisting}

  \vspace{\baselineskip}
¿Cuál es el resultado de la siguiente expresión?
  \begin{lstlisting}
let x = "Volvo" + 16 + 4;
  \end{lstlisting}

  En el primer ejemplo, JavaScript trata el 16 y el 4 como números,
  hasta que llega a "Volvo".

  \vspace{\baselineskip}
  En el segundo ejemplo, dado que el primer operando es una cadena,
  todos los operandos se tratan como cadenas.
\end{frame}

\begin{frame}[fragile]
  \frametitle{Los tipos de JavaScript son dinámicos}

  JavaScript tiene tipos dinámicos. Esto significa que la misma
  variable se puede utilizar para contener diferentes tipos de datos:

  \vspace{\baselineskip}
  \begin{lstlisting}
let x;           // ahora x es sin definir (undefined)
x = 5;           // ahora x es un Numero
x = "John";      // ahora x es una Cadena de Texto (String)
  \end{lstlisting}
\end{frame}

\begin{frame}[fragile]
  \frametitle{Cadenas JavaScript}

  Una cadena (o una cadena de texto) es una serie de
  caracteres como \textbf{"Juan Pérez"}.

  Las cadenas se escriben con comillas. Puede utilizar
  comillas simples o dobles:
  \vspace{\baselineskip}
  \begin{lstlisting}
let nombreDeCarro1 = "Volvo XC60";   // Usando comillas dobles
let nombreDeCarro2 = 'Volvo XC60';   // Usando comillas simples
  \end{lstlisting}
\end{frame}

\begin{frame}[fragile]
  \frametitle{Cadenas JavaScript}

  Puede usar comillas dentro de una cadena, siempre que no
  coincidan con las comillas que rodean la cadena:

  \vspace{\baselineskip}
  \begin{lstlisting}
let respuesta2 = "El se llama 'Juan'"; // Comilla simple dentro de dobles
let respuesta3 = 'El se llama "Juan"'; // Comilla doble dentro de simples
  \end{lstlisting}

  \vspace{\baselineskip}
  Aprenderá más sobre las cadenas más adelante.
\end{frame}

\begin{frame}[fragile]
  \frametitle{Números JavaScript}

  JavaScript tiene un solo tipo de números.

  Los números se pueden escribir con o sin decimales:
  \vspace{\baselineskip}
  \begin{lstlisting}
let x1 = 34.00;     // escrito con desimales
let x2 = 34;        // escrito sin desimales
  \end{lstlisting}
\end{frame}

\begin{frame}[c]{}
\end{frame}

\begin{frame}[fragile]
  \frametitle{Números JavaScript}

  Los números extra grandes o extra pequeños se pueden escribir
  con notación científica (exponencial):

  \vspace{\baselineskip}
  \begin{lstlisting}
let y = 123e5;      // 12300000 =  123 x 10^5
let z = 123e-5;     // 0.00123 
  \end{lstlisting}
\end{frame}

\begin{frame}[fragile]
  \frametitle{Booleanos de JavaScript}

  \vspace{\baselineskip}
  \begin{lstlisting}
  \end{lstlisting}
\end{frame}

\begin{frame}[fragile]
  \frametitle{Matrices de JavaScript}

  \vspace{\baselineskip}
  \begin{lstlisting}
  \end{lstlisting}
\end{frame}

\begin{frame}[fragile]
  \frametitle{Objetos de JavaScript}

  \vspace{\baselineskip}
  \begin{lstlisting}
  \end{lstlisting}
\end{frame}

\begin{frame}[fragile]
  \frametitle{El operador "typeof"}

  \vspace{\baselineskip}
  \begin{lstlisting}
  \end{lstlisting}
\end{frame}

\begin{frame}[fragile]
  \frametitle{Indefinido}

  \vspace{\baselineskip}
  \begin{lstlisting}
  \end{lstlisting}
\end{frame}

\begin{frame}[fragile]
  \frametitle{Valores vacíos}

  \vspace{\baselineskip}
  \begin{lstlisting}
  \end{lstlisting}
\end{frame}

\begin{frame}[c]{}
\end{frame}

\begin{frame}[fragile]
  \frametitle{}

  \vspace{\baselineskip}
  \begin{lstlisting}
  \end{lstlisting}
\end{frame}

