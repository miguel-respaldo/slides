% ex: ts=2 sw=2 sts=2 et filetype=tex
% SPDX-License-Identifier: CC-BY-SA-4.0

\section{¿Dónde se codifica JavaScript?}

\begin{frame}[fragile]
  \frametitle{La etiqueta <script>}

  En HTML, el código JavaScript se inserta entre las etiquetas
  \textbf{<script>} y \textbf{</script>}.

  \vspace{\baselineskip}
  \begin{lstlisting}
<script>
document.getElementById("demo").innerHTML = "Mi primer JavaScript";
</script>
  \end{lstlisting}

  \begin{block}{}
    Los ejemplos antiguos de JavaScript pueden usar un atributo de tipo:
    \textbf{<script type="text/javascript">}.

    \vspace{\baselineskip}
    El atributo de tipo no es obligatorio. JavaScript es el lenguaje de
    secuencias de comandos predeterminado en HTML.
  \end{block}
\end{frame}

\begin{frame}[fragile]
  \frametitle{Ejemplo 1}
  \lstinputlisting{02-ejemplo01.html}
\end{frame}

\begin{frame}[c]{Funciones y eventos de JavaScript}

  Una función JavaScript es un bloque de código JavaScript,
  que se puede ejecutar cuando se "llama".

  \vspace{\baselineskip}
  Por ejemplo, se puede llamar a una función cuando ocurre un evento,
  como cuando el usuario hace clic en un botón.

  \vspace{\baselineskip}
  \begin{exampleblock}{}
    Aprenderás mucho más sobre funciones y eventos en
    capítulos posteriores.
  \end{exampleblock}
\end{frame}

\begin{frame}[c]{JavaScript en <head> o <body>}

  Puedes colocar cualquier número de secuencias de comandos
  en un documento HTML.

  \vspace{\baselineskip}
  Los scripts se pueden colocar en la sección \textbf{<body>} o
  \textbf{<head>} de una página HTML, o en ambas.
\end{frame}

\begin{frame}[c]{JavaScript en <head>}

  En el siguiente ejemplo, se coloca una función de JavaScript
  en la sección \textbf{<head>} de una página HTML.

  \vspace{\baselineskip}
  La función se invoca (llama) cuando se hace clic en un botón:
\end{frame}

\begin{frame}[fragile]
  \frametitle{Ejemplo 2}
  \lstinputlisting{02-ejemplo02.html}
\end{frame}

\begin{frame}[c]{JavaScript en <body>}

  En este ejemplo, se coloca una función de JavaScript en la
  sección <body> de una página HTML.

  \vspace{\baselineskip}
  La función se invoca (llama) cuando se hace clic en un botón:

  \vspace{\baselineskip}
  \begin{exampleblock}{Nota:}
    Colocar el script en la parte inferior del elemento <body>
    mejora la velocidad de visualización, porque la interpretación de
    los scripts ralentiza la visualización.
  \end{exampleblock}
\end{frame}

\begin{frame}[fragile]
  \frametitle{Ejemplo 3}
  \lstinputlisting{02-ejemplo03.html}
\end{frame}

\begin{frame}[fragile]
  \frametitle{JavaScript externo}

  Los scripts también se pueden colocar en archivos externos

  \vspace{\baselineskip}
  Los scripts externos son prácticos cuando se usa el mismo
  código en muchas páginas web diferentes.

  \vspace{\baselineskip}
  Los archivos JavaScript tienen la extensión de archivo \textbf{.js}.

  \vspace{\baselineskip}
  Para usar una secuencia de comandos externa, coloque el nombre del
  archivo de secuencia de comandos en el atributo src (fuente) de
  una etiqueta <script>:

  \vspace{\baselineskip}
  \begin{lstlisting}
<script src="miScript.js"></script>
  \end{lstlisting}
\end{frame}

\begin{frame}[c]{JavaScript externo}
  Puede colocar una referencia de secuencia de comandos externa
  en \textbf{<head>} o \textbf{<body>} como desee.

  El script se comportará como si estuviera ubicado exactamente
  donde se encuentra la etiqueta \textbf{<script>}.

  \begin{alertblock}{Nota:}
    Los scripts externos no pueden contener etiquetas \textbf{<script>}.
  \end{alertblock}
\end{frame}

\begin{frame}[fragile]
  \frametitle{JavaScript externo}

  Has un archivo llamado \textbf{miScript.js} que contenga:

  \vspace{\baselineskip}
  \begin{lstlisting}
function miFuncion() {
  document.getElementById("demo").innerHTML = "cambió el párrafo.";
}
  \end{lstlisting}
\end{frame}

\begin{frame}[fragile]
  \frametitle{Ejemplo 4}
  \lstinputlisting{02-ejemplo04.html}
\end{frame}

\begin{frame}[fragile]
  \frametitle{Ventajas de JavaScript externo}

  Colocar scripts en archivos externos tiene algunas ventajas:

  \vspace{\baselineskip}
  \begin{itemize}
    \item Separa HTML y código.
    \item Hace que HTML y JavaScript sean más fáciles de leer y mantener.
    \item Los archivos JavaScript almacenados en caché pueden
      acelerar la carga de la página
  \end{itemize}

  \vspace{\baselineskip}
  Para agregar varios archivos de script a una página,
  use varias etiquetas de script:

  \vspace{\baselineskip}
  \begin{lstlisting}
<script src="miScript1.js"></script>
<script src="miScript2.js"></script>
  \end{lstlisting}
\end{frame}

\begin{frame}[fragile]
  \frametitle{Referencias externas}

  Se puede hacer referencia a un script externo de 3 maneras diferentes:

  \vspace{\baselineskip}
  \begin{itemize}
    \item Con una URL completa (una dirección web completa)
    \item Con una ruta de archivo (como /js/)
    \item sin ningún camino
  \end{itemize}

  \vspace{\baselineskip}
  Este ejemplo usa una URL completa para vincular a myScript.js:

  \vspace{\baselineskip}
  \begin{lstlisting}
<script src="https://www.w3schools.com/js/myScript.js"></script>
  \end{lstlisting}
\end{frame}

\begin{frame}[fragile]
  \frametitle{Referencias externas}

  \vspace{\baselineskip}
  Este ejemplo utiliza una ruta de archivo para vincular a miScript.js:

  \vspace{\baselineskip}
  \begin{lstlisting}
<script src="/js/miScript.js"></script>
  \end{lstlisting}

  \vspace{\baselineskip}
  Este ejemplo no usa una ruta para vincular a myScript.js (el archivo esta
  en el mismo directorio que el HTML):

  \vspace{\baselineskip}
  \begin{lstlisting}
<script src="miScript.js"></script>
  \end{lstlisting}
\end{frame}

\section{Salida de JavaScript}

\begin{frame}[c]{Posibilidades de visualización de JavaScript}

  JavaScript puede "mostrar" datos de diferentes maneras:

  \begin{itemize}
    \item Escribiendo en un elemento HTML, usando \textbf{innerHTML}.
    \item Escribiendo en la salida HTML usando \textbf{document.write()}.
    \item Escribiendo en un cuadro de alerta, usando \textbf{window.alert()}.
    \item Escribiendo en la consola del navegador, usando
      \textbf{console.log()}.
  \end{itemize}
\end{frame}
