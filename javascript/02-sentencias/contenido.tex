% ex: ts=2 sw=2 sts=2 et filetype=tex
% SPDX-License-Identifier: CC-BY-SA-4.0

\section{¿Dónde se codifica JavaScript?}

\begin{frame}[fragile]
  \frametitle{La etiqueta <script>}

  En HTML, el código JavaScript se inserta entre las etiquetas
  \textbf{<script>} y \textbf{</script>}.

  \vspace{\baselineskip}
  \begin{lstlisting}
<script>
document.getElementById("demo").innerHTML = "Mi primer JavaScript";
</script>
  \end{lstlisting}

  \begin{block}{}
    Los ejemplos antiguos de JavaScript pueden usar un atributo de tipo:
    \textbf{<script type="text/javascript">}.

    \vspace{\baselineskip}
    El atributo de tipo no es obligatorio. JavaScript es el lenguaje de
    secuencias de comandos predeterminado en HTML.
  \end{block}
\end{frame}

\begin{frame}[fragile]
  \frametitle{Ejemplo 1}
  \lstinputlisting{02-ejemplo01.html}
\end{frame}

\begin{frame}[c]{Funciones y eventos de JavaScript}

  Una función JavaScript es un bloque de código JavaScript,
  que se puede ejecutar cuando se "llama".

  \vspace{\baselineskip}
  Por ejemplo, se puede llamar a una función cuando ocurre un evento,
  como cuando el usuario hace clic en un botón.

  \vspace{\baselineskip}
  \begin{exampleblock}{}
    Aprenderás mucho más sobre funciones y eventos en
    capítulos posteriores.
  \end{exampleblock}
\end{frame}

\begin{frame}[c]{JavaScript en <head> o <body>}

  Puedes colocar cualquier número de secuencias de comandos
  en un documento HTML.

  \vspace{\baselineskip}
  Los scripts se pueden colocar en la sección \textbf{<body>} o
  \textbf{<head>} de una página HTML, o en ambas.
\end{frame}

\begin{frame}[c]{JavaScript en <head>}

  En el siguiente ejemplo, se coloca una función de JavaScript
  en la sección \textbf{<head>} de una página HTML.

  \vspace{\baselineskip}
  La función se invoca (llama) cuando se hace clic en un botón:
\end{frame}

\begin{frame}[fragile]
  \frametitle{Ejemplo 2}
  \lstinputlisting{02-ejemplo02.html}
\end{frame}

\begin{frame}[c]{JavaScript en <body>}

  En este ejemplo, se coloca una función de JavaScript en la
  sección <body> de una página HTML.

  \vspace{\baselineskip}
  La función se invoca (llama) cuando se hace clic en un botón:

  \vspace{\baselineskip}
  \begin{exampleblock}{Nota:}
    Colocar el script en la parte inferior del elemento <body>
    mejora la velocidad de visualización, porque la interpretación de
    los scripts ralentiza la visualización.
  \end{exampleblock}
\end{frame}

\begin{frame}[fragile]
  \frametitle{Ejemplo 3}
  \lstinputlisting{02-ejemplo03.html}
\end{frame}

\begin{frame}[fragile]
  \frametitle{JavaScript externo}

  Los scripts también se pueden colocar en archivos externos

  \vspace{\baselineskip}
  Los scripts externos son prácticos cuando se usa el mismo
  código en muchas páginas web diferentes.

  \vspace{\baselineskip}
  Los archivos JavaScript tienen la extensión de archivo \textbf{.js}.

  \vspace{\baselineskip}
  Para usar una secuencia de comandos externa, coloque el nombre del
  archivo de secuencia de comandos en el atributo src (fuente) de
  una etiqueta <script>:

  \vspace{\baselineskip}
  \begin{lstlisting}
<script src="miScript.js"></script>
  \end{lstlisting}
\end{frame}

\begin{frame}[c]{JavaScript externo}
  Puede colocar una referencia de secuencia de comandos externa
  en \textbf{<head>} o \textbf{<body>} como desee.

  El script se comportará como si estuviera ubicado exactamente
  donde se encuentra la etiqueta \textbf{<script>}.

  \begin{alertblock}{Nota:}
    Los scripts externos no pueden contener etiquetas \textbf{<script>}.
  \end{alertblock}
\end{frame}

\begin{frame}[fragile]
  \frametitle{JavaScript externo}

  Has un archivo llamado \textbf{miScript.js} que contenga:

  \vspace{\baselineskip}
  \begin{lstlisting}
function miFuncion() {
  document.getElementById("demo").innerHTML = "cambió el párrafo.";
}
  \end{lstlisting}
\end{frame}

\begin{frame}[fragile]
  \frametitle{Ejemplo 4}
  \lstinputlisting{02-ejemplo04.html}
\end{frame}

\begin{frame}[fragile]
  \frametitle{Ventajas de JavaScript externo}

  Colocar scripts en archivos externos tiene algunas ventajas:

  \vspace{\baselineskip}
  \begin{itemize}
    \item Separa HTML y código.
    \item Hace que HTML y JavaScript sean más fáciles de leer y mantener.
    \item Los archivos JavaScript almacenados en caché pueden
      acelerar la carga de la página
  \end{itemize}

  \vspace{\baselineskip}
  Para agregar varios archivos de script a una página,
  use varias etiquetas de script:

  \vspace{\baselineskip}
  \begin{lstlisting}
<script src="miScript1.js"></script>
<script src="miScript2.js"></script>
  \end{lstlisting}
\end{frame}

\begin{frame}[fragile]
  \frametitle{Referencias externas}

  Se puede hacer referencia a un script externo de 3 maneras diferentes:

  \vspace{\baselineskip}
  \begin{itemize}
    \item Con una URL completa (una dirección web completa)
    \item Con una ruta de archivo (como /js/)
    \item sin ningún camino
  \end{itemize}

  \vspace{\baselineskip}
  Este ejemplo usa una URL completa para vincular a myScript.js:

  \vspace{\baselineskip}
  \begin{lstlisting}
<script src="https://www.w3schools.com/js/myScript.js"></script>
  \end{lstlisting}
\end{frame}

\begin{frame}[fragile]
  \frametitle{Referencias externas}

  \vspace{\baselineskip}
  Este ejemplo utiliza una ruta de archivo para vincular a miScript.js:

  \vspace{\baselineskip}
  \begin{lstlisting}
<script src="/js/miScript.js"></script>
  \end{lstlisting}

  \vspace{\baselineskip}
  Este ejemplo no usa una ruta para vincular a myScript.js (el archivo esta
  en el mismo directorio que el HTML):

  \vspace{\baselineskip}
  \begin{lstlisting}
<script src="miScript.js"></script>
  \end{lstlisting}
\end{frame}

\section{Salida de JavaScript}

\begin{frame}[c]{Posibilidades de visualización de JavaScript}

  JavaScript puede "mostrar" datos de diferentes maneras:

  \vspace{\baselineskip}
  \begin{itemize}
    \item Escribiendo en un elemento HTML, usando \textbf{innerHTML}.
    \item Escribiendo en la salida HTML usando \textbf{document.write()}.
    \item Escribiendo en un cuadro de alerta, usando \textbf{window.alert()}.
    \item Escribiendo en la consola del navegador, usando
      \textbf{console.log()}.
  \end{itemize}
\end{frame}

\begin{frame}[c]{Uso de HTML interno}

  Para acceder a un elemento HTML, JavaScript puede usar el
  método \textbf{document.getElementById(id)}.

  \vspace{\baselineskip}
  El atributo \textbf{id} define el elemento HTML.
  La propiedad \textbf{innerHTML} define el contenido HTML

  \begin{exampleblock}{Nota:}
    Cambiar la propiedad \textbf{innerHTML} de un elemento HTML
    es una forma común de mostrar datos en HTML.
  \end{exampleblock}
\end{frame}

\begin{frame}[fragile]
  \frametitle{Ejemplo 5}
  \lstinputlisting{02-ejemplo05.html}
\end{frame}

\begin{frame}[fragile]
  \frametitle{Usando document.write()}

  Para propósitos de prueba, es conveniente usar
  \textbf{document.write()}:

  \vspace{\baselineskip}
  \begin{lstlisting}
<script>
document.write(5 + 6);
</script>
  \end{lstlisting}
\end{frame}

\begin{frame}[fragile]
  \frametitle{Usando document.write()}

  \begin{alertblock}{Nota:}
    El uso de \textbf{document.write()} después de cargar un
    documento HTML eliminará todo el HTML existente:
  \end{alertblock}

  \vspace{\baselineskip}
  \begin{lstlisting}
<button type="button" onclick="document.write(5 + 6)">Borrar todo</button>
  \end{lstlisting}

  \begin{exampleblock}{Nota:}
    El método \textbf{document.write()} solo debe usarse para pruebas.
  \end{exampleblock}
\end{frame}

\begin{frame}[fragile]
  \frametitle{Usando window.alert()}

  Puede utilizar un cuadro de alerta para mostrar datos:

  \vspace{\baselineskip}
  \begin{lstlisting}
<script>
window.alert(5 + 6);
</script>
  \end{lstlisting}
\end{frame}

\begin{frame}[fragile]
  \frametitle{Usando window.alert()}

  Se puede omitir la palabra clave \textbf{window}.

  \vspace{\baselineskip}
  En JavaScript, el objeto de window es el objeto de alcance global,
  lo que significa que las variables, propiedades y métodos por
  defecto pertenecen al objeto de window. Esto también significa que
  especificar la palabra clave de la ventana es opcional:

  \vspace{\baselineskip}
  \begin{lstlisting}
<script>
alert(5 + 6);
</script>
  \end{lstlisting}
\end{frame}

\begin{frame}[fragile]
  \frametitle{Usando console.log()}

  Con fines de depuración, se puede llamar al método \textbf{console.log()}
  en el navegador para mostrar datos.

  \vspace{\baselineskip}
  \begin{lstlisting}
<script>
console.log(5 + 6);
</script>
  \end{lstlisting}

  \begin{exampleblock}{}
  Aprenderemos más sobre la depuración en un capítulo posterior.
  \end{exampleblock}
\end{frame}

\begin{frame}[fragile]
  \frametitle{Imprimir en JavaScript}

  JavaScript no tiene ningún objeto de impresión o métodos de impresión.

  \vspace{\baselineskip}
  No puede acceder a los dispositivos de salida (como una impresora)
  desde JavaScript

  \vspace{\baselineskip}
  La única excepción es que puede llamar al método
  \textbf{window.print()} en el navegador para imprimir el
  contenido de la ventana actual.

  \vspace{\baselineskip}
  \begin{lstlisting}
<button onclick="window.print()">Imprimir esta página</button>
  \end{lstlisting}
\end{frame}

\section{Sentencias JavaScript}

\begin{frame}[fragile]
  \frametitle{Programas JavaScript}

  Un \textbf{programa de computadora} es una lista de
  "instrucciones" para ser "ejecutadas" por una computadora.

  \vspace{\baselineskip}
  En un lenguaje de programación, estas instrucciones de
  programación se denominan \textbf{sentencias}.

  \vspace{\baselineskip}
  Un programa JavaScript es una lista de sentencias de programación.

  \vspace{\baselineskip}
  \begin{lstlisting}
let x, y, z;    // Sentencia 1
x = 5;          // Sentencia 2
y = 6;          // Sentencia 3
z = x + y;      // Sentencia 4
  \end{lstlisting}

  \begin{exampleblock}{}
  En HTML, los programas JavaScript son ejecutados por el navegador web.
  \end{exampleblock}
\end{frame}

\begin{frame}[fragile]
  \frametitle{Declaraciones JavaScript}

  Las declaraciones de JavaScript se componen de:

  \vspace{\baselineskip}
  Valores, operadores, expresiones, palabras clave y comentarios.

  \vspace{\baselineskip}
  Esta declaración le dice al navegador que escriba "Hola, Mundo".
  dentro de un elemento HTML con id="demo":

  \vspace{\baselineskip}
  \begin{lstlisting}
document.getElementById("demo").innerHTML = "Hola Mundo";
  \end{lstlisting}

  \vspace{\baselineskip}
  La mayoría de los programas de JavaScript contienen muchas
  declaraciones de JavaScript.

  Las declaraciones se ejecutan, una por una, en el mismo orden
  en que se escriben.

  \begin{exampleblock}{}
    Los programas de JavaScript (y las declaraciones de JavaScript) a menudo se denominan código de JavaScript.
  \end{exampleblock}
\end{frame}

\begin{frame}[fragile]
  \frametitle{punto y coma ";"}

  Los puntos y comas separan las declaraciones de JavaScript.

  \vspace{\baselineskip}
  Hay que agregar un punto y coma al final de cada instrucción ejecutable:

  \vspace{\baselineskip}
  \begin{lstlisting}
let a, b, c;  // Declará 3 variables
a = 5;        // Asigna el valor de 5 a la variable a
b = 6;        // Asigna el valor de 6 a la variable b
c = a + b;    // Asigna la suma de a y b a la variable c
  \end{lstlisting}
\end{frame}

\begin{frame}[fragile]
  \frametitle{punto y coma ";"}

  Cuando se separan por punto y coma,
  se permiten varias declaraciones en una línea:

  \vspace{\baselineskip}
  \begin{lstlisting}
 a = 5; b = 6; c = a + b;
  \end{lstlisting}

  \begin{block}{En la Web}
    es posible que vea ejemplos sin punto y coma.
    No es obligatorio terminar las declaraciones con punto y coma,
    pero es muy recomendable.
  \end{block}
\end{frame}

\begin{frame}[fragile]
  \frametitle{Espacios en blanco de JavaScript}

  JavaScript ignora múltiples espacios. Puede agregar espacios
  en blanco a su secuencia de comandos para que sea más legible.

  \vspace{\baselineskip}
  Las siguientes líneas son equivalentes:

  \vspace{\baselineskip}
  \begin{lstlisting}
let persona = "Maria";
let persona="Maria";
  \end{lstlisting}
\end{frame}

\begin{frame}[fragile]
  \frametitle{Espacios en blanco de JavaScript}

  Una buena práctica es poner espacios alrededor
  de los operadores ( = + - * / ):

  \vspace{\baselineskip}
  \begin{lstlisting}

let x = y + z;

  \end{lstlisting}
\end{frame}

\begin{frame}[fragile]
  \frametitle{Longitud de línea de JavaScript y saltos de línea}

  Para una mejor legibilidad, a los programadores a menudo
  les gusta evitar las líneas de código de más de 80 caracteres.

  \vspace{\baselineskip}
  Si una declaración de JavaScript no cabe en una línea,
  el mejor lugar para dividirla es después de un operador:

  \vspace{\baselineskip}
  \begin{lstlisting}

document.getElementById("demo").innerHTML =
"Hola Mundo!";

  \end{lstlisting}
\end{frame}

\begin{frame}[fragile]
  \frametitle{Bloques de código JavaScript}

  Las declaraciones de JavaScript se pueden agrupar
  en bloques de código, dentro de corchetes \textbf{\{}...\textbf{\}}.

  \vspace{\baselineskip}
  El propósito de los bloques de código es definir declaraciones
  que se ejecutarán juntas.

  \vspace{\baselineskip}
  Un lugar donde encontrará declaraciones agrupadas en bloques,
  es en funciones de JavaScript:

  \vspace{\baselineskip}
  \begin{lstlisting}
function miFuncion() {
  document.getElementById("demo1").innerHTML = "Hola, Don Pepito";
  document.getElementById("demo2").innerHTML = "Hola, Don Jose";
}
  \end{lstlisting}

  \vspace{\baselineskip}
  En el siguiente ejemplo se muestra el código completo.
\end{frame}

\begin{frame}[fragile]
  \frametitle{Ejemplo 6}
  \lstinputlisting{02-ejemplo06.html}
\end{frame}

\begin{frame}[c]{Bloques de código JavaScript}

  \begin{block}{En este curso}
    Usamos 2 espacios de sangría para bloques de código.
    Aprenderemos más sobre las funciones más adelante en este curso.
  \end{block}
\end{frame}

\begin{frame}[c]{Palabras clave de JavaScript}

  Las declaraciones de JavaScript a menudo comienzan con una palabra
  clave para identificar la acción de JavaScript que se va a realizar.

  \vspace{\baselineskip}
  Aquí hay una lista de algunas de las palabras clave que aprenderá
  en este curso:

  \begin{table}[]
  \begin{tabular}{ll}
    \textbf{Palabra clave} &  \textbf{Descripción} \\
    \rowcolor{light-gray}
    var 	 & Declara una variable \\
    let 	 & Declara una variable de bloque \\
    \rowcolor{light-gray}
    const  & Declara una constante de bloque \\
    if 	   & Marca un bloque de declaraciones para ser
             ejecutado en una condición \\
    \rowcolor{light-gray}
    switch & Marca un bloque de declaraciones para ser ejecutado en
             diferentes casos \\
    for 	 & Marca un bloque de sentencias para ser ejecutado en un bucle \\
    \rowcolor{light-gray}
    function & Declara una función \\
    return & Sale de una función \\
    \rowcolor{light-gray}
    try 	 & Implementa el manejo de errores en un bloque de declaraciones. \\
  \end{tabular}
  \end{table}
\end{frame}

\begin{frame}[c]{Palabras clave de JavaScript}
  \begin{exampleblock}{Nota:}
    Las palabras clave de JavaScript son palabras reservadas.
    Las palabras reservadas no se pueden utilizar como nombres de variables.
  \end{exampleblock}
\end{frame}


