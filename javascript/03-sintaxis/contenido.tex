% ex: ts=2 sw=2 sts=2 et filetype=tex
% SPDX-License-Identifier: CC-BY-SA-4.0

\section{Sintaxis}

\begin{frame}[fragile]
  \frametitle{Sintaxis de JavaScript}

  La sintaxis de JavaScript es el conjunto de reglas,
  cómo se construyen los programas de JavaScript:

  \vspace{\baselineskip}
  \begin{lstlisting}
// Como se crean las variables:
var x;
let y;

// Como se usan las variables:
x = 5;
y = 6;
let z = x + y;
  \end{lstlisting}
\end{frame}

\begin{frame}[c]{Valores de JavaScript}

  La sintaxis de JavaScript define dos tipos de valores:

  \begin{itemize}
    \item Valores \textbf{fijos}
    \item Valores \textbf{variables}
  \end{itemize}

  \vspace{\baselineskip}
  Los valores fijos se llaman \textbf{Literales}.

  \vspace{\baselineskip}
  Los valores de las variables se denominan \textbf{Variables}.
\end{frame}


\begin{frame}[fragile]
  \frametitle{Literales de JavaScript}

  Las dos reglas de sintaxis más importantes para valores fijos son:

  \begin{enumerate}
    \item Los \textbf{números} se escriben con o sin decimales:
          \vspace{\baselineskip}
          \begin{lstlisting}
10.50

1001
          \end{lstlisting}
    \item Las \textbf{cadenas} son texto, escrito entre comillas
          simples o dobles:
          \vspace{\baselineskip}
          \begin{lstlisting}
10.50

1001
          \end{lstlisting}
  \end{enumerate}
\end{frame}

\begin{frame}[fragile]
  \frametitle{Variables en JavaScript}

  En un lenguaje de programación, las variables se utilizan
  para \textbf{almacenar} valores de datos.

  \vspace{\baselineskip}
  JavaScript usa las palabras clave \textbf{var}, \textbf{let} y
  \textbf{const} para \textbf{declarar} variables.

  \vspace{\baselineskip}
  Se utiliza un \textbf{signo igual} para \textbf{asignar valores}
  a las variables.

  \vspace{\baselineskip}
  En este ejemplo, x se define como una variable.
  Entonces, a x se le asigna (se le da) el valor 6:

  \vspace{\baselineskip}
  \begin{lstlisting}
let x;

x = 6
  \end{lstlisting}
\end{frame}


\begin{frame}[fragile]
  \frametitle{Operadores en JavaScript}

  JavaScript usa \textbf{operadores aritméticos} ( + - * /) para
  \textbf{calcular} valores:

  \vspace{\baselineskip}
  \begin{lstlisting}
(5 + 6) * 10
  \end{lstlisting}

  \vspace{\baselineskip}
  JavaScript usa un \textbf{operador de asignación} ( = )
  para \textbf{asignar} valores a las variables:

  \vspace{\baselineskip}
  \begin{lstlisting}
let x, y;
x = 5;
y = 6;
  \end{lstlisting}
\end{frame}

\begin{frame}[fragile]
  \frametitle{Expresiones JavaScript}

  Una expresión es una combinación de valores,
  variables y operadores, que calcula un valor.

  \vspace{\baselineskip}
  El cálculo se llama \textbf{evaluación}.

  \vspace{\baselineskip}
  Por ejemplo, 5 * 10 se evalúa como 50:

  \vspace{\baselineskip}
  \begin{lstlisting}
5 * 10
  \end{lstlisting}

  \vspace{\baselineskip}
  Las expresiones también pueden contener valores de variables:

  \vspace{\baselineskip}
  \begin{lstlisting}
x * 10
  \end{lstlisting}
\end{frame}

\begin{frame}[fragile]
  \frametitle{Expresiones JavaScript}

  \vspace{\baselineskip}
  Los valores pueden ser de varios tipos, como números y cadenas.

  \vspace{\baselineskip}
  Por ejemplo, "Juan" + " " + "Pérez", se evalúa como "Juan Pérez":

  \vspace{\baselineskip}
  \begin{lstlisting}
"Juan" + " " + "Pérez"
  \end{lstlisting}
\end{frame}

\begin{frame}[fragile]
  \frametitle{Palabras clave de JavaScript}

  Las palabras clave de JavaScript se utilizan para
  identificar las acciones a realizar.

  \vspace{\baselineskip}
  La palabra clave \textbf{let} le dice al navegador que cree variables:

  \vspace{\baselineskip}
  \begin{lstlisting}
let x, y;
x = 5 + 6;
y = x * 10;
  \end{lstlisting}
\end{frame}

\begin{frame}[fragile]
  \frametitle{Palabras clave de JavaScript}

  La palabra clave \textbf{var} también le dice
  al navegador que cree variables:

  \vspace{\baselineskip}
  \begin{lstlisting}
var x, y;
x = 5 + 6;
y = x * 10;
  \end{lstlisting}

  \begin{block}{}
  En estos ejemplos, usar var o let producirá el mismo resultado.
  Aprenderás más sobre var y let más adelante en este curso.
  \end{block}
\end{frame}

\begin{frame}[fragile]
  \frametitle{}

  \vspace{\baselineskip}
  \begin{lstlisting}
  \end{lstlisting}
\end{frame}

\begin{frame}[fragile]
  \frametitle{}

  \vspace{\baselineskip}
  \begin{lstlisting}
  \end{lstlisting}
\end{frame}

